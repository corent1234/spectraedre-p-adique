\documentclass{amsart}

\usepackage{amsmath}
%\usepackage{amsthm}
%\usepackage{amssymb} %%% ça ne fait pas compiler en acmart
\usepackage{amsfonts}
\usepackage{mathrsfs} 
\usepackage{graphicx}
\usepackage{color,xcolor}
%\usepackage{ulem}
\usepackage{bm}

\usepackage[T1]{fontenc} 

%%%% pour le "def" sur \leftrightarrow
\makeatletter
\newcommand\xleftrightarrow[2][]{%
  \ext@arrow 9999{\Leftrightarrowfill@}{#1}{#2}}
\newcommand\longleftrightarrowfill@{%
  \arrowfill@\leftarrow\relbar\rightarrow}
\makeatother
%%%%%%%%%%%%%%%%%%%%%%%%%%%%%%%%%%%%%%%

\usepackage{tikz}
\usetikzlibrary{arrows.meta, positioning}

\usepackage{hyperref}
\usepackage{cleveref}

% --------- ALGORITHMS ---------- <<:
\usepackage{algorithm,algorithmic}
%\usepackage[noend]{algpseudocode}      %%% this is for the algorithms
\renewcommand{\algorithmicrequire}{\textbf{Input:}} % require --> input
\renewcommand{\algorithmicensure}{\textbf{Output:}} % ensure --> output
% :>> ALGORITHMS

\usepackage{stmaryrd} %for double brackets

\newcommand{\R}{\mathbb{R}} % real numbers
\newcommand{\C}{\mathbb{C}} % complex numbers
\newcommand{\N}{\mathbb{N}} % integers
%%%\newcommand{\allmat}{\mathbb{M}} % matrices ---> remplacé par K^{m \times n}
\newcommand{\sym}{\mathbb{S}} % matrices symétriques
\newcommand{\PP}{\mathcal{P}}
\newcommand{\Qc}{\mathcal{Q}}
\renewcommand{\span}[1]{{\text{span}(#1)}} % simone's comments
\newcommand{\aff}[1]{{\text{aff}(#1)}} % simone's comments
\newcommand{\relint}[1]{{\text{relint}(#1)}} % simone's comments
\newcommand{\calL}{\mathcal{L}} % simone's comments

\newcommand{\simone}[1]{{\color{blue} #1}} % simone's comments
\newcommand{\corentin}[1]{{\color{red} #1}} % corentin's comments
\newcommand{\tristan}[1]{{\color{red} #1}} % tristan's comments
\newcommand{\todo}[1]{{\color{blue}\bf \text{TODO: } #1}} % TODOs
\newcommand{\rev}[1]{{\color{blue} #1}} % TODOs

\newtheorem{theorem}{Theorem}[section]
\newtheorem{lemma}[theorem]{Lemma}
\newtheorem{proposition}[theorem]{Proposition}
\newtheorem{corollary}[theorem]{Corollary}
\newtheorem{conjecture}[theorem]{Conjecture}
\newtheorem{condition}[theorem]{Condition}
\newtheorem{definition}[theorem]{Definition}
\newtheorem{assumption}[theorem]{Assumption}
\newtheorem{remark}[theorem]{Remark}
\newtheorem{problem}[theorem]{Problem}
\newtheorem{example}[theorem]{Example}
\newtheorem{notation}[theorem]{Notation}

%p-adic commands
\DeclareMathOperator{\val}{val}
\def\QQ{\ensuremath{\mathbb{Q}}}
\def\ZZ{\ensuremath{\mathbb{Z}}}
\newcommand{\OK}{\mathcal{O}_K}
\def\diag{\mathrm{diag}}

\newcommand{\GL}{\mathrm{GL}}


\begin{document}

\title{Follow up on ISSAC paper}
\author{}

\maketitle

\begin{abstract}
This note is about further questions concerning the paper \cite{cornou2026}.
\end{abstract}

%
% ---- Bibliography ----
%
%\begin{thebibliography}{99}
%\end{thebibliography}

\section{Combinatorial aspects of polyhedra/spectrahedra}

In \cite[Def.\,3.1]{cornou2026} we define polyhedra in $K^n$, for a valued field $(K, \val)$, as the
set of vectors whose entries have valuation nonnegative or infinite and satisfying some
linear constraint, {\it i.e.} subsets of the form
\begin{equation}\label{eq_def_polyhedra}
  \PP =
  \left\{
  x \in K^n :
  \begin{array}{ll}
    \val(\ell_i(x)) \geq 0 & i=1,\ldots,d \\
    \val(m_j(x)) = +\infty & j=1,\ldots,e
  \end{array}
  \right\},
\end{equation}
for some affine forms $\ell_1,\ldots,\ell_d,m_1,\ldots,m_e \in K[x]_1$.
We say that a form $m$ vanishes on $x$ if $\val(m(x))=+\infty$.

Note that a given polyhedron $\PP$ might have different formulations
of type \eqref{eq_def_polyhedra}. To make it more canonical one can make
the following assumption (which is what we do from now on):

\begin{assumption}\label{assumpt}
  For every index $i \in \{1,\ldots,d\}$, there exists an element $x \in \PP$
  such that $\val(\ell_i(x)) < +\infty$; that is, $\ell_1,\ldots,\ell_d$
  do not vanish identically on $\PP$.
\end{assumption}

Nevertheless the linear forms $\ell_i$ will vanish at some $x$
in the <<boundary>> of $\PP$. This leads to the following definition:

\begin{definition}
  Let $\PP$ be defined as in \eqref{eq_def_polyhedra}, and let $x \in\PP$.
  The \emph{corank} of $x$ is
  $$
  \mathrm{corank}(x) = \#\Big\{i \in \{1,\ldots,d\} : \val(\ell_i(x)) = +\infty\Big\}.
  $$
  Its \emph{rank} is $d-\mathrm{corank}(x)$, the number of forms $\ell_i$ such
  that $\val(\ell_i(x)) < +\infty$. By \Cref{assumpt} $\mathrm{rank}(x)$
\end{definition}
Rank and corank of $x \in\PP$ depend on the description \eqref{eq_def_polyhedra}.


\section{The case $K=\QQ_p$}

\newpage

%\bibliographystyle{ACM-Reference-Format}
\bibliographystyle{acm}
\bibliography{../issac2026/refs}

\end{document}
