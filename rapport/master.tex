\documentclass[twoside,a4paper]{article}

\input{preamble.tex}
\addbibresource{bibstage.bib}

\title{
	Vers les Spectraèdres \texorpdfstring{$p$}{p}-adiques, définition et propriétés algorithmiques
}

\newcommand{\sage}{SageMath}
\newcommand\emp[1]{\textcolor{teal}{\textit{#1}}}
\newcommand{\complalgo}{\max\left( m,n \right)^2  }%complexité de l'algorithme
\begin{document}

    
\vspace*{\fill}
    \begin{center}
        \begin{figure}[!ht]
            \begin{minipage}{0.6\textwidth}
                \includegraphics[scale=0.20]{figures/logo_ens_ps.png}
            \end{minipage}
            \begin{minipage}{0.4\textwidth}
                \includegraphics[scale=0.3]{figures/logo_xlim.png}
            \end{minipage}
        \end{figure}
    
        \makeatletter
        {\large \textbf{\@title}}
        \\
    
    
        {\textsc{\@author}}
        \\
    
        \makeatother
    
        {
            Sous la supervision de \textsc{Tristan Vaccon} et  \textsc{Simone Naldi}
        }
    \end{center}

    \vspace*{\fill}

    \newpage

%\maketitle
    \section*{Abstract}

	Les spectraèdres sont une des généralisations les plus communes et utilisées des polyèdres. Ce sont des ensembles réels convexes définis comme les ensembles de matrices linéaires $A(x) = A_0 + x_1 A_1 + \ldots + x_s A_s$ qui sont symétriques définies positives, avec $x = \left( x_1,\ldots,x_s \right)$ parcourant $\in \mathbb{R}^s $ et $A_0,A_1,\ldots,A_s$ des matrices symétriques réelles fixées.
Ils sont d'une grande importance en optimisation car au cœur de la programmation semi-définie, une généralisation de la programmation linéaire visant à minimiser une forme linéaire sur un spectraèdre au lieu et en place d'un polyèdre convexe. Ainsi, une large variété de problèmes peuvent se réduire à la programmation semi-définie. En effet, aux nombreuses applications de la programmation linéaire (problème de la coupe maximale, nombreuses applications en finance, médecine, dans l'industrie\ldots) s'ajoutent des problèmes plus spécifiques dont un certain nombre est par exemple présentés dans \cite{vandenberghe_applications_1999}. Si les spectraèdres ont été largement étudiés dans le cas réel, en 2016  Allamigeon, Gaubert et Skomra ont ouvert la voie à l'étude des spéctraèdres sur les corps non-archimédien dans \cite{allamigeon_tropical_2020} en définissant les spectraèdres tropicaux.

L'objectif de ce stage était alors d'essayer de trouver une définition de spectraèdre générale sur les corps non-archimédien. (Rappelons qu'un corps (valué) non-archimédien est un corps muni d'une valeur absolue\footnote{une valeur absolue sur un corps est une norme sur ce même corps compatible avec le produit.} $\left| \cdot  \right|$ vérifiant l'inégalité non-archimédienne : pour tous $ x,y \in \mathbb{K}  \left| x+y \right|\le \max\left( \left| x \right|, \left| y \right|\right)$ ). En s'appuyant, pour ce faire, sur les corps $p$-adiques. Un corps $p$-adique $\mathbb{Q}_{p}$ étant un ensemble de nombres définis par les sommes formelles $x = \ldots + x_k p^k + \ldots + x_1 p + x_0 +x_{-1} p^{-1} + \ldots x_{n}p^n$ avec $n \in \mathbb{Z}$ et où les $(x_k)_{k\ge m}$ sont éléments de $\{0,\ldots,p-1\} $, pour $p$ un nombre premier fixé. On peut munir ces corps de la valuation $p$-adique $\val\left(x\right) = \sup \{i \ge m | x_{i}\neq 0\} $ qui définit une valeur absolue non-archimédienne $\left| \cdot  \right|_p : x\mapsto p^{-\val\left(x\right)} $. 

Ce rapport propose une définition de spectraèdre sur les corps $p$-adiques. Pour ce faire, on introduira une nouvelle notion, celle de matrice semi-définie positive. Une matrice à coefficients $p$-adiques étant semi-définie positive si et seulement si les racines de son polynôme caractéristiques sont de valuation positive. Ce qui permet de définir les spéctraèdres comme les matrices linéaires $A(x) = A_0 + x_1 A_1 + \ldots x_s A_s$ semi-définies positives avec $x = \left( x_1,\ldots,x_s \right)$ parcourant $\mathbb{Q}_{p} ^s$ et $A_0,\ldots,A_s$ quelconques fixées. On peut alors en déduire le principal résultat découvert lors de ce stage : les couronnes $p$-adiques sont des projections de spectraèdres. De plus, ce rapport définit les polyèdres $p$-adiques comme les ensembles de points $x \in \mathbb{Q}_{p} ^s$ définis par inégalités de la forme $\val\left( \ell_1(x)\right)\ge 0,\ldots, \val\left( \ell_n\left( x \right) \right)\ge 0$ pour $\ell_1,\ldots, \ell_n$ des formes linéaires $p$-adiques.
Ainsi, si le temps a manqué pour proposer des résultats sur la résolution de la programmation semi-définie positive $p$-adique, il sera toute fois présenté un algorithme résolvant la programmation linéaire en $O\left( \complalgo \right) $, où $m$ et $n$ sont les dimension de la matrice de contraintes. De plus, des pistes pour la résolution du problème $LMI$ ( \textit{Linear Matrix Inequality}) $p$-adique visant à déterminer la vacuité ou non d'un spectraèdre sont également évoquées. 

    %Les principales sections du rapport
 
    \tableofcontents
<<<<<<< HEAD
    \section{Introduction} 
Les spectraèdres se présentent comme une généralisation des polyèdres, définis comme les points $x = (x_1,\ldots,x_s)$pour lesquels une matrice linéaire $A(x) = A_0 + x_1A_1+\ldots+x_s A_s$ est symétrique semi-définie positives, avec $A_0,\ldots,A_s$ des matrice symétriques. Ils sont d'une grand utilité en optimisation car ils permettent de résoudre non seulement des problèmes d'optimisation linéaire mais également d'autres plus spécifiques. En effet, une sur-classe des problèmes de programmation linéaire, les problèmes de programmation semi-définie consiste à maximiser une application linéaire sur un spectraèdre.

Si ces objets ont été largement étudiés dans le cadre réel, une voie vers leur étude dans des corps non archimédiens s'est récemment ouverte dans \cite{allamigeon_tropical_2020}. C'est ce que ce rapport étudie dans le cadre des corps $p$-adiques, corps non-archimédiens qui peuvent être vus comme des extensions du coprs $\mathbb{Q}$ des rationnels autres que le corps des réels et dans lesquels les techniques traditionnelles ne marchent pas. Ainsi, le produit scalaire n'est en $p$-adique pas une forme bilinéaire particulièrement distinguée, ce qui annule tout bénéfice de la symétrie et le corps $p$-adique $\mathbb{Q}_{p}$ n'est de plus pas algébriquement clos. Il a donc fallu trouver une nouvelle définition de matrice semi-définie positive. Celle choisie ici est celle des matrices dont la valuation $p$-adique des valeurs propres est positive dans la clôture de $\mathbb{Q}_{p}$. On en déduit alors aisément une définition de spectraèdre $p$-adique et prouve qu'avec cette dernière les couronnes $p$-adiques sont des projetés de spectraèdre.


Ce rapport commence par décrire le déroulement du stage. Suite à quoi, les spectraèdres sont définis et quelques unes de leur propriétés décrites. Puis, suit une présentation élémentaire des corps $p$-adiques. Ensuite, les polyèdres seront définis dans le cas $p$-adique, et un algorithme résolvant le problème de la programmation linéaire sera décrit. Enfin, on y construira une définition des spectraèdre $p$-adiques, basé sur la nouvelle définition de matrice semi-définie positive après avoir brièvement discuté des clôtures algébriques des corps $p$-adiques. Ultimement, on prouvera que les couronnes $p$-adiques s'écrivent comme ombre de spectraèdre et y adjoindra d'éventuelle piste pour l'étude des propriétés algorithmiques des spectraèdre nouvellement définis.


    %Les principales sections du rapport 
 
    \section{Spectraèdres réels}
\label{sec:casreel} 
Cette section consiste en une introduction très brève à la notion de spectraèdre réel. Elle n'offre au lecteur que l'outillage nécessaire à la bonne compréhension de l'objet en vue de son adaptation aux corps $p$-adiques, en insistant toutefois sur son usage en optimisation. On ne pourra que conseiller la lecture de \cite{grigoriy_semidefinite_2012} pour des approfondissements.  
\subsection{Définition et premières propriétés}

\begin{definition}
	On appelle \emp{matrice symétrique semi-définie positive} toute matrice réelle symétrique et à valeurs propres positives ou nulles.
	On notera $\mathcal{S}_n^+\left(\mathbb{R}\right) $ l'ensemble de telles matrices et $M \succeq 0$ le fait que $M \in \mathcal{S}_n^+\left(\mathbb{R}\right)$.
\end{definition}

\begin{remarque}
	On remarquera que demander la symétrie permet de s'assurer d'obtenir des valeurs propres réelles grâce au théorème spectral, demander leur positivité fait alors sens.
\end{remarque}
\begin{propriete}
	L'ensemble $\mathcal{S}_n^+\left( \mathbb{R} \right) $ est un cône convexe fermé.
\end{propriete}

\begin{definition}
	On appelle \emp{spectraèdre} l'intersection de $\mathcal{S}_n^+\left(\mathbb{R}\right)$ avec un espace affine $\mathcal{L}$ de $\mathcal{S}_n\left(\mathbb{R}\right)$, l'ensemble des matrices symétriques de taille $n$ sur $\mathbb{R}$.
\end{definition}

En écrivant l'hyperplan $\mathcal{L}$ de $\mathcal{S}_n^+\left(\mathbb{R}\right)$ sous sa forme paramétrique \textit{i.e.} comme l'ensemble des matrices de la forme $A_0 + x_1 A_1 + \ldots x_s A_s$ pour $A_0,\ldots, A_s$ des matrices symétriques fixées on peut définir le spéctraèdre $\mathcal{S} = \mathcal{L} \cap \mathcal{S}_n^+\left(\mathbb{R}\right)$ comme
$\mathcal{S} = \{A := A_0 + x_1A_1 + \ldots + x_s A_s | A \succeq 0, (x_1,\ldots,x_s) \in \mathbb{R}^s\}$. On identifie alors souvent ce dernier à sa préimage dans $\mathbb{R}^s$
$S = \{(x_1,\ldots,x_s) \in \mathbb{R}^s | A_0+ x_1A_1 + \ldots+ x_s A_s \succeq 0 \}. $


  \begin{ex} Un exemple célèbre de spectraèdre est l'ensemble des matrices symétriques semi-définies positives avec diagonale $(1,1,1)$:
    $$
    S = \left\{
    \begin{pmatrix} x_1 \\ x_2 \\ x_3 \end{pmatrix} \in \mathbb{R}^3 |
    A:=\begin{pmatrix} 1 & x_1 & x_2 \\ x_1 & 1 & x_3 \\ x_2 & x_3 & 1 \end{pmatrix} \succeq 0
    \right\}.
    $$
    La surface algébrique définie par $\det A(x_1,x_2,x_3) = 0$ est appelée {\it cubique de Cayley} (\ref{cayley}).
    %Le spectraèdre $S$ est souvent appelé {\it samosa} et peut être obtenu comme {\it dérivée au sens de Renegar} du tetraèdre régulier \cite{sanyal}.
    Les quatre points singuliers correspondent à quatre matrices semi-définies positives de rang un ; les autres points de la surface, correspondent à des matrices de rang deux (semi-définies sur la frontière du spectraèdre, avec au moins une valeur propre négative autrement) ; enfin, les matrices à l'intérieur du spectraèdre sont définies positives (toutes valeurs singulières strictement positives).
    \begin{figure}[!ht]
      \centering
      \includegraphics[scale=0.3]{figures/cayley.pdf}
      \caption{Cubique de Cayley}
      \label{cayley}
    \end{figure}
  \end{ex}


\subsection{Programmation semi-définie}
\label{subsec:psdreele} 
On appelle alors \emp{programmation semi-définie} le problème d'optimisation
consistant à minimiser une application linéaire sur un spéctraèdre que l'on formulera comme :
\begin{equation}
  \tag{PSD}
\begin{aligned}
  \text{Minimiser } & \left\langle c,x \right\rangle \\
  \text{tel que }   & A_0 + \sum \limits_{i=1}^s x_{i}A_{i} \succeq 0
%	\begin{matrix}
%		\text{Minimiser } \simone{\left\langle c,x \right\rangle} \text{ tel que}\\
%		A_0 + \sum \limits_{i=1}^s x_{i}A_{i} \succeq 0
%	\end{matrix}
\end{aligned}
	\label{Psd} 
\end{equation}
pour $A_0,\ldots, A_s$ des matrices symétriques fixées, $c=(c_1,\ldots,c_s)$ un vecteur représentant le coût et $x \mapsto \left\langle c,x \right\rangle := c_1x_1+\cdots+c_sx_s$ le produit scalaire Euclidien. Le problème d'admissibilité associé au problème d'optimisation \eqref{Psd}, c'est-à-dire, la question si le spectraèdre $S = \{(x_1,\ldots,x_s) \in \mathbb{R}^s | A_0+ x_1A_1 + \ldots+ x_s A_s \succeq 0 \}$ est vide, est appelée \emp{inégalité matricielle linéaire (LMI)}.

En précision finie $\epsilon$, ce problème se résout en temps polynomial en la dimension de l'entrée (taille des matrices, nombre de variables, taille binaire des coefficients), en $\log(1/\epsilon)$ et $\log(R)$, où $R$ est une borne {\it a priori} sur la norme d'une solution. En arithmétique exacte, la complexité de la programmation semi-définie est un problème essentiellement ouvert, cf \cite[Sec.1.9]{deKlerk}, \cite{ramana1997exact,porkolab1997complexity} et \cite{henrion2016exact}.

Si à première vue ce problème peut sembler très spécifique il n'en est rien et de nombreux autres problèmes se rapportent à celui-ci. Par exemple, tout problème d'optimisation linéaire est en particulier un problème SDP:

\begin{remarque} Un polyèdre est un spectraèdre; en particulier, l'optimisation linéaire est une sous-classe de l'optimisation semi-définie. En effet, soit $P = \{x \in \mathbb{R}^s | \ell_1(x) \geq 0, \ldots, \ell_d(x) \geq 0\}$ le polyèdre défini par les inégalités linéaires $\ell_1,\ldots,\ell_d$, et soit $D$ la matrice linéaire diagonale avec entrées $\ell_1,\ldots,\ell_d$. Alors $P$ est le spectraèdre défini par $D \succeq 0$.
\end{remarque}
 
    \section{Introduction aux nombres \texorpdfstring{p}{$p$}-adiques}

On se contentera dans cette section d'une description très élémentaire des différentes définitions et propriétés des nombres $p$-adiques. La plupart des preuves relative à cette section ainsi que de plus amples informations sont disponibles en annexe \todo{l'annexe}. Cette section est très largement inspiré du cours de Xavier Caruso \parencite{caruso_computations_2017} que l'on invite d'ailleurs à aller consulter pour une vision plus complète mais très largement compréhensible.

\begin{notation}
	On considère pour tout ce rapport $p$ un nombre premier.
\end{notation}

\subsection{Entiers \texorpdfstring{p}{$p$}-adique} 
\begin{definition}{Entier $p$-adique }

On appelle entier $p$-adique la somme formelle :
\[
	z  = a_0 + a_1 p + \ldots+a_{n}p^n+\ldots
\]
ou les $a_i$ sont des entiers compris entre $0$ et $p-1$.

\end{definition}

\begin{remarques}
	\begin{itemize}
		\item[$\circ$]  On note $\Zp$ l'ensemble des entier $p-$adiques.
		\item[$\circ$] Par commodité on notera $\ldots a_n\ldots a_1 a_0$ l'entier $p$-adique $\sum a_{i}p^i$ 
\end{itemize}
\end{remarques}

\begin{ex}

	Ainsi les sommes $\sum\limits_{i=0}^{ \infty} p^i = \ldots1111111$ ou $\sum\limits_{i=0}^{ \infty} (i\ \text{mod}\ p) p^i = \ldots210(p-1)\ldots21 $ sont des entiers $p$-adiques parfaitement définis bien que ne convergeant pas dans le cas réel. 
\end{ex}

\begin{propriete}
	$\mathbb{Z}_p$ peut être muni d'une structure d'anneau commutatif intègre en lui adjoignant l'addition terme à terme avec retenue et la multiplication.
\end{propriete}

Par exemple dans $\mathbb{Z}_5$ 

\begin{tabular}{lS}
     & \ldots34202243\\
  $+$& \ldots01423401\\
  \hline
  & \ldots 41131144 \\
  
\end{tabular}

\begin{tabular}{lS}
     & \ldots02243\\
  $\times $& \ldots23401\\
  \hline
  & \ldots 02243\\
  & \ldots 0000 \\
  & \ldots 132\\
  & \ldots 34\\
  $+$ & \ldots 1\\
\hline
& \ldots14443	
\end{tabular}

\begin{propriete}
	 $\mathbb{Z}$ est un sous-anneau de $\mathbb{Z}_p$. 	
\end{propriete}
\textit{Preuve :} Tout entier naturel $a$ admet une décomposition en base $p$ (qui est unique) i.e. s'écrit $a = \sum\limits_{i=0}^{n} a_{i}p^i$ avec $n = \left\lfloor \log_p a\right\rfloor$ et s'associe naturellement à l'élément $\ldots0000a_n\ldots a_0$ de $\mathbb{Z}_p$. Puis à tout entier négatif $b$ on associe l'opposé dans $\mathbb{Z}_p$ de $|b|$. Il n'est alors pas compliqué de vérifier que les opérations de $\mathbb{Z}_p$ restreintes à la projection de $\mathbb{Z}$ coïncident avec les opération dans $\mathbb{Z}$. \hfill \qedsymbol

\begin{remarque}
	Si l'on a vu que les entiers au sens réel était des entiers $p$-adiques, certains entiers $p$-adique ont du sens en tant que nombre rationnels sans être des entiers relatifs, ainsi on a par exemple $\frac{1}{2} = \ldots 2223 \in \mathbb{Z}_5$. Cependant tous les rationnels ne sont pas éléments de $\mathbb{Z}_p$, $\frac{1}{p}$ n'étant par exemple jamais inclus dans $\mathbb{Z}_p$.
\end{remarque}

\subsection{Nombres \texorpdfstring{p}{$p$}-adiques}

\begin{definition} Nombres $p$-adiques 

	On définit l'ensemble $\Qp$ des nombres $p$-adiques comme $\mathbb{Z}_p \left[ \frac{1}{p} \right] $.   
\end{definition}

Un nombre $p$-adique $x$ s'écrit alors comme une somme de la forme $x = \sum \limits_{i=k}^{\infty} x_{i} p^i$ avec $k \in \mathbb{Z}$ et les $x_{i}$ compris entre $0$ et $p-1$. Si $k<0$ on écrira plus couramment $x = \ldots x_i \ldots x_1 x_0 , x_{-1}\ldots x_{k}$. 

\begin{propriete}
	$\mathbb{Q}_{p}$ est un corps qui étend les opérations de $\mathbb{Z}_p$.
\end{propriete}
\textit{Preuve :} Voir \todo{l'annexe}

\todo{exemple d'opérations dans $\mathbb{Q}_{p} $ } 
\begin{corollaire}
	Le corps $\mathbb{Q}$ des rationnels est un sous-corps de $\mathbb{Q}_{p} $.
\end{corollaire}

Ce dernier résultat permet de construire de manière assez élémentaire des éléments de $\mathbb{Q}_{p}$ qui ne sont pas des entiers $p$-adique.

\subsection{Valuation et norme}

	\todo{Petit paragraphe introducif} 
On définit la valuation $p$-adique dans $\mathbb{Z}$ $\val^{\mathbb{Z}}:\mathbb{Z}\to \mathbb{N}\cup \{+\infty\}  $ comme l'application qui à 0 associe $+\infty$ et à un entier $a$ non nul associe le plus grand entier naturel $k$ tel que $p^k | a$% ou de façon équivalente en considérant $\sum \limits_{i=1}^{n} a_{i} p^i$ la décomposition de $a$ en base $p$, la valuation $p$-adique de $a$ est le plus petit $a_{i}$ non nul. 

La valuation $p$-adique s'étend ensuite aux nombres rationnels en une application $\val^{\mathbb{Q}} : \mathbb{Q} \to \mathbb{Z}\cup \{+\infty\}   $ en définissant pour tout $r \in \mathbb{Q}$  $\val^{\mathbb{Q}} \left( r \right) = \val^\mathbb{Z}(a)- \val^\mathbb{Z}\left( b \right) $ avec $a,b \in \mathbb{Z} \times \mathbb{N}^*$ tels que $r=\frac{a}{b}.$   

La valuation $p$-adique s'étend alors également à $\mathbb{Q}_p$ depuis $\mathbb{Q}$ comme suit :
\begin{definition} Valuation $p$-adique
  
	On appelle valuation $p$-adique l'application $\val: \mathbb{Q}_p \to \mathbb{Z}\cup \{+\infty\}  $ qui à un nombre $p$-adique $x$ associe $\max \{k \in \mathbb{Z}\cup \{+\infty\}| x\in p^k \mathbb{Z}_p\}$. 
\end{definition}
Une manière simple de visualiser la valuation d'un nombre $p$-adique est de compter la "distance à la virgule".

En effet, la valuation d'un entier $p$-adique correspond au nombres de $0$ à la fin de son écriture décimale \todo{p-imale ?} et pour un nombre $p$-adique non entier il s'agit de l'opposé nombre de décimales \todo{p-imales ?} après la virgule. Par exemple, $\val\left( \ldots 2413000 \right) = 3$ et $\val\left( \ldots 251,24 \right) = -2$.       

Le principal intérêt qu'offre la notion de valuation pour le sujet développé ici est qu'elle permet de définir une notion de positivité dans un corps qui n'est pas totalement ordonnable \todo{définition peut-être en footnote ?}. À cet effet on introduira la notation suivante :
\begin{notation}
	Pour tout élément $x \in \mathbb{Q}_{p} $, on dit que $x$ est \textit{positif} et on note $x\ge 0$ si $\val\left(x\right)\ge 0$. On en induit alors les notations $x> 0$, $x\le 0$ et $x<0$.  
\end{notation}

        On évitera la notation $x\ge y$ qui pourrait laisser penser de manière trompeuse que $x\ge y \Rightarrow x-y\ge 0$\footnote{Par exemple, $\val(\ldots11,11) \ge \val(\ldots00,01) $ mais $\val(\ldots11,11 - \ldots00,01) = \val( \ldots 11,1) < 0$}.

\begin{propriete}
	La valuation $p$-adique possède les propriétés suivantes, pour tous $x$ et $y$ appartenant à $\mathbb{Q}_{p} $ :   
	\begin{enumerate}
		\item $\val(x+y) \ge \min\left( \val\left( x \right), \val\left( y \right)  \right) $ 
		\item $\val\left( xy \right) = \val (x) + \val( y)$ 
	\end{enumerate}
\end{propriete}
\textit{Preuve :} \todo{annex ?}  
\todo{Demander si mettre des exercices dans un rapport de stage c'est bien vu} 

Ces propriétés permettent alors de munir $\mathbb{Q}_{p} $ d'une valeur absolue \footnote{c'est-à-dire une norme sur $\mathbb{Q}_{p} $ vu comme $\mathbb{Q}_{p} $-espace vectoriel} que l'on définira comme suit:

\begin{definition} Valeur absolue $p$-adique

	On appelle valeur absolue $p$-adique l'application 
\begin{align*}
|\cdot|_p : \mathbb{Q}_{p} & \longrightarrow \mathbb{R}^*_+\\
x & \longmapsto p^{-\val\left(x\right)} 
\end{align*}
\end{definition}

\begin{propriete}
	$\left| \cdot  \right|_p$ est une valeur absolue sur $\mathbb{Q}_{p}$ 
\end{propriete}
\textit{Preuve :} \todo{l'annexe pour changer}

On remarque en particulier d'après \todo{lien vers la prop} que pour tous $x,y \in \mathbb{Q}_{p}, \left|x+y\right|_p \le \max\left( x,y \right)$. Ce qui en fait un espace non archimédien \footnote{c'est-à-dire que $\mathbb{N}$ est borné dans $\left( \mathbb{Q}_{p}, \left| \cdot \right|_p \right) $} et rend la géométrie $p$-adique très différente du cas réel peu intuitive. Ce qui explique le manque de figure et d'explications par le dessin dans la suite de ce rapport.

On terminera cette section en discutant la proposition suivante, qui est d'une importance cruciale puisqu'elle offre une caractérisation simple de la positivité.

\begin{proposition}
	Soit $x \in \mathbb{Q}_{p} $. Les trois propriétés suivantes sont équivalentes 
	\begin{enumerate}[label= \textit{\roman*}.]
		\item $x \in \mathbb{Z}_p$
		\item $\val\left(x\right)\ge 0$
		\item $\left| x \right|_p\le 1$
	\end{enumerate}
\end{proposition}

\textcolor{red}{ On dira alors indistinctement qu'un nombre $x$ est un entier, est un élément de la boule unité ou est positif (conformément à \todo{lien vers la notation}).}

\textit{ Preuve de la propriété : } L'équivalence entre $ii$. et $iii$. découle directement de la définition de $\left| \cdot  \right|_p$. Puis on conclut en remarquant que $x \in \mathbb{Z}_p = p^0 \mathbb{Z}_p$ si et seulement si $\val\left(x\right)\ge 0$ c'est-à-dire $i. \iff ii.$.

    \section{Polyèdres convexes \texorpdfstring{$p$}{p}-adiques}
\label{sec:polyedre} 
\iffalse
\subsection{Matrices symétriques positive} 

\begin{definition}
	On note $\P_n\left( \Qp \right) $ l'ensemble des matrices symétriques dont tous les mineurs principaux ont une valuation positive.
\end{definition} 

\begin{rappel}
	
Un élément de $ \Qp$ a une valuation positive si et seulement si il est élément de $\Zp$. 
\end{rappel}

\begin{propriete}
	
	$\P_n\left( \Qp \right) = \{ M \in S_n\left( \Qp \right)$ | les\- min\-eurs\- prin\-ci\-paux\- de\- $M$ \-sont \-à \-va\-leur \-dans\- $\Zp \} $.
\end{propriete}

	\textit{Preuve :} découle directement du rappel précédent. 
	\medskip


\begin{prop}
	 \[
		 \Pn = S_n\left( \Zp \right) 
	.\]  
\end{prop}

\begin{remarque}
	On remarque alors que l'ensemble $\P_n$ correspond aux matrices symétriques à coefficients positifs.   
\end{remarque}
	\textit{Preuve :}

 Le déterminant étant une fonction polynomiale en les coefficients de la matrice, toute matrice de à coefficient dans $\Zp$ a un déterminant à valeur dans $\Zp$. D'où, par la propriété 2, $S_n\left( \Zp \right) \subset \Pn $.

 L'inclusion réciproque se montre par récurrence. On note pour tout $n \in \mathbf{N}$ $\mathcal{H}_n :  \Pn \subset  S_n(\Zp )$.

 
 On notera $ \Delta_{i_1,\ldots,i_n}\left( M \right) $ le mineur principal de $M$ composé des lignes et des colonnes d'indices $i_1,\ldots,i_n \in \left\{ 1,\ldots,n \right\} $ pour tout matrice $M$. On notera d'ailleurs simplement  $ \Delta_{i_1,\ldots,i_n}$ lorsque le contexte est explicite.

 Les cas $n=0$ et $n=1$ se démontrent sans difficultés aucunes. Montrons le cas $n=2$ qui servira par la suite.

 Soit $M \in \P_2\left( \Qp \right) $, $M$ s'écrit $M = \begin{pmatrix} \alpha & \gamma \\ \gamma & \beta \end{pmatrix}$, avec $\alpha, \beta, \gamma \in \Qp^3$.



 On sait alors que $ \alpha = \Delta_1$ et $ \beta = \Delta_j$ sont des entiers $p$-adiques, il suffit de montrer que $\gamma$ en est également un. Pour ce faire supposons que $ \val (\gamma) < 0$, on a alors $\val ( \gamma^2) = 2 \val\left( \gamma \right) < \val\left( \alpha \beta \right) $ et on en déduit $ \val \left( \Delta_{1,2} \right)= \min\left( \val\left( \alpha \beta \right),2~ \val \left( \gamma \right) \right) = 2 \val\left( \gamma \right) <0 $ ce qui contredit la positivité de $\Delta_{1,2}$ et est donc absurde. On conclut alors que $ \gamma \in \Zp$ et $M \in  S_n\left( \Zp \right) $. On a montré $\mathcal{H}_2$. 

 Soit $n \in \mathbf{N}$ tel que la propriété $\mathcal{H}_n $ soit vérifiée et $M$ une matrice de $\Pn$. 

 $M$ s'écrit 
 \[
 M= \left(\begin{array}{ccc|c}
  &      &     &   \beta_1  \\
  &  M'  &     &\vdots\\
  &      &     &   \beta_n  \\
\hline
\beta_1 &\cdots&  \alpha_n  & \alpha_{n+1}
\end{array} \right)
\]
avec $M' \in S_n\left( \Qp \right) $ et $\beta_1,\ldots, \beta_n, \alpha_{n+1} \in \Qp$.

On note $ \alpha_1,\ldots, \alpha_n$ les coefficients diagonaux de $M'$ qui sont des entiers $p$-adique par hypothèse de récurrence. 


 Par définition $ \alpha_{n+1} = \Delta_{n+1}$ est un entier $p$-adique. Puis on se ramène au cas $n=2$ en utilisant le fait que pour i=1,\ldots, n, $\Delta_{i,n+1} = \begin{vmatrix} \alpha_{i} & \beta_{i}\\ \beta_{i} & \alpha_{n+1} \end{vmatrix} $ et on en déduit que $\beta_{i} \in \Zp$ pour $i=1,\ldots,n$. On conclut en appliquant l'hypothèse de récurrence à $M'$.


\hfill \qedsymbol
\begin{remarque}
	
	La preuve de la proposition précédente montre qu'il suffit en réalité que les mineurs principaux de taille au plus $2$ aient une valuation positive (ou soient éléments de $\Zp$) ce qui correspond à la définition de matrice semi-définie positive sur le semi-corps tropical développée par Allamigeon, Gaubert et Skorma dans \cite{allamigeon_tropical_2020} . Ce n'est toutefois pas la définition qui sera choisie ici, pour des raisons développées en partie \hyperlink{subsection.1.2}{1.2}.
\end{remarque}

\fi


\subsection{Polyèdres convexes \texorpdfstring{$p$}{p}-adiques} 

\begin{definition}
	(Matrice positive)


	Une matrice $M$ de $\mathcal{M}_n(\Qp)$ est dite \emph{positive} si tous ses coefficients sont positifs ou nuls, c'est-à-dire, par définition si elle est à coefficient dans $\mathbb{Z}_p$.
	On notera alors $M\ge 0$ le fait que $M \in \mathcal{M}_{n}\left(\mathbb{Z}_p\right) $

\end{definition}

%Tout ce beau monde ira peut-être en annexe si 
\begin{propriete}
	L'ensemble $\Pn$ est :
	\begin{enumerate}[label=\roman*.]
	\item ouvert
	\item fermé
	\item borné
	\item compact
	\item convexe au sens de \parencite{monna_ensembles_1958}
\end{enumerate}
\end{propriete}
\textit{Preuve :}
$i.$ et $ii.$ se déduisent du fait que $\Zp$ soit ouvert et fermé dans $\Qp$, $ii.$ découle directement du fait que $\|M\|_\infty = \sup |M_{i,j}|_p \le 1$ et $iv.$ se déduit de $ii.$ et $iii.$.
Quand à $v.$ c'est une conséquence directe de la convexité de $\Zp$.

\todo{$\Pn$ est un cône $p$-adique pour la définition :Soit $\mathbb{E} $ un $\mathbb{Q}_{ p } $ espace vectoriel $C \subset \mathbb{E} $ est un cône si pour tout $x \in C$ et $\lambda \ge 0 $ $\lambda x \in C$. La preuve pour $ \Pn$ est assez triviale et pour $S_n^+\left( \mathbb{Q}_p \right)$ elle est laissée en exercice au lecteur} 
\begin{definition}
	
On définit un polyèdre convexe $P$ comme l'intersection de $\Pn$ avec un espace affine $\mathcal{L} $ de $\mathcal{M}_n\left( \Qp \right) $. 
\end{definition}
Comme en \ref{sec:casreel} dans le cas réel on identifiera un polyèdre à sa préimage. Ce qui permet le résultat suivant :


\iffalse
Soit $P$ un polyèdre convexe et $\mathcal{L} $ un plan affine tel que $P = \Pn \cap \mathcal{L}$, et dont on note $s$ la dimension de l'espace vectoriel associé. On dispose de donc de $s +1 $ matrices $M^0,M^1,\ldots, M^s$ telles que $\mathcal{L} = M^0 + \text{Vect}\left( M^1,\ldots,M^s \right)$. Le polyèdre $P$ s'écrit alors $P = \left\{ M^0 + x_1 M^1 + \ldots + x_s M^s \ge 0 \right\}$. On identifie alors souvent $P$ avec $\left\{ \left( x_1,\ldots,x_s \right) \in \Qp^s |M^0 + x_1 M^1 + \ldots + x_s M^s \ge 0 \right\}$ ce qui permet d'écrire qu'un vecteur $x_1,\ldots,x_s$ de $\Qp^s$ est élément de $P$ si et seulement si il vérifie :
 
	\begin{equation}
	\label{eq:1} 
\forall i,j ~  M^0_{i,j} + x_1 M^1_{i,j} + \ldots + M^s_{i,j} \ge 0
	\end{equation}

On peut alors réécrire l'inégalité matricielle en  $P = \left\{ x \in \Qp^s \-| Ax + b \ge 0 \right\} $ avec 
\[A = \begin{pmatrix} M^1_{1,1} & \ldots & M^s_{1,1} \\
\vdots & & \vdots \\
M^1_{n,n} & \ldots & M^s_{1,1}\\ \end{pmatrix} \in M_{n^2,s}\left( \Qp \right) \text{ et } 
b = \begin{pmatrix} M^0_{1,1} \\
\vdots\\
M^0_{n,n} \end{pmatrix} \in M_{n^2, 1}\left( \Qp \right) 
.\]  
\begin{remarque}
	On peut réduire de moitié la tailles des matrices $A$ et $b$ en considérant que les coefficients diagonaux et supradigonaux des $M^i$ pour $i = 0,1,\ldots,n$. Ce qui permet de se ramener à des matrices équivalente\footnote{puisque les inéquations de \ref{eq:1} impliquant le couple $(i,j)$ $i>j$ sont redondantes avec les équations impliquant $(j,i)$}  avec $\frac{n(n+1)}{2}$ lignes.  
\end{remarque}

En mettant sous cette forme le polyèdre on reconnait alors aisément que le problème de minimiser une application linéaire sur un polyèdre correspond exactement à résoudre le problème de la \textit{Programmation linéaire}.

\begin{remarque}
	On se ramène au cas quelconque du problème de la \textit{Programmation linéaire} (taille quelconque et non seulement avec un nombre de ligne en $\frac{n(n+1)}{2}$ ou $n^2$) en annulant des coefficients $M^k_{i,j}$ pour tout $i=1,\ldots,n$.   
\end{remarque}

\begin{propriete}
Un polyèdre $p$-adique convexe est convexe au sens de \cite{monna_ensembles_1958}.  
\end{propriete}

\textit{Preuve :} Provient immédiatement du fait qu'un polyèdre s'écrit comme intersection d'ensemble convexe.
\fi
\begin{ex}
	La boule unité de $\mathbb{Q}_{p}^n$ pour la norme infinie est un polyèdre. En effet, la boule infinie s'écrit comme l'ensemble des points $\left( x_1,\ldots,x_n \right)$ tels que $
	\begin{pmatrix} x_1 &  & & \\
		  & x_2 &0 &  \\
		&  0& \ddots & \\
		&  & & x_n
	\end{pmatrix} \ge 0$. En effet, la boule unité de $\mathbb{Q}_{p} ^n$ est $\mathbb{Z}_p^n$. %En effet, considérons le polyèdre défini par l'intersection de $S_n\left( \mathbb{Q}_{ p }  \right) $ avec le plan linéaire induit par les matrices $E_k$, k=1\ldots n, de $S_n\left( \Zp \right) $ telles que le seul coefficient non nul de $E_k$ soit le k-ième coefficient diagonal lequel est égal à $1$ . On observe que pour tout $x_1,\ldots,x_n \in \mathbb{Q}_{ p } ^n$, $\sum x_k E_k \ge 0$ si et seulement si $x_1,\ldots,x_s \in \mathbb{Z}_{ p }^n $ i.e. si et seulement si $\|(x_1,\ldots,x_n)\|_\infty \le 1$.

\end{ex}

\subsection{Programmation linéaire \texorpdfstring{$p$-adique}{p-adique} } 
\label{sectionalgo} 
Dans ce paragraphe, il sera étudié une forme équivalente du problème de programmation linéaire, appelée \textit{programmation linéaire $p$-adique} (\ref{eqn:Proglinp}) . Lequel consiste simplement à minimiser la norme $p$-adique d'une application linéaire sur un polyèdre $p$-adique. 

	Du fait des natures profondément différentes des polyèdres $p$-adiques et du cas réel, les techniques classiques de résolution sont mises à mal. Il est effet complexe d'appliquer la méthode du simplexe à un ensemble sans frontière ou des techniques d'analyse convexe dans un espace sans notion de convexité. Il convient donc alors de développer de nouvelles techniques pour résoudre ces problèmes. C'est ce qui est proposé dans cette section, qui présente un algorithme en $O( \max\left( m,n \right)^2 )$, avec $m,n$ les dimensions de la matrice de contrainte, pour résoudre le problème de la programmation linéaire en $p$-adique, dont une écriture en pseudo-code ainsi qu'une implémentation en \sage sont disponible en \ref{appendixalgo}.

	Cet algorithme est centré sur l'utilisation de la forme normale de Smith d'une matrice, dont seul la définition et quelques remarques sont présenté dans cette section, les preuves des résultats présentés ici ainsi que d'autres résultats sont disponible en \ref{smith}.

	On appelle \textit{programmation linéaire $p$-adique} le problème :
\begin{equation}
	  \tag{PLp}
\begin{matrix}
	\text{Minimiser } \val\left(\left<c,x \right>\right) \text{ tel que }\\
	Ax + b \ge 0
 \end{matrix}
	    \label{eqn:Proglinp}
\end{equation}

avec $x$ un vecteur de taille $n$ que l'on fait varier, $c$ un vecteur de taille $n$ représentant le coût, $A$ une matrice de taille $m \times n $ et $b$ un vecteur de taille $m$.

La méthode choisie ici consiste à mettre la matrice $A$ sous forme normale de Smith, un factorisation classique en $p$-adique et qui permet de grandement simplifier le problème posé.

\begin{definition}
	Forme Normale de Smith

		On appelle forme normale de Smith d'une matrice $M \in \mathcal{M}_{m,n}\left(\mathbb{Q}_{p} \right) $ de rang $r$ l'unique matrice $S$ de la forme $$S =  
	\begin{pmatrix} p^{a_1} & \\
		 & \ddots \\
		 & & p^{ar}\\
		 & & & 0\\
		 & & & & \ddots \end{pmatrix} $$
		 telle que $a_1\le  \ldots\le a_r$ et $M =  Q^{-1} S P$ avec $P \in \mathcal{G}L_n\left( \mathbb{Z}_p \right) $ et $Q \in \mathcal{G}L_m\left( \mathbb{Z}_p \right) $.
\end{definition}

\todo{donner un exemple} 
\begin{remarques}

	
	\begin{enumerate}[label=\roman*.]
		\item Les coefficients de la forme normale de Smith sont uniques à chaque matrice et sont appelés \textit{facteurs invariants de Smith} ou, plus simplement, \textit{invariants de Smith}.
		\item La valuation $p$-adique du premier coefficient de la forme normale de Smith d'une matrice $M \in \mathcal{M}_n \left( \Qp \right) $ est égale au minimum des valuation des termes de $M$.
		\item En particulier, la forme normale de Smith d'une matrice de $\mathcal{M}_n \left( \mathbb{Z}_p \right) $ est à coefficients dans $\mathbb{Z}_p$ .
		\item Les $r = \text{rang} M$ premiers coefficients diagonaux de $S$ sont exactement ses coefficients non nuls.  
	\end{enumerate}

\end{remarques}


Avant de pouvoir utiliser la forme de normale de Smith pour résoudre \ref{eqn:Proglinp}, il nous faut démontrer le lemme suivant:

\begin{lemme}
Pour tous $z \in \mathbb{Q}_{ p } ^n$ et $M \in \mathcal{M}_{m,n}\left( \mathbb{Q}_{ p }  \right)  $ si $z\ge 0$ et $M\ge 0$ alors $Mz\ge 0$.  
\end{lemme}
\textit{Preuve :} $\mathbb{Z}_p$ est un anneau. \hfill\qedsymbol



En mettant alors la matrice $A$ de \ref{eqn:Proglinp} sous sa forme normale normale de Smith il vient que résoudre \ref{eqn:Proglinp} revient à résoudre :    


\begin{equation}
	  \tag{PLp'}
\begin{matrix}
	\text{Minimiser } \val\left(\left<c',x \right>\right) \text{ tel que }\\
	Sy + b' \ge 0
 \end{matrix}
	    \label{eqn:Proglinp2}
\end{equation}
où $b' = Qb$, $c' = P^Tc$, $S$ est la forme normale de Smith de $A$ et $A = Q^{-1} S P$ avec $P \in \mathcal{G}L_m\left( \mathbb{Z}_p \right)$ et $ Q \in \mathcal{G}L_n\left( \mathbb{Z}_p \right)$.

\begin{remarques}
	
Il en vient immédiatement plusieurs résultats:
\begin{enumerate}
	\item $x^*$ est une solution admissible de \ref{eqn:Proglinp} si et seulement si $y^* := P x^*$ est une solution admissible de \ref{eqn:Proglinp2}
	\item \ref{eqn:Proglinp2} possède des solutions admissible si et seulement si les $m-r$ coefficients de $b'$ sont non nuls, où $r$ le rang de $S$.     
	\item Si les $n-r$ tous derniers coefficients de $c'$ sont non nuls \ref{eqn:Proglinp2} n'est pas borné et n'admet donc pas de solution.   
\end{enumerate}

\end{remarques}

\begin{propriete}
	\begin{itemize}~

		\item[$\circ$] Si les $m-r$ coefficients de $b'$ ne sont pas tous nuls alors il n'existe pas de $y \in \mathbb{Q}_{p} ^n$ tel que $A'y+b' \ge  0$.
	\item[$\circ$] Si les $n-r$ coefficients de $c'$ ne sont pas tous nuls alors le problème n'est pas borné et n'admet donc pas de solution. \end{itemize} 
\end{propriete}
En ne considérant alors que les itérations du problème admettant des solutions on peut réduire le problème en ne considérant que les $r$ premiers coefficients de $y, b', c'$ et la sous matrice de $S$ composée des $r$ premières lignes et colonnes et dont les coefficients sont alors les exactement les facteurs invariants de Smith non nuls. L'ensemble $Adm$ des solutions admissible s'écrit alors comme l'ensemble des vecteurs $y \in \mathbb{Q}_{ p } ^n$ vérifiant $
\forall 1 \le i\le r \ s_i y_i + b'_i \in \mathbb{Z}_p$ c'est-à-dire vérifiant :

\begin{equation}
	\forall 1 \le i\le r  y_i \in -\frac{b’_i}{s_i} + \frac{1}{s_i} \mathbb{Z}_p
\end{equation}
 

Résoudre \ref{eqn:Proglinp2} revient donc à minimiser $\val\left(\left<c',y \right>\right)$ sur $Adm$. L'image de $Adm$ par $y \mapsto \left<c',y \right>$ est $\sum_{i=1}^r -c'_i.\frac{b'_i}{s_i} + \sum_{i=1}^r\left( \frac{c'_i}{s_{i}} \mathbb{Z}_p \right)$ qui se réécrit :
$$c'Adm = \lambda + p^{v} \mathbb{Z}_p$$
où $\lambda = \sum\limits_{i=1}^r -c'_i.\frac{b'_i}{s_i}$ et $v = \min\limits_{1\le i\le r} \val \frac{c'_{i}}{s_{i}} $.
Ainsi, deux cas apparaissent.
\begin{itemize}
	\item Soit $\val( \lambda) < v$ auquel cas le minimum de $y\mapsto \val\left(\left<c',y \right>\right)$ sur $Adm$ est atteint en n'importe quel point de $Adm$ et vaut $\val\left( \lambda\right)$.
	\item Soit $\val\left( \lambda \right) \ge v$, auquel cas $\lambda \in p^{v} \mathbb{Z}_p$ et le minimum vaut $v $ et est atteint en tous les points $y$ de $Adm$ vérifiant
	$$\val\sum \limits_{\begin{array}{c} 1\le i\le r\\ \val\left(c'_{i}/{s_{i}} \right) = v  \end{array}   } y_{i} = 0.$$ 

\end{itemize}
\begin{remarque}
	Si l'on souhaite maximiser la valuation d'un application linéaire su un polyèdre $p$-adique au lieu de la minimiser (ce qui revient à maximiser la valeur absolue) on pourra utiliser un raisonnement similaire à celui présenté dans cette partie. La seule différence est que si $\val\left( \lambda\right) < v$ le problème n'est pas borné et n'admet donc pas de solution. 
\end{remarque}


    \partie{2}{mardi 20 juin 11:00}{Spectraèdres $p$-adiques }
\newcommand\mat{matrice symétrique semie-définie positive } 
\newcommand\Mat{Matrice symétrique semie-définie positive }
\newcommand\mats{matrices symétriques semie-définies positives }
\newcommand\Mats{Matrices symétriques semie-définies positives }


\section{Spectraèdre \texorpdfstring{$p$}{p}-adiques } 
Ce paragraphe tend à fournir un définition de la notion de \mat sur les corps $p-$adiques pour en déduire une définition de spectraèdre qui serait pertinente sur un corps non-archimédien. 
\subsection{\Mats} 

\begin{definition}
	On appelle \mat toute matrice $M \in S_n\left( \Qp \right) $ dont toutes les valeurs propres sont de valuation positive ou nulle[lien du pragarâgre avec les extesnions de corps lolololololololol].

	On note $S_n^+\left( \mathbb{Q}_p \right)$ l'ensemble des \mats.
\end{definition}
\begin{propriete}
	\label{caracsnp}
	Caractérisation des \mats

	Une matrice est symétrique définie positive si et seulement si son polynôme caractéristique est à coefficient dans $\Zp$ .
	
\end{propriete}
	\textit{Preuve :} Voir le LIEEEEEEEEEEEEEEEEEEEEEEEEEEEEEEEEEEEEEEEEEEEEEN0

\begin{consequence}
	$\Pn \subset S_n^+\left( \mathbb{Q}_p \right)$ 
\end{consequence}

\textit{Preuve : }  Le polynôme caractéristique d'une matrice à coefficients dans $\Zp$ étant à coefficient dans $\Zp$ on obtient le résultat par la propriété \ref{caracsnp} .

\begin{remarque}
	En général l'inclusion réciproque est fausse. Ainsi pour $M = \begin{pmatrix} 5 + \frac{3}{5} & \frac{4}{5} \\ \frac{4}{5} & -\frac{3}{5} \end{pmatrix} $, on a $\chi_M = X^2  - 5 X - 4$. Or une fois $M$ plongé dans $\mathbb{Q}_5$ on a $\chi_M \in \mathbb{Z}_5[X]$ donc $M \in S_2^+\left( \mathbb{Q}_5 \right)$ or aucun des coefficients de $M$ n'est dans $S_2^+\left( \mathbb{Q}_5 \right)$

\end{remarque}

\note{vérifier si le fait qu'une matrice snp soit soit positive ou non est lié à la présence ou non de racine uniquement dans $\mathbb{Z}_p$ }.
 


=======

    \newpage
    
    \section{Introduction} 
Les spectraèdres se présentent comme une généralisation des polyèdres, définis comme les points $x = (x_1,\ldots,x_s)$pour lesquels une matrice linéaire $A(x) = A_0 + x_1A_1+\ldots+x_s A_s$ est symétrique semi-définie positives, avec $A_0,\ldots,A_s$ des matrice symétriques. Ils sont d'une grand utilité en optimisation car ils permettent de résoudre non seulement des problèmes d'optimisation linéaire mais également d'autres plus spécifiques. En effet, une sur-classe des problèmes de programmation linéaire, les problèmes de programmation semi-définie consiste à maximiser une application linéaire sur un spectraèdre.

Si ces objets ont été largement étudiés dans le cadre réel, une voie vers leur étude dans des corps non archimédiens s'est récemment ouverte dans \cite{allamigeon_tropical_2020}. C'est ce que ce rapport étudie dans le cadre des corps $p$-adiques, corps non-archimédiens qui peuvent être vus comme des extensions du coprs $\mathbb{Q}$ des rationnels autres que le corps des réels et dans lesquels les techniques traditionnelles ne marchent pas. Ainsi, le produit scalaire n'est en $p$-adique pas une forme bilinéaire particulièrement distinguée, ce qui annule tout bénéfice de la symétrie et le corps $p$-adique $\mathbb{Q}_{p}$ n'est de plus pas algébriquement clos. Il a donc fallu trouver une nouvelle définition de matrice semi-définie positive. Celle choisie ici est celle des matrices dont la valuation $p$-adique des valeurs propres est positive dans la clôture de $\mathbb{Q}_{p}$. On en déduit alors aisément une définition de spectraèdre $p$-adique et prouve qu'avec cette dernière les couronnes $p$-adiques sont des projetés de spectraèdre.


Ce rapport commence par décrire le déroulement du stage. Suite à quoi, les spectraèdres sont définis et quelques unes de leur propriétés décrites. Puis, suit une présentation élémentaire des corps $p$-adiques. Ensuite, les polyèdres seront définis dans le cas $p$-adique, et un algorithme résolvant le problème de la programmation linéaire sera décrit. Enfin, on y construira une définition des spectraèdre $p$-adiques, basé sur la nouvelle définition de matrice semi-définie positive après avoir brièvement discuté des clôtures algébriques des corps $p$-adiques. Ultimement, on prouvera que les couronnes $p$-adiques s'écrivent comme ombre de spectraèdre et y adjoindra d'éventuelle piste pour l'étude des propriétés algorithmiques des spectraèdre nouvellement définis.

 
    \section{Méta-informations}

Cette courte section dévoile quelques informations non-scientifique sur le déroulement du stage.

Pour les activités non liées aux stage de façon immédiate, j'ai pu lors de mon séjour participer à 2 repas organisés respectivement par le laboratoire et par le département dans lequel je me trouvais et j'ai également assisté à deux conférences données sur place, une par mon encadrant Tristan Vaccon et une seconde par un intervenant extérieur. De plus, je mangeais régulièrement avec les chercheurs.

Je disposais d'un bureau dans une salle que je partageais avec 3 autres stagiaires \footnote{tous fort sympathiques}. Il était convenu d'un rendez-vous hebdomadaire afin de discuter des mes avancées ou de mes doutes et interrogations. Cependant, en cas de questionnement je pouvais contacter directement mes encadrants qui étaient présents une majeure partie de la journée.

Le stage a approximativement suivie le déroulement suivant :

La première semaine a été passée à se documenter sur les corps $p$-adiques. La deuxième et la troisième ont été consacrées alternativement à essayer de trouver une définition pertinente de spectraèdre $p$-adique. La quatrième a principalement servie à commencer l'écriture du rapport ainsi qu'à se documenter sur les spectraèdres et l'optimisation convexe afin de dresser des similarités entre spectraèdres réels et $p$-adiques. La cinquième et la sixième ont été utilisées afin de continuer ce rapport démontrer que les couronnes $p$-adiques étaient des spectraèdres et concevoir l'algorithme présenté en \ref{sectionalgo}. L'ultime semaine semaine a permis de peaufiner quelques détails (preuves plus formelles, légères erreurs corrigées, etc.) et de continuer mon rapport de stage. Cependant cette dernière a été peu productive car je devais préparer mon départ et je suis tombé malade (de manière totalement indépendante).

Enfin, je tiens à remercier chaleureusement mes encadrants Tristan Vaccon et Simone Naldi pour, entre autres, leur disponibilité, leurs conseils et leur patience. Mes camarades de bureaux Léo et Lucile qui ont su égayer le bureau ainsi que Abdu Razik pour sa participation à \cite{rozik_borgir_2021}. 


    \section{Spectraèdres réels}
\label{sec:casreel} 
Cette section consiste en une introduction très brève à la notion de spectraèdre réel. Elle n'offre au lecteur que l'outillage nécessaire à la bonne compréhension de l'objet en vue de son adaptation aux corps $p$-adiques, en insistant toutefois sur son usage en optimisation. On ne pourra que conseiller la lecture de \cite{grigoriy_semidefinite_2012} pour des approfondissements.  
\subsection{Définition et premières propriétés}

\begin{definition}
	On appelle \emp{matrice symétrique semi-définie positive} toute matrice réelle symétrique et à valeurs propres positives ou nulles.
	On notera $\mathcal{S}_n^+\left(\mathbb{R}\right) $ l'ensemble de telles matrices et $M \succeq 0$ le fait que $M \in \mathcal{S}_n^+\left(\mathbb{R}\right)$.
\end{definition}

\begin{remarque}
	On remarquera que demander la symétrie permet de s'assurer d'obtenir des valeurs propres réelles grâce au théorème spectral, demander leur positivité fait alors sens.
\end{remarque}
\begin{propriete}
	L'ensemble $\mathcal{S}_n^+\left( \mathbb{R} \right) $ est un cône convexe fermé.
\end{propriete}

\begin{definition}
	On appelle \emp{spectraèdre} l'intersection de $\mathcal{S}_n^+\left(\mathbb{R}\right)$ avec un espace affine $\mathcal{L}$ de $\mathcal{S}_n\left(\mathbb{R}\right)$, l'ensemble des matrices symétriques de taille $n$ sur $\mathbb{R}$.
\end{definition}

En écrivant l'hyperplan $\mathcal{L}$ de $\mathcal{S}_n^+\left(\mathbb{R}\right)$ sous sa forme paramétrique \textit{i.e.} comme l'ensemble des matrices de la forme $A_0 + x_1 A_1 + \ldots x_s A_s$ pour $A_0,\ldots, A_s$ des matrices symétriques fixées on peut définir le spéctraèdre $\mathcal{S} = \mathcal{L} \cap \mathcal{S}_n^+\left(\mathbb{R}\right)$ comme
$\mathcal{S} = \{A := A_0 + x_1A_1 + \ldots + x_s A_s | A \succeq 0, (x_1,\ldots,x_s) \in \mathbb{R}^s\}$. On identifie alors souvent ce dernier à sa préimage dans $\mathbb{R}^s$
$S = \{(x_1,\ldots,x_s) \in \mathbb{R}^s | A_0+ x_1A_1 + \ldots+ x_s A_s \succeq 0 \}. $


  \begin{ex} Un exemple célèbre de spectraèdre est l'ensemble des matrices symétriques semi-définies positives avec diagonale $(1,1,1)$:
    $$
    S = \left\{
    \begin{pmatrix} x_1 \\ x_2 \\ x_3 \end{pmatrix} \in \mathbb{R}^3 |
    A:=\begin{pmatrix} 1 & x_1 & x_2 \\ x_1 & 1 & x_3 \\ x_2 & x_3 & 1 \end{pmatrix} \succeq 0
    \right\}.
    $$
    La surface algébrique définie par $\det A(x_1,x_2,x_3) = 0$ est appelée {\it cubique de Cayley} (\ref{cayley}).
    %Le spectraèdre $S$ est souvent appelé {\it samosa} et peut être obtenu comme {\it dérivée au sens de Renegar} du tetraèdre régulier \cite{sanyal}.
    Les quatre points singuliers correspondent à quatre matrices semi-définies positives de rang un ; les autres points de la surface, correspondent à des matrices de rang deux (semi-définies sur la frontière du spectraèdre, avec au moins une valeur propre négative autrement) ; enfin, les matrices à l'intérieur du spectraèdre sont définies positives (toutes valeurs singulières strictement positives).
    \begin{figure}[!ht]
      \centering
      \includegraphics[scale=0.3]{figures/cayley.pdf}
      \caption{Cubique de Cayley}
      \label{cayley}
    \end{figure}
  \end{ex}


\subsection{Programmation semi-définie}
\label{subsec:psdreele} 
On appelle alors \emp{programmation semi-définie} le problème d'optimisation
consistant à minimiser une application linéaire sur un spéctraèdre que l'on formulera comme :
\begin{equation}
  \tag{PSD}
\begin{aligned}
  \text{Minimiser } & \left\langle c,x \right\rangle \\
  \text{tel que }   & A_0 + \sum \limits_{i=1}^s x_{i}A_{i} \succeq 0
%	\begin{matrix}
%		\text{Minimiser } \simone{\left\langle c,x \right\rangle} \text{ tel que}\\
%		A_0 + \sum \limits_{i=1}^s x_{i}A_{i} \succeq 0
%	\end{matrix}
\end{aligned}
	\label{Psd} 
\end{equation}
pour $A_0,\ldots, A_s$ des matrices symétriques fixées, $c=(c_1,\ldots,c_s)$ un vecteur représentant le coût et $x \mapsto \left\langle c,x \right\rangle := c_1x_1+\cdots+c_sx_s$ le produit scalaire Euclidien. Le problème d'admissibilité associé au problème d'optimisation \eqref{Psd}, c'est-à-dire, la question si le spectraèdre $S = \{(x_1,\ldots,x_s) \in \mathbb{R}^s | A_0+ x_1A_1 + \ldots+ x_s A_s \succeq 0 \}$ est vide, est appelée \emp{inégalité matricielle linéaire (LMI)}.

En précision finie $\epsilon$, ce problème se résout en temps polynomial en la dimension de l'entrée (taille des matrices, nombre de variables, taille binaire des coefficients), en $\log(1/\epsilon)$ et $\log(R)$, où $R$ est une borne {\it a priori} sur la norme d'une solution. En arithmétique exacte, la complexité de la programmation semi-définie est un problème essentiellement ouvert, cf \cite[Sec.1.9]{deKlerk}, \cite{ramana1997exact,porkolab1997complexity} et \cite{henrion2016exact}.

Si à première vue ce problème peut sembler très spécifique il n'en est rien et de nombreux autres problèmes se rapportent à celui-ci. Par exemple, tout problème d'optimisation linéaire est en particulier un problème SDP:

\begin{remarque} Un polyèdre est un spectraèdre; en particulier, l'optimisation linéaire est une sous-classe de l'optimisation semi-définie. En effet, soit $P = \{x \in \mathbb{R}^s | \ell_1(x) \geq 0, \ldots, \ell_d(x) \geq 0\}$ le polyèdre défini par les inégalités linéaires $\ell_1,\ldots,\ell_d$, et soit $D$ la matrice linéaire diagonale avec entrées $\ell_1,\ldots,\ell_d$. Alors $P$ est le spectraèdre défini par $D \succeq 0$.
\end{remarque}

    \section{Introduction aux nombres \texorpdfstring{p}{$p$}-adiques}

On se contentera dans cette section d'une description très élémentaire des différentes définitions et propriétés des nombres $p$-adiques. La plupart des preuves relative à cette section ainsi que de plus amples informations sont disponibles en annexe \todo{l'annexe}. Cette section est très largement inspiré du cours de Xavier Caruso \parencite{caruso_computations_2017} que l'on invite d'ailleurs à aller consulter pour une vision plus complète mais très largement compréhensible.

\begin{notation}
	On considère pour tout ce rapport $p$ un nombre premier.
\end{notation}

\subsection{Entiers \texorpdfstring{p}{$p$}-adique} 
\begin{definition}{Entier $p$-adique }

On appelle entier $p$-adique la somme formelle :
\[
	z  = a_0 + a_1 p + \ldots+a_{n}p^n+\ldots
\]
ou les $a_i$ sont des entiers compris entre $0$ et $p-1$.

\end{definition}

\begin{remarques}
	\begin{itemize}
		\item[$\circ$]  On note $\Zp$ l'ensemble des entier $p-$adiques.
		\item[$\circ$] Par commodité on notera $\ldots a_n\ldots a_1 a_0$ l'entier $p$-adique $\sum a_{i}p^i$ 
\end{itemize}
\end{remarques}

\begin{ex}

	Ainsi les sommes $\sum\limits_{i=0}^{ \infty} p^i = \ldots1111111$ ou $\sum\limits_{i=0}^{ \infty} (i\ \text{mod}\ p) p^i = \ldots210(p-1)\ldots21 $ sont des entiers $p$-adiques parfaitement définis bien que ne convergeant pas dans le cas réel. 
\end{ex}

\begin{propriete}
	$\mathbb{Z}_p$ peut être muni d'une structure d'anneau commutatif intègre en lui adjoignant l'addition terme à terme avec retenue et la multiplication.
\end{propriete}

Par exemple dans $\mathbb{Z}_5$ 

\begin{tabular}{lS}
     & \ldots34202243\\
  $+$& \ldots01423401\\
  \hline
  & \ldots 41131144 \\
  
\end{tabular}

\begin{tabular}{lS}
     & \ldots02243\\
  $\times $& \ldots23401\\
  \hline
  & \ldots 02243\\
  & \ldots 0000 \\
  & \ldots 132\\
  & \ldots 34\\
  $+$ & \ldots 1\\
\hline
& \ldots14443	
\end{tabular}

\begin{propriete}
	 $\mathbb{Z}$ est un sous-anneau de $\mathbb{Z}_p$. 	
\end{propriete}
\textit{Preuve :} Tout entier naturel $a$ admet une décomposition en base $p$ (qui est unique) i.e. s'écrit $a = \sum\limits_{i=0}^{n} a_{i}p^i$ avec $n = \left\lfloor \log_p a\right\rfloor$ et s'associe naturellement à l'élément $\ldots0000a_n\ldots a_0$ de $\mathbb{Z}_p$. Puis à tout entier négatif $b$ on associe l'opposé dans $\mathbb{Z}_p$ de $|b|$. Il n'est alors pas compliqué de vérifier que les opérations de $\mathbb{Z}_p$ restreintes à la projection de $\mathbb{Z}$ coïncident avec les opération dans $\mathbb{Z}$. \hfill \qedsymbol

\begin{remarque}
	Si l'on a vu que les entiers au sens réel était des entiers $p$-adiques, certains entiers $p$-adique ont du sens en tant que nombre rationnels sans être des entiers relatifs, ainsi on a par exemple $\frac{1}{2} = \ldots 2223 \in \mathbb{Z}_5$. Cependant tous les rationnels ne sont pas éléments de $\mathbb{Z}_p$, $\frac{1}{p}$ n'étant par exemple jamais inclus dans $\mathbb{Z}_p$.
\end{remarque}

\subsection{Nombres \texorpdfstring{p}{$p$}-adiques}

\begin{definition} Nombres $p$-adiques 

	On définit l'ensemble $\Qp$ des nombres $p$-adiques comme $\mathbb{Z}_p \left[ \frac{1}{p} \right] $.   
\end{definition}

Un nombre $p$-adique $x$ s'écrit alors comme une somme de la forme $x = \sum \limits_{i=k}^{\infty} x_{i} p^i$ avec $k \in \mathbb{Z}$ et les $x_{i}$ compris entre $0$ et $p-1$. Si $k<0$ on écrira plus couramment $x = \ldots x_i \ldots x_1 x_0 , x_{-1}\ldots x_{k}$. 

\begin{propriete}
	$\mathbb{Q}_{p}$ est un corps qui étend les opérations de $\mathbb{Z}_p$.
\end{propriete}
\textit{Preuve :} Voir \todo{l'annexe}

\todo{exemple d'opérations dans $\mathbb{Q}_{p} $ } 
\begin{corollaire}
	Le corps $\mathbb{Q}$ des rationnels est un sous-corps de $\mathbb{Q}_{p} $.
\end{corollaire}

Ce dernier résultat permet de construire de manière assez élémentaire des éléments de $\mathbb{Q}_{p}$ qui ne sont pas des entiers $p$-adique.

\subsection{Valuation et norme}

	\todo{Petit paragraphe introducif} 
On définit la valuation $p$-adique dans $\mathbb{Z}$ $\val^{\mathbb{Z}}:\mathbb{Z}\to \mathbb{N}\cup \{+\infty\}  $ comme l'application qui à 0 associe $+\infty$ et à un entier $a$ non nul associe le plus grand entier naturel $k$ tel que $p^k | a$% ou de façon équivalente en considérant $\sum \limits_{i=1}^{n} a_{i} p^i$ la décomposition de $a$ en base $p$, la valuation $p$-adique de $a$ est le plus petit $a_{i}$ non nul. 

La valuation $p$-adique s'étend ensuite aux nombres rationnels en une application $\val^{\mathbb{Q}} : \mathbb{Q} \to \mathbb{Z}\cup \{+\infty\}   $ en définissant pour tout $r \in \mathbb{Q}$  $\val^{\mathbb{Q}} \left( r \right) = \val^\mathbb{Z}(a)- \val^\mathbb{Z}\left( b \right) $ avec $a,b \in \mathbb{Z} \times \mathbb{N}^*$ tels que $r=\frac{a}{b}.$   

La valuation $p$-adique s'étend alors également à $\mathbb{Q}_p$ depuis $\mathbb{Q}$ comme suit :
\begin{definition} Valuation $p$-adique
  
	On appelle valuation $p$-adique l'application $\val: \mathbb{Q}_p \to \mathbb{Z}\cup \{+\infty\}  $ qui à un nombre $p$-adique $x$ associe $\max \{k \in \mathbb{Z}\cup \{+\infty\}| x\in p^k \mathbb{Z}_p\}$. 
\end{definition}
Une manière simple de visualiser la valuation d'un nombre $p$-adique est de compter la "distance à la virgule".

En effet, la valuation d'un entier $p$-adique correspond au nombres de $0$ à la fin de son écriture décimale \todo{p-imale ?} et pour un nombre $p$-adique non entier il s'agit de l'opposé nombre de décimales \todo{p-imales ?} après la virgule. Par exemple, $\val\left( \ldots 2413000 \right) = 3$ et $\val\left( \ldots 251,24 \right) = -2$.       

Le principal intérêt qu'offre la notion de valuation pour le sujet développé ici est qu'elle permet de définir une notion de positivité dans un corps qui n'est pas totalement ordonnable \todo{définition peut-être en footnote ?}. À cet effet on introduira la notation suivante :
\begin{notation}
	Pour tout élément $x \in \mathbb{Q}_{p} $, on dit que $x$ est \textit{positif} et on note $x\ge 0$ si $\val\left(x\right)\ge 0$. On en induit alors les notations $x> 0$, $x\le 0$ et $x<0$.  
\end{notation}

        On évitera la notation $x\ge y$ qui pourrait laisser penser de manière trompeuse que $x\ge y \Rightarrow x-y\ge 0$\footnote{Par exemple, $\val(\ldots11,11) \ge \val(\ldots00,01) $ mais $\val(\ldots11,11 - \ldots00,01) = \val( \ldots 11,1) < 0$}.

\begin{propriete}
	La valuation $p$-adique possède les propriétés suivantes, pour tous $x$ et $y$ appartenant à $\mathbb{Q}_{p} $ :   
	\begin{enumerate}
		\item $\val(x+y) \ge \min\left( \val\left( x \right), \val\left( y \right)  \right) $ 
		\item $\val\left( xy \right) = \val (x) + \val( y)$ 
	\end{enumerate}
\end{propriete}
\textit{Preuve :} \todo{annex ?}  
\todo{Demander si mettre des exercices dans un rapport de stage c'est bien vu} 

Ces propriétés permettent alors de munir $\mathbb{Q}_{p} $ d'une valeur absolue \footnote{c'est-à-dire une norme sur $\mathbb{Q}_{p} $ vu comme $\mathbb{Q}_{p} $-espace vectoriel} que l'on définira comme suit:

\begin{definition} Valeur absolue $p$-adique

	On appelle valeur absolue $p$-adique l'application 
\begin{align*}
|\cdot|_p : \mathbb{Q}_{p} & \longrightarrow \mathbb{R}^*_+\\
x & \longmapsto p^{-\val\left(x\right)} 
\end{align*}
\end{definition}

\begin{propriete}
	$\left| \cdot  \right|_p$ est une valeur absolue sur $\mathbb{Q}_{p}$ 
\end{propriete}
\textit{Preuve :} \todo{l'annexe pour changer}

On remarque en particulier d'après \todo{lien vers la prop} que pour tous $x,y \in \mathbb{Q}_{p}, \left|x+y\right|_p \le \max\left( x,y \right)$. Ce qui en fait un espace non archimédien \footnote{c'est-à-dire que $\mathbb{N}$ est borné dans $\left( \mathbb{Q}_{p}, \left| \cdot \right|_p \right) $} et rend la géométrie $p$-adique très différente du cas réel peu intuitive. Ce qui explique le manque de figure et d'explications par le dessin dans la suite de ce rapport.

On terminera cette section en discutant la proposition suivante, qui est d'une importance cruciale puisqu'elle offre une caractérisation simple de la positivité.

\begin{proposition}
	Soit $x \in \mathbb{Q}_{p} $. Les trois propriétés suivantes sont équivalentes 
	\begin{enumerate}[label= \textit{\roman*}.]
		\item $x \in \mathbb{Z}_p$
		\item $\val\left(x\right)\ge 0$
		\item $\left| x \right|_p\le 1$
	\end{enumerate}
\end{proposition}

\textcolor{red}{ On dira alors indistinctement qu'un nombre $x$ est un entier, est un élément de la boule unité ou est positif (conformément à \todo{lien vers la notation}).}

\textit{ Preuve de la propriété : } L'équivalence entre $ii$. et $iii$. découle directement de la définition de $\left| \cdot  \right|_p$. Puis on conclut en remarquant que $x \in \mathbb{Z}_p = p^0 \mathbb{Z}_p$ si et seulement si $\val\left(x\right)\ge 0$ c'est-à-dire $i. \iff ii.$.

    \section{Polyèdres convexes \texorpdfstring{$p$}{p}-adiques}
\label{sec:polyedre} 
\iffalse
\subsection{Matrices symétriques positive} 

\begin{definition}
	On note $\P_n\left( \Qp \right) $ l'ensemble des matrices symétriques dont tous les mineurs principaux ont une valuation positive.
\end{definition} 

\begin{rappel}
	
Un élément de $ \Qp$ a une valuation positive si et seulement si il est élément de $\Zp$. 
\end{rappel}

\begin{propriete}
	
	$\P_n\left( \Qp \right) = \{ M \in S_n\left( \Qp \right)$ | les\- min\-eurs\- prin\-ci\-paux\- de\- $M$ \-sont \-à \-va\-leur \-dans\- $\Zp \} $.
\end{propriete}

	\textit{Preuve :} découle directement du rappel précédent. 
	\medskip


\begin{prop}
	 \[
		 \Pn = S_n\left( \Zp \right) 
	.\]  
\end{prop}

\begin{remarque}
	On remarque alors que l'ensemble $\P_n$ correspond aux matrices symétriques à coefficients positifs.   
\end{remarque}
	\textit{Preuve :}

 Le déterminant étant une fonction polynomiale en les coefficients de la matrice, toute matrice de à coefficient dans $\Zp$ a un déterminant à valeur dans $\Zp$. D'où, par la propriété 2, $S_n\left( \Zp \right) \subset \Pn $.

 L'inclusion réciproque se montre par récurrence. On note pour tout $n \in \mathbf{N}$ $\mathcal{H}_n :  \Pn \subset  S_n(\Zp )$.

 
 On notera $ \Delta_{i_1,\ldots,i_n}\left( M \right) $ le mineur principal de $M$ composé des lignes et des colonnes d'indices $i_1,\ldots,i_n \in \left\{ 1,\ldots,n \right\} $ pour tout matrice $M$. On notera d'ailleurs simplement  $ \Delta_{i_1,\ldots,i_n}$ lorsque le contexte est explicite.

 Les cas $n=0$ et $n=1$ se démontrent sans difficultés aucunes. Montrons le cas $n=2$ qui servira par la suite.

 Soit $M \in \P_2\left( \Qp \right) $, $M$ s'écrit $M = \begin{pmatrix} \alpha & \gamma \\ \gamma & \beta \end{pmatrix}$, avec $\alpha, \beta, \gamma \in \Qp^3$.



 On sait alors que $ \alpha = \Delta_1$ et $ \beta = \Delta_j$ sont des entiers $p$-adiques, il suffit de montrer que $\gamma$ en est également un. Pour ce faire supposons que $ \val (\gamma) < 0$, on a alors $\val ( \gamma^2) = 2 \val\left( \gamma \right) < \val\left( \alpha \beta \right) $ et on en déduit $ \val \left( \Delta_{1,2} \right)= \min\left( \val\left( \alpha \beta \right),2~ \val \left( \gamma \right) \right) = 2 \val\left( \gamma \right) <0 $ ce qui contredit la positivité de $\Delta_{1,2}$ et est donc absurde. On conclut alors que $ \gamma \in \Zp$ et $M \in  S_n\left( \Zp \right) $. On a montré $\mathcal{H}_2$. 

 Soit $n \in \mathbf{N}$ tel que la propriété $\mathcal{H}_n $ soit vérifiée et $M$ une matrice de $\Pn$. 

 $M$ s'écrit 
 \[
 M= \left(\begin{array}{ccc|c}
  &      &     &   \beta_1  \\
  &  M'  &     &\vdots\\
  &      &     &   \beta_n  \\
\hline
\beta_1 &\cdots&  \alpha_n  & \alpha_{n+1}
\end{array} \right)
\]
avec $M' \in S_n\left( \Qp \right) $ et $\beta_1,\ldots, \beta_n, \alpha_{n+1} \in \Qp$.

On note $ \alpha_1,\ldots, \alpha_n$ les coefficients diagonaux de $M'$ qui sont des entiers $p$-adique par hypothèse de récurrence. 


 Par définition $ \alpha_{n+1} = \Delta_{n+1}$ est un entier $p$-adique. Puis on se ramène au cas $n=2$ en utilisant le fait que pour i=1,\ldots, n, $\Delta_{i,n+1} = \begin{vmatrix} \alpha_{i} & \beta_{i}\\ \beta_{i} & \alpha_{n+1} \end{vmatrix} $ et on en déduit que $\beta_{i} \in \Zp$ pour $i=1,\ldots,n$. On conclut en appliquant l'hypothèse de récurrence à $M'$.


\hfill \qedsymbol
\begin{remarque}
	
	La preuve de la proposition précédente montre qu'il suffit en réalité que les mineurs principaux de taille au plus $2$ aient une valuation positive (ou soient éléments de $\Zp$) ce qui correspond à la définition de matrice semi-définie positive sur le semi-corps tropical développée par Allamigeon, Gaubert et Skorma dans \cite{allamigeon_tropical_2020} . Ce n'est toutefois pas la définition qui sera choisie ici, pour des raisons développées en partie \hyperlink{subsection.1.2}{1.2}.
\end{remarque}

\fi


\subsection{Polyèdres convexes \texorpdfstring{$p$}{p}-adiques} 

\begin{definition}
	(Matrice positive)


	Une matrice $M$ de $\mathcal{M}_n(\Qp)$ est dite \emph{positive} si tous ses coefficients sont positifs ou nuls, c'est-à-dire, par définition si elle est à coefficient dans $\mathbb{Z}_p$.
	On notera alors $M\ge 0$ le fait que $M \in \mathcal{M}_{n}\left(\mathbb{Z}_p\right) $

\end{definition}

%Tout ce beau monde ira peut-être en annexe si 
\begin{propriete}
	L'ensemble $\Pn$ est :
	\begin{enumerate}[label=\roman*.]
	\item ouvert
	\item fermé
	\item borné
	\item compact
	\item convexe au sens de \parencite{monna_ensembles_1958}
\end{enumerate}
\end{propriete}
\textit{Preuve :}
$i.$ et $ii.$ se déduisent du fait que $\Zp$ soit ouvert et fermé dans $\Qp$, $ii.$ découle directement du fait que $\|M\|_\infty = \sup |M_{i,j}|_p \le 1$ et $iv.$ se déduit de $ii.$ et $iii.$.
Quand à $v.$ c'est une conséquence directe de la convexité de $\Zp$.

\todo{$\Pn$ est un cône $p$-adique pour la définition :Soit $\mathbb{E} $ un $\mathbb{Q}_{ p } $ espace vectoriel $C \subset \mathbb{E} $ est un cône si pour tout $x \in C$ et $\lambda \ge 0 $ $\lambda x \in C$. La preuve pour $ \Pn$ est assez triviale et pour $S_n^+\left( \mathbb{Q}_p \right)$ elle est laissée en exercice au lecteur} 
\begin{definition}
	
On définit un polyèdre convexe $P$ comme l'intersection de $\Pn$ avec un espace affine $\mathcal{L} $ de $\mathcal{M}_n\left( \Qp \right) $. 
\end{definition}
Comme en \ref{sec:casreel} dans le cas réel on identifiera un polyèdre à sa préimage. Ce qui permet le résultat suivant :


\iffalse
Soit $P$ un polyèdre convexe et $\mathcal{L} $ un plan affine tel que $P = \Pn \cap \mathcal{L}$, et dont on note $s$ la dimension de l'espace vectoriel associé. On dispose de donc de $s +1 $ matrices $M^0,M^1,\ldots, M^s$ telles que $\mathcal{L} = M^0 + \text{Vect}\left( M^1,\ldots,M^s \right)$. Le polyèdre $P$ s'écrit alors $P = \left\{ M^0 + x_1 M^1 + \ldots + x_s M^s \ge 0 \right\}$. On identifie alors souvent $P$ avec $\left\{ \left( x_1,\ldots,x_s \right) \in \Qp^s |M^0 + x_1 M^1 + \ldots + x_s M^s \ge 0 \right\}$ ce qui permet d'écrire qu'un vecteur $x_1,\ldots,x_s$ de $\Qp^s$ est élément de $P$ si et seulement si il vérifie :
 
	\begin{equation}
	\label{eq:1} 
\forall i,j ~  M^0_{i,j} + x_1 M^1_{i,j} + \ldots + M^s_{i,j} \ge 0
	\end{equation}

On peut alors réécrire l'inégalité matricielle en  $P = \left\{ x \in \Qp^s \-| Ax + b \ge 0 \right\} $ avec 
\[A = \begin{pmatrix} M^1_{1,1} & \ldots & M^s_{1,1} \\
\vdots & & \vdots \\
M^1_{n,n} & \ldots & M^s_{1,1}\\ \end{pmatrix} \in M_{n^2,s}\left( \Qp \right) \text{ et } 
b = \begin{pmatrix} M^0_{1,1} \\
\vdots\\
M^0_{n,n} \end{pmatrix} \in M_{n^2, 1}\left( \Qp \right) 
.\]  
\begin{remarque}
	On peut réduire de moitié la tailles des matrices $A$ et $b$ en considérant que les coefficients diagonaux et supradigonaux des $M^i$ pour $i = 0,1,\ldots,n$. Ce qui permet de se ramener à des matrices équivalente\footnote{puisque les inéquations de \ref{eq:1} impliquant le couple $(i,j)$ $i>j$ sont redondantes avec les équations impliquant $(j,i)$}  avec $\frac{n(n+1)}{2}$ lignes.  
\end{remarque}

En mettant sous cette forme le polyèdre on reconnait alors aisément que le problème de minimiser une application linéaire sur un polyèdre correspond exactement à résoudre le problème de la \textit{Programmation linéaire}.

\begin{remarque}
	On se ramène au cas quelconque du problème de la \textit{Programmation linéaire} (taille quelconque et non seulement avec un nombre de ligne en $\frac{n(n+1)}{2}$ ou $n^2$) en annulant des coefficients $M^k_{i,j}$ pour tout $i=1,\ldots,n$.   
\end{remarque}

\begin{propriete}
Un polyèdre $p$-adique convexe est convexe au sens de \cite{monna_ensembles_1958}.  
\end{propriete}

\textit{Preuve :} Provient immédiatement du fait qu'un polyèdre s'écrit comme intersection d'ensemble convexe.
\fi
\begin{ex}
	La boule unité de $\mathbb{Q}_{p}^n$ pour la norme infinie est un polyèdre. En effet, la boule infinie s'écrit comme l'ensemble des points $\left( x_1,\ldots,x_n \right)$ tels que $
	\begin{pmatrix} x_1 &  & & \\
		  & x_2 &0 &  \\
		&  0& \ddots & \\
		&  & & x_n
	\end{pmatrix} \ge 0$. En effet, la boule unité de $\mathbb{Q}_{p} ^n$ est $\mathbb{Z}_p^n$. %En effet, considérons le polyèdre défini par l'intersection de $S_n\left( \mathbb{Q}_{ p }  \right) $ avec le plan linéaire induit par les matrices $E_k$, k=1\ldots n, de $S_n\left( \Zp \right) $ telles que le seul coefficient non nul de $E_k$ soit le k-ième coefficient diagonal lequel est égal à $1$ . On observe que pour tout $x_1,\ldots,x_n \in \mathbb{Q}_{ p } ^n$, $\sum x_k E_k \ge 0$ si et seulement si $x_1,\ldots,x_s \in \mathbb{Z}_{ p }^n $ i.e. si et seulement si $\|(x_1,\ldots,x_n)\|_\infty \le 1$.

\end{ex}

\subsection{Programmation linéaire \texorpdfstring{$p$-adique}{p-adique} } 
\label{sectionalgo} 
Dans ce paragraphe, il sera étudié une forme équivalente du problème de programmation linéaire, appelée \textit{programmation linéaire $p$-adique} (\ref{eqn:Proglinp}) . Lequel consiste simplement à minimiser la norme $p$-adique d'une application linéaire sur un polyèdre $p$-adique. 

	Du fait des natures profondément différentes des polyèdres $p$-adiques et du cas réel, les techniques classiques de résolution sont mises à mal. Il est effet complexe d'appliquer la méthode du simplexe à un ensemble sans frontière ou des techniques d'analyse convexe dans un espace sans notion de convexité. Il convient donc alors de développer de nouvelles techniques pour résoudre ces problèmes. C'est ce qui est proposé dans cette section, qui présente un algorithme en $O( \max\left( m,n \right)^2 )$, avec $m,n$ les dimensions de la matrice de contrainte, pour résoudre le problème de la programmation linéaire en $p$-adique, dont une écriture en pseudo-code ainsi qu'une implémentation en \sage sont disponible en \ref{appendixalgo}.

	Cet algorithme est centré sur l'utilisation de la forme normale de Smith d'une matrice, dont seul la définition et quelques remarques sont présenté dans cette section, les preuves des résultats présentés ici ainsi que d'autres résultats sont disponible en \ref{smith}.

	On appelle \textit{programmation linéaire $p$-adique} le problème :
\begin{equation}
	  \tag{PLp}
\begin{matrix}
	\text{Minimiser } \val\left(\left<c,x \right>\right) \text{ tel que }\\
	Ax + b \ge 0
 \end{matrix}
	    \label{eqn:Proglinp}
\end{equation}

avec $x$ un vecteur de taille $n$ que l'on fait varier, $c$ un vecteur de taille $n$ représentant le coût, $A$ une matrice de taille $m \times n $ et $b$ un vecteur de taille $m$.

La méthode choisie ici consiste à mettre la matrice $A$ sous forme normale de Smith, un factorisation classique en $p$-adique et qui permet de grandement simplifier le problème posé.

\begin{definition}
	Forme Normale de Smith

		On appelle forme normale de Smith d'une matrice $M \in \mathcal{M}_{m,n}\left(\mathbb{Q}_{p} \right) $ de rang $r$ l'unique matrice $S$ de la forme $$S =  
	\begin{pmatrix} p^{a_1} & \\
		 & \ddots \\
		 & & p^{ar}\\
		 & & & 0\\
		 & & & & \ddots \end{pmatrix} $$
		 telle que $a_1\le  \ldots\le a_r$ et $M =  Q^{-1} S P$ avec $P \in \mathcal{G}L_n\left( \mathbb{Z}_p \right) $ et $Q \in \mathcal{G}L_m\left( \mathbb{Z}_p \right) $.
\end{definition}

\todo{donner un exemple} 
\begin{remarques}

	
	\begin{enumerate}[label=\roman*.]
		\item Les coefficients de la forme normale de Smith sont uniques à chaque matrice et sont appelés \textit{facteurs invariants de Smith} ou, plus simplement, \textit{invariants de Smith}.
		\item La valuation $p$-adique du premier coefficient de la forme normale de Smith d'une matrice $M \in \mathcal{M}_n \left( \Qp \right) $ est égale au minimum des valuation des termes de $M$.
		\item En particulier, la forme normale de Smith d'une matrice de $\mathcal{M}_n \left( \mathbb{Z}_p \right) $ est à coefficients dans $\mathbb{Z}_p$ .
		\item Les $r = \text{rang} M$ premiers coefficients diagonaux de $S$ sont exactement ses coefficients non nuls.  
	\end{enumerate}

\end{remarques}


Avant de pouvoir utiliser la forme de normale de Smith pour résoudre \ref{eqn:Proglinp}, il nous faut démontrer le lemme suivant:

\begin{lemme}
Pour tous $z \in \mathbb{Q}_{ p } ^n$ et $M \in \mathcal{M}_{m,n}\left( \mathbb{Q}_{ p }  \right)  $ si $z\ge 0$ et $M\ge 0$ alors $Mz\ge 0$.  
\end{lemme}
\textit{Preuve :} $\mathbb{Z}_p$ est un anneau. \hfill\qedsymbol



En mettant alors la matrice $A$ de \ref{eqn:Proglinp} sous sa forme normale normale de Smith il vient que résoudre \ref{eqn:Proglinp} revient à résoudre :    


\begin{equation}
	  \tag{PLp'}
\begin{matrix}
	\text{Minimiser } \val\left(\left<c',x \right>\right) \text{ tel que }\\
	Sy + b' \ge 0
 \end{matrix}
	    \label{eqn:Proglinp2}
\end{equation}
où $b' = Qb$, $c' = P^Tc$, $S$ est la forme normale de Smith de $A$ et $A = Q^{-1} S P$ avec $P \in \mathcal{G}L_m\left( \mathbb{Z}_p \right)$ et $ Q \in \mathcal{G}L_n\left( \mathbb{Z}_p \right)$.

\begin{remarques}
	
Il en vient immédiatement plusieurs résultats:
\begin{enumerate}
	\item $x^*$ est une solution admissible de \ref{eqn:Proglinp} si et seulement si $y^* := P x^*$ est une solution admissible de \ref{eqn:Proglinp2}
	\item \ref{eqn:Proglinp2} possède des solutions admissible si et seulement si les $m-r$ coefficients de $b'$ sont non nuls, où $r$ le rang de $S$.     
	\item Si les $n-r$ tous derniers coefficients de $c'$ sont non nuls \ref{eqn:Proglinp2} n'est pas borné et n'admet donc pas de solution.   
\end{enumerate}

\end{remarques}

\begin{propriete}
	\begin{itemize}~

		\item[$\circ$] Si les $m-r$ coefficients de $b'$ ne sont pas tous nuls alors il n'existe pas de $y \in \mathbb{Q}_{p} ^n$ tel que $A'y+b' \ge  0$.
	\item[$\circ$] Si les $n-r$ coefficients de $c'$ ne sont pas tous nuls alors le problème n'est pas borné et n'admet donc pas de solution. \end{itemize} 
\end{propriete}
En ne considérant alors que les itérations du problème admettant des solutions on peut réduire le problème en ne considérant que les $r$ premiers coefficients de $y, b', c'$ et la sous matrice de $S$ composée des $r$ premières lignes et colonnes et dont les coefficients sont alors les exactement les facteurs invariants de Smith non nuls. L'ensemble $Adm$ des solutions admissible s'écrit alors comme l'ensemble des vecteurs $y \in \mathbb{Q}_{ p } ^n$ vérifiant $
\forall 1 \le i\le r \ s_i y_i + b'_i \in \mathbb{Z}_p$ c'est-à-dire vérifiant :

\begin{equation}
	\forall 1 \le i\le r  y_i \in -\frac{b’_i}{s_i} + \frac{1}{s_i} \mathbb{Z}_p
\end{equation}
 

Résoudre \ref{eqn:Proglinp2} revient donc à minimiser $\val\left(\left<c',y \right>\right)$ sur $Adm$. L'image de $Adm$ par $y \mapsto \left<c',y \right>$ est $\sum_{i=1}^r -c'_i.\frac{b'_i}{s_i} + \sum_{i=1}^r\left( \frac{c'_i}{s_{i}} \mathbb{Z}_p \right)$ qui se réécrit :
$$c'Adm = \lambda + p^{v} \mathbb{Z}_p$$
où $\lambda = \sum\limits_{i=1}^r -c'_i.\frac{b'_i}{s_i}$ et $v = \min\limits_{1\le i\le r} \val \frac{c'_{i}}{s_{i}} $.
Ainsi, deux cas apparaissent.
\begin{itemize}
	\item Soit $\val( \lambda) < v$ auquel cas le minimum de $y\mapsto \val\left(\left<c',y \right>\right)$ sur $Adm$ est atteint en n'importe quel point de $Adm$ et vaut $\val\left( \lambda\right)$.
	\item Soit $\val\left( \lambda \right) \ge v$, auquel cas $\lambda \in p^{v} \mathbb{Z}_p$ et le minimum vaut $v $ et est atteint en tous les points $y$ de $Adm$ vérifiant
	$$\val\sum \limits_{\begin{array}{c} 1\le i\le r\\ \val\left(c'_{i}/{s_{i}} \right) = v  \end{array}   } y_{i} = 0.$$ 

\end{itemize}
\begin{remarque}
	Si l'on souhaite maximiser la valuation d'un application linéaire su un polyèdre $p$-adique au lieu de la minimiser (ce qui revient à maximiser la valeur absolue) on pourra utiliser un raisonnement similaire à celui présenté dans cette partie. La seule différence est que si $\val\left( \lambda\right) < v$ le problème n'est pas borné et n'admet donc pas de solution. 
\end{remarque}


    \partie{2}{mardi 20 juin 11:00}{Spectraèdres $p$-adiques }
\newcommand\mat{matrice symétrique semie-définie positive } 
\newcommand\Mat{Matrice symétrique semie-définie positive }
\newcommand\mats{matrices symétriques semie-définies positives }
\newcommand\Mats{Matrices symétriques semie-définies positives }


\section{Spectraèdre \texorpdfstring{$p$}{p}-adiques } 
Ce paragraphe tend à fournir un définition de la notion de \mat sur les corps $p-$adiques pour en déduire une définition de spectraèdre qui serait pertinente sur un corps non-archimédien. 
\subsection{\Mats} 

\begin{definition}
	On appelle \mat toute matrice $M \in S_n\left( \Qp \right) $ dont toutes les valeurs propres sont de valuation positive ou nulle[lien du pragarâgre avec les extesnions de corps lolololololololol].

	On note $S_n^+\left( \mathbb{Q}_p \right)$ l'ensemble des \mats.
\end{definition}
\begin{propriete}
	\label{caracsnp}
	Caractérisation des \mats

	Une matrice est symétrique définie positive si et seulement si son polynôme caractéristique est à coefficient dans $\Zp$ .
	
\end{propriete}
	\textit{Preuve :} Voir le LIEEEEEEEEEEEEEEEEEEEEEEEEEEEEEEEEEEEEEEEEEEEEEN0

\begin{consequence}
	$\Pn \subset S_n^+\left( \mathbb{Q}_p \right)$ 
\end{consequence}

\textit{Preuve : }  Le polynôme caractéristique d'une matrice à coefficients dans $\Zp$ étant à coefficient dans $\Zp$ on obtient le résultat par la propriété \ref{caracsnp} .

\begin{remarque}
	En général l'inclusion réciproque est fausse. Ainsi pour $M = \begin{pmatrix} 5 + \frac{3}{5} & \frac{4}{5} \\ \frac{4}{5} & -\frac{3}{5} \end{pmatrix} $, on a $\chi_M = X^2  - 5 X - 4$. Or une fois $M$ plongé dans $\mathbb{Q}_5$ on a $\chi_M \in \mathbb{Z}_5[X]$ donc $M \in S_2^+\left( \mathbb{Q}_5 \right)$ or aucun des coefficients de $M$ n'est dans $S_2^+\left( \mathbb{Q}_5 \right)$

\end{remarque}

\note{vérifier si le fait qu'une matrice snp soit soit positive ou non est lié à la présence ou non de racine uniquement dans $\mathbb{Z}_p$ }.

    \section{Conclusion}
Il a été décrit au cours de ce rapport de nouvelles définitions pour les polyèdres et les spectraèdres sur les corps $p$-adiques. Les polyèdres sont définis par des inégalités de valuation linéaire. On leur a de plus adjoint un algorithme permettant de résoudre la programmation linéaire en $O( \complalgo)$. Afin de définir les spectraèdres sur les corps $p$-adiques il a été nécessaire de trouver une nouvelle définition de matrice semi-définie positive. Pour ce faire l'on s'est débarrassé de la symétrie, obligatoire dans le cas réel, en demandant à ce que les valeurs propres de la matrice soit de valuation positive dans la clôture de $\mathbb{Q}_{p}$. La définition de spectraèdre était alors immédiate et nous a permis de trouver que les couronnes $p$-adiques sont des ombres de spectraèdre, dressant un portrait bien différent du cas réel.

Ces résultats gagneraient toutefois à être d'avantage étudiés, en effet, si la définition des polyèdres dans le cadre choisi semble difficilement pouvoir varier, celle des spectraèdre peut toutefois être discutée. En effet, l'on peut par exemple se demander si l'on obtient de résultats similaires en demandant la symétrie aux matrices semi définies positives, comme dans le cas réel en demandant des condition particulières sur les matrices de sortes à ce que leurs valeurs propres soit dans $\mathbb{Q}_{p} $. De plus, les problèmes informatiques comme la programmation semi-définie ou le problème $\left( LMI \right) $ n'ont pas été résolus (loin s'en faut). Mais il peut toutefois être interessant d'explorer les pistes laissées en \ref{piste} dans d'éventuels travaux futurs. 
 
>>>>>>> afed46df6ee66c8447d25c77be41fc228335ae00

    \newpage
    %Appendice
    \appendix
<<<<<<< HEAD
    \section{Complément sur les corps \texorpdfstring{$p$}{p}-adique}
\label{cpadic}

Cette section de l'appendice présente la plupart des preuves qui n'ont pas été traitées en section \ref{padic} ainsi que quelques compléments sur les nombres $p$-adique pour en avoir une meilleur appréhension.


\subsection*{Représentation sous forme d'arbre}
Un façon intuitive de se représenter les nombres $p$-adiques est de les écrire comme les feuilles d'un arbre infini.

<<<<<<< HEAD
On considère l'arbre $\mathcal{T}(\mathbb{Z}_p)$ dont les nœuds sont les suites finies à coefficients dans $\{1,\ldots,p\} $ et tels que deux sommets sont reliés entre eux si et seulement si 
=======
On considère l'arbre $\mathcal{T}(\mathbb{Z}_p)$ dont les nœuds sont les suites finies à coefficients dans $\{1,\ldots,p\} $ et tels que deux sommets $N_1 \to N_2$ sont reliés entre eux si et seulement si $N_1 \subset N_2$ et $|N_2| = |N_1| + 1$. 

On encode alors les $p$-adique comme une suite de sommets reliés entre eux.


\begin{figure}[htpb]
	\centering
	% https://q.uiver.app/#q=WzAsMTUsWzEsMywiMDAiXSxbMCw1LCIwMDAiXSxbMiw1LCIwMDEiXSxbNCwzLCIwMSJdLFszLDUsIjAxMCJdLFs1LDUsIjAxMSJdLFs3LDUsIjEwMCJdLFs5LDUsIjEwMSJdLFsxMCw1LCIxMTAiXSxbMTIsNSwiMTExIl0sWzMsMSwiMCJdLFs4LDMsIjEwIl0sWzExLDMsIjExIl0sWzksMSwiMSJdLFs2LDAsIlxcYnVsbGV0Il0sWzAsMV0sWzAsMl0sWzMsNF0sWzMsNV0sWzEwLDBdLFsxMCwzXSxbMTEsNl0sWzExLDddLFsxMiw4XSxbMTIsOV0sWzEzLDExXSxbMTMsMTJdLFsxNCwxMF0sWzE0LDEzXV0=
\newcommand\lacolora{purple} 
\newcommand\lacouleur{\color{\lacolora}}  
\[\begin{tikzcd}[sep=tiny]
	&&&&&&& \lacouleur{\bullet} \\
	&&&& \lacouleur{0} &&&&&& 1 \\
	\\
	&& 00 &&& \lacouleur{10} &&&& 01 &&& 11 \\
	\\
	&000 && 100 & \lacouleur{010} && 110 && 001 && 101 & 011 && 111\\
	&\vdots && \vdots & \lacouleur{\vdots} && \vdots && \vdots && \vdots & \vdots && \vdots\\
	\hline\\
	\mathbb{Z}_p&      &&& \lacouleur{\ldots011010}&&&&&&&&&
	\arrow[from=4-3, to=6-2]
	\arrow[from=4-3, to=6-4]
	\arrow[from=4-6, to=6-5, \lacolora]
	\arrow[from=4-6, to=6-7]
	\arrow[from=2-5, to=4-3]
	\arrow[from=2-5, to=4-6, \lacolora]
	\arrow[from=4-10, to=6-9]
	\arrow[from=4-10, to=6-11]
	\arrow[from=4-13, to=6-12]
	\arrow[from=4-13, to=6-14]
	\arrow[from=2-11, to=4-10]
	\arrow[from=2-11, to=4-13]
	\arrow[from=1-8, to=2-5, \lacolora]
	\arrow[from=1-8, to=2-11]
\end{tikzcd}\]

 
	\caption{$\mathcal{T}\left( \mathbb{Z}_p \right) $}
	\label{fig:TZp} 
\end{figure}



Cette interprétation en arbre de $\mathbb{Z}_p$ est utile car permet de se figurer certaine propriétés géométriques des $p$-adiques, puisqu'en effet chaque sommet représente exactement une boule 

>>>>>>> afed46df6ee66c8447d25c77be41fc228335ae00

\begin{proof} \hypertarget{qpcorpspreuve}{Proposition \ref{qpcorps}} 
En réalité on dispose même d'un résultat plus précis : $\mathbb{Q}_{p}$ est le corps des fractions de $\mathbb{Z}_p$. Pour prouver ce résultat on utilisera le lemme suivant :
\begin{lemme}
	\label{lemmeinversible} 
	Les inversibles de $\mathbb{Z}_p$ sont exactement les entiers $p$-adique $\overline{\ldots x_n\ldots x_1x_0}$ tels que $x_0$ est non nul.
\end{lemme}
\textit{Preuve du lemme :} 
Un entier $p$-adique $x = \overline{\ldots x_n \ldots x_1 x_0}$ est inversible si et seulement si il est inversible dans $\mathbb{Z}{/p^n\mathbb{Z}}$ pour tout $n \in \mathbb{N}$, c'est-à-dire si et seulement si $ \sum \limits_{i=0}^{n}p^i x_{i}$ est premier avec $p^n$ pour tout $n \in \mathbb{N}$ ce qui est équivalent à $x_0$ premier avec $p$ et donc $x_0 \neq 0$.

Ensuite il suffit de remarquer que tout entier $p$-adique non nul $x = \overline{\ldots x_n \ldots x_1 x_0}$ s'écrit $p^n \tilde{x}$ avec $\tilde{x} \in \mathbb{Z}_p^\times $ et $n$ un entier naturel. On a de plus unicité par \ref{lemmeinversible}. 

En effet, si on pose $n$ le plus petit entier naturel tel que $x_n \neq 0$\footnote{c'est-à-dire la valuation de $x$} et $\tilde{x} := \overline{\ldots x_n }$ on a immédiatement $ x = p^n \tilde{x}$ et $\tilde{x} \in \mathbb{Z}_p^\times $. Il est alors immédiat que $\mathbb{Q}_{p} = \mathbb{Z}_p[\frac{1}{p}]$ est le plus petit corps contenant $\mathbb{Z}_p$

\end{proof}

\begin{proof} \hypertarget{propvalpreuve}{Propriété \ref{propval} }  
	Soient $x,y$ deux entiers naturels de valuation $n := \val\left(x\right)$ et $m := \val\left(y\right)$.
	On peut alors écrire $x = p^n \tilde{x}$ et $y = p^m \tilde{y}$ avec $\tilde{x}, \tilde{y} \in \mathbb{Z}_p^\times $, comme vu \hyperref{qpcorpspreuve}{plus haut}.
	On a alors tout d'abord $xy = (\tilde{x}\tilde{y}) p^{n+m}$ et donc par unicité de la décomposition $\val\left(xy\right) = \val\left(x\right)+\val\left(y\right) $.

Ensuite, on trouve que $x+y \in p^n \mathbb{Z}_p+ p^m \mathbb{Z}_p = p ^{\min\left( m,n \right) }\mathbb{Z}_p $ ce qui signifie que $\val\left(x+y\right)\ge \min\left( \val\left(x\right), \val\left(y\right) \right) $. Puis si $m\neq n$ on peut supposer sans perte de généralité que $m>n$ et $x+y$ s'écrit alors $p^n(p^{m-n} \tilde{x} + \tilde{y})$ et par \ref{lemmeinversible}, $p^{m-n}\tilde{x}+\tilde{y} \in \mathbb{Z}_p^\times$.  
\end{proof}

<<<<<<< HEAD
\begin{proof} \hypertarget{caracsdppreuve}{Théorème \ref{caracsdp} }

=======
\begin{proof} \hypertarget{caractères}{Théorème \ref{caracsdp} }
	
Afin de prouver le théorème \ref{caracsdp} nous allons reformuler ce dernier sous la forme équivalente suivante : 

	\begin{propriete}
		Un polynôme de $\mathbb{Q}_{p} [X] $\emp{unitaire et de terme constant entier}est a uniquement des racines de valuation positive ou nulle dans $\overline{\mathbb{Q}_{p} }$ si et seulement si il est à coefficients dans $\mathbb{Z}_p$. 
	\end{propriete}
Si prouver le sens réciproque de cette équivalence se fait sans difficulté, le sens direct est, lui plus complexe à démontrer. Pour ce faire, nous utiliserons un outil particulièrement utile pour l'étude des polynômes à coefficients $p$-adique : les polygones de Newton.

\textit{Preuve du sens réciproque :} 
	Soit $P = X^n + \sum \limits_{i=0}^{n-1} a_{i} X^i $ un polynôme à coefficients dans $\mathbb{Z}_{p}$. 
	Montrons par l'absurde que toute racine de $P$ est de valuation positive. Soit $\rho \in \overline{\mathbb{Q}_{p} }$ une racine de $P$, supposons $\val\left( \rho\right)<0$. D'après la propriété \ref{propval}, $\val\left(P( \rho)\right) = \min \left( \val\left( \rho^n \right), \val(a_{n-1} \rho^{n-1}),\ldots, \val(a_0) \right)$. Or on a l'inégalité $\val\left( \rho^n\right) = n \val\left( \rho\right)\le \val\left(a_{i} p^{i} \right) = i\val\left( \rho \right) + \val\left(a_{i}\right)$ pour $i=1,\ldots,n$, car $a_i \in \mathbb{Z}_p$. Ainsi $\val\left(P( \rho)\right) = n \val\left( \rho\right) <0$, or $\rho$ est une racine de $P$ et donc $\val\left( P( \rho) \right) = \val\left(0\right) = +\infty$. Ce qui est absurde, donc $\val\left( \rho\right) \ge 0$.

	\textit{Preuve du sens direct :} 

	Pour la preuve du sens direct on utilisera les polygones de Newton, un outil permettant de relier très facilement les coefficient d'un polynôme à la valuation de ses racines. On ne prouvera pas le principal résultat sur les polygones de Newton dont la preuve peut se retrouver en section 6.4 de \cite{gouvea_p-adic_2003} . 

\begin{definition}
	Polygone de Newton

Soit $P = \sum 	\limits_{i=0}^{n} a_i X^i$ un polynôme à coefficients dans $\mathbb{Q}_{p}$. On appelle \emp{polygone de Newton} l'ensemble des points enveloppe convexe inférieure de l'ensemble  $\mathcal{P} = \{\left( i, \val\left(a_{i}\right) \right)| 1\le i\le n \}$.
\end{definition}

On entend par enveloppe convexe inférieure le graphe de la plus grande fonction convexe $f$ de $[0,n]$ étant "sous" les points de  $\mathcal{P}$, i.e. telle que $f(i) > a_i$ pour $i=1\ldots n$.
 \begin{ex}
	 Par exemple au polynôme $P = 3 + \frac{5}{2} X+ 32 X^2+ \frac{9}{4} X^4+ 44 X^5 + 2 X^6$ on associe 
\begin{figure}[!ht]
      \centering
      \includegraphics[scale=0.25]{figures/polygone_newton.pdf}
      \caption{Polygone de Newton $\mathcal{P}$ associé à $P$}
      \label{poly_newton}
    \end{figure}
  \end{ex}

On pourra alors prouver que l'enveloppe convexe inférieure ainsi définie est une ligne brisée constitués de segments de pentes deux à deux différentes. On appelle pentes du polygones les pentes des segments de la ligne brisée et longueur d'un segment de la ligne brisé la longueur du projeté du segment sur l'axe des abscisses. 

\begin{theoreme}
Les pentes du polygone de Newton $\mathcal{P}$ sont exactement les opposés des valuations des racines de $P$ (dans $\overline{\mathbb{Q}_{p}}$). De plus, le nombre de racines de valuation $v$ est égal à la longueur du segment de pente $-v$.
\end{theoreme}

On considère alors un polynôme $P =X^n \sum_{i=1}^{n-1} a_{i}X^i$ à coefficient dans $\mathbb{Q}_p$ unitaire. On remarque immédiatement que le point d'abscisse $n$ du polygone de Newton $\mathcal{P}$ associé à $P$ est d'ordonnée nulle. On en déduit alors que si un des coefficients $(a_{i})_{0\le i\le n-1}$ est de valuation strictement négative, $\mathcal{P}$ admet une pente strictement positive et donc $P$ admet une racine de valuation strictement négative. 

Par contraposée, on en déduit que si polynôme unitaire à coefficient dans $\mathbb{Q}_{p} $ est à racine positive alors il est élément de $\mathbb{Z}_p[X]$.
>>>>>>> afed46df6ee66c8447d25c77be41fc228335ae00

\end{proof} 
 
    \section{Forme normale de Smith}
\label{smith} 
On revient dans cette partie sur la construction de la forme normale de Smith d'une matrice ainsi que sur les principaux résultats sur cette dernière. Ces résultats sont utilisé dans l'algorithme présenté en section \ref{sectionalgo} 

\begin{propriete}
	Il existe une unique matrice $0$ 


\end{propriete}

\begin{proof}
i
\end{proof}
     
    \section{Résolution de la programmation linéaire \texorpdfstring{$p$}{p}-adique} 
\label{appendixalgo} 
Cette section présente le pseudo-code ainsi qu'une implémentation en \sage de l'algorithme de la section \ref{sectionalgo}.
\usepackage{algorithms}


\renewcommand{\algorithmicrequire}{\textbf{Require:}}
\renewcommand{\algorithmicensure}{\textbf{Ensure:}}
\renewcommand{\algorithmiccomment}[1]{\{#1\}}
\renewcommand{\algorithmicend}{\textbf{end}}
\renewcommand{\algorithmicif}{\textbf{if}}
\renewcommand{\algorithmicthen}{\textbf{then}}
\renewcommand{\algorithmicelse}{\textbf{else}}
\renewcommand{\algorithmicelsif}{\algorithmicelse\ \algorithmicif}
\renewcommand{\algorithmicendif}{\algorithmicend\ \algorithmicif}
\renewcommand{\algorithmicfor}{\textbf{for}}
\renewcommand{\algorithmicforall}{\textbf{for all}}
\renewcommand{\algorithmicdo}{\textbf{do}}
\renewcommand{\algorithmicendfor}{\algorithmicend\ \algorithmicfor}
\renewcommand{\algorithmicwhile}{\textbf{while}}


 

\begin{algorithm}
\caption{Résolution de la programmation $p$-adique}
%\Entry Une matrice $A$ de taille $m \times n$, un vecteur $b$ de taille $m$ et un vecteur $c$ de taille $n$.
\begin{algorithmic}[0]
\State $S,P,Q$ = FormeNormaleDeSmith($A$)
\State $r$ = rang($S$)
\State $b' = P \times b$
\State $c'$ = $c \times Q$
\For{$i$ allant de $r+1$ à $m$}
\If{$\val\left(b'[i] \right) < 0$}
\State Pas de solution
\EndIf
\EndFor
\For{$i$ allant de $r+1$ à $n$} 
\If{$c'[i]\neq 0$}
\State Le problème n'est pas borné
\EndIf
\EndFor
\State $\tilde{c} =$Projection$\left( c', 1, r \right) $ \Comment On réduit le problème à un problème de taille $r$ 
\State $\tilde{b} =$ Projection$\left(b', 1, r \right) $ 
\State $\tilde{S} = $SousMatrice$\left( S, (1,r),(1,r) \right) $
\State $\lambda_0 = \tilde{c}. S^{-1}. \tilde{b} $
\State $v_0 = \min \left\{\val\left(z_{i}\right)| \left( z_1,\ldots,z_r \right) = \tilde{c}.S^{-1}  \right\}$

\If{$\val\left( \lambda_0\right) < v_0$}
\Return $\val\left( \lambda_0 \right)$
\Else \Return $v_0$ \Comment Si l'on cherche à calculer la valuation maximale à la place il suffit de soulever une erreur au lieu de renvoyer $v_0$.
\EndIf
\end{algorithmic}
\end{algorithm}

Ou FormeNormaleDeSmith renvoie la forme normale de Smith de la matrice $A$ ainsi que les matrices $P$ et $Q$ telles que \todo{stick to a convention}. 

L'algorithme s'exécute en $O\left( n^3 \right) $
 

=======
    \section{Complément sur les corps \texorpdfstring{$p$}{p}-adique}
\label{cpadic}

Cette section de l'appendice présente la plupart des preuves qui n'ont pas été traitées en section \ref{padic} ainsi que quelques compléments sur les nombres $p$-adique pour en avoir une meilleur appréhension.


\subsection*{Représentation sous forme d'arbre}
Un façon intuitive de se représenter les nombres $p$-adiques est de les écrire comme les feuilles d'un arbre infini.

<<<<<<< HEAD
On considère l'arbre $\mathcal{T}(\mathbb{Z}_p)$ dont les nœuds sont les suites finies à coefficients dans $\{1,\ldots,p\} $ et tels que deux sommets sont reliés entre eux si et seulement si 
=======
On considère l'arbre $\mathcal{T}(\mathbb{Z}_p)$ dont les nœuds sont les suites finies à coefficients dans $\{1,\ldots,p\} $ et tels que deux sommets $N_1 \to N_2$ sont reliés entre eux si et seulement si $N_1 \subset N_2$ et $|N_2| = |N_1| + 1$. 

On encode alors les $p$-adique comme une suite de sommets reliés entre eux.


\begin{figure}[htpb]
	\centering
	% https://q.uiver.app/#q=WzAsMTUsWzEsMywiMDAiXSxbMCw1LCIwMDAiXSxbMiw1LCIwMDEiXSxbNCwzLCIwMSJdLFszLDUsIjAxMCJdLFs1LDUsIjAxMSJdLFs3LDUsIjEwMCJdLFs5LDUsIjEwMSJdLFsxMCw1LCIxMTAiXSxbMTIsNSwiMTExIl0sWzMsMSwiMCJdLFs4LDMsIjEwIl0sWzExLDMsIjExIl0sWzksMSwiMSJdLFs2LDAsIlxcYnVsbGV0Il0sWzAsMV0sWzAsMl0sWzMsNF0sWzMsNV0sWzEwLDBdLFsxMCwzXSxbMTEsNl0sWzExLDddLFsxMiw4XSxbMTIsOV0sWzEzLDExXSxbMTMsMTJdLFsxNCwxMF0sWzE0LDEzXV0=
\newcommand\lacolora{purple} 
\newcommand\lacouleur{\color{\lacolora}}  
\[\begin{tikzcd}[sep=tiny]
	&&&&&&& \lacouleur{\bullet} \\
	&&&& \lacouleur{0} &&&&&& 1 \\
	\\
	&& 00 &&& \lacouleur{10} &&&& 01 &&& 11 \\
	\\
	&000 && 100 & \lacouleur{010} && 110 && 001 && 101 & 011 && 111\\
	&\vdots && \vdots & \lacouleur{\vdots} && \vdots && \vdots && \vdots & \vdots && \vdots\\
	\hline\\
	\mathbb{Z}_p&      &&& \lacouleur{\ldots011010}&&&&&&&&&
	\arrow[from=4-3, to=6-2]
	\arrow[from=4-3, to=6-4]
	\arrow[from=4-6, to=6-5, \lacolora]
	\arrow[from=4-6, to=6-7]
	\arrow[from=2-5, to=4-3]
	\arrow[from=2-5, to=4-6, \lacolora]
	\arrow[from=4-10, to=6-9]
	\arrow[from=4-10, to=6-11]
	\arrow[from=4-13, to=6-12]
	\arrow[from=4-13, to=6-14]
	\arrow[from=2-11, to=4-10]
	\arrow[from=2-11, to=4-13]
	\arrow[from=1-8, to=2-5, \lacolora]
	\arrow[from=1-8, to=2-11]
\end{tikzcd}\]

 
	\caption{$\mathcal{T}\left( \mathbb{Z}_p \right) $}
	\label{fig:TZp} 
\end{figure}



Cette interprétation en arbre de $\mathbb{Z}_p$ est utile car permet de se figurer certaine propriétés géométriques des $p$-adiques, puisqu'en effet chaque sommet représente exactement une boule 

>>>>>>> afed46df6ee66c8447d25c77be41fc228335ae00

\begin{proof} \hypertarget{qpcorpspreuve}{Proposition \ref{qpcorps}} 
En réalité on dispose même d'un résultat plus précis : $\mathbb{Q}_{p}$ est le corps des fractions de $\mathbb{Z}_p$. Pour prouver ce résultat on utilisera le lemme suivant :
\begin{lemme}
	\label{lemmeinversible} 
	Les inversibles de $\mathbb{Z}_p$ sont exactement les entiers $p$-adique $\overline{\ldots x_n\ldots x_1x_0}$ tels que $x_0$ est non nul.
\end{lemme}
\textit{Preuve du lemme :} 
Un entier $p$-adique $x = \overline{\ldots x_n \ldots x_1 x_0}$ est inversible si et seulement si il est inversible dans $\mathbb{Z}{/p^n\mathbb{Z}}$ pour tout $n \in \mathbb{N}$, c'est-à-dire si et seulement si $ \sum \limits_{i=0}^{n}p^i x_{i}$ est premier avec $p^n$ pour tout $n \in \mathbb{N}$ ce qui est équivalent à $x_0$ premier avec $p$ et donc $x_0 \neq 0$.

Ensuite il suffit de remarquer que tout entier $p$-adique non nul $x = \overline{\ldots x_n \ldots x_1 x_0}$ s'écrit $p^n \tilde{x}$ avec $\tilde{x} \in \mathbb{Z}_p^\times $ et $n$ un entier naturel. On a de plus unicité par \ref{lemmeinversible}. 

En effet, si on pose $n$ le plus petit entier naturel tel que $x_n \neq 0$\footnote{c'est-à-dire la valuation de $x$} et $\tilde{x} := \overline{\ldots x_n }$ on a immédiatement $ x = p^n \tilde{x}$ et $\tilde{x} \in \mathbb{Z}_p^\times $. Il est alors immédiat que $\mathbb{Q}_{p} = \mathbb{Z}_p[\frac{1}{p}]$ est le plus petit corps contenant $\mathbb{Z}_p$

\end{proof}

\begin{proof} \hypertarget{propvalpreuve}{Propriété \ref{propval} }  
	Soient $x,y$ deux entiers naturels de valuation $n := \val\left(x\right)$ et $m := \val\left(y\right)$.
	On peut alors écrire $x = p^n \tilde{x}$ et $y = p^m \tilde{y}$ avec $\tilde{x}, \tilde{y} \in \mathbb{Z}_p^\times $, comme vu \hyperref{qpcorpspreuve}{plus haut}.
	On a alors tout d'abord $xy = (\tilde{x}\tilde{y}) p^{n+m}$ et donc par unicité de la décomposition $\val\left(xy\right) = \val\left(x\right)+\val\left(y\right) $.

Ensuite, on trouve que $x+y \in p^n \mathbb{Z}_p+ p^m \mathbb{Z}_p = p ^{\min\left( m,n \right) }\mathbb{Z}_p $ ce qui signifie que $\val\left(x+y\right)\ge \min\left( \val\left(x\right), \val\left(y\right) \right) $. Puis si $m\neq n$ on peut supposer sans perte de généralité que $m>n$ et $x+y$ s'écrit alors $p^n(p^{m-n} \tilde{x} + \tilde{y})$ et par \ref{lemmeinversible}, $p^{m-n}\tilde{x}+\tilde{y} \in \mathbb{Z}_p^\times$.  
\end{proof}

<<<<<<< HEAD
\begin{proof} \hypertarget{caracsdppreuve}{Théorème \ref{caracsdp} }

=======
\begin{proof} \hypertarget{caractères}{Théorème \ref{caracsdp} }
	
Afin de prouver le théorème \ref{caracsdp} nous allons reformuler ce dernier sous la forme équivalente suivante : 

	\begin{propriete}
		Un polynôme de $\mathbb{Q}_{p} [X] $\emp{unitaire et de terme constant entier}est a uniquement des racines de valuation positive ou nulle dans $\overline{\mathbb{Q}_{p} }$ si et seulement si il est à coefficients dans $\mathbb{Z}_p$. 
	\end{propriete}
Si prouver le sens réciproque de cette équivalence se fait sans difficulté, le sens direct est, lui plus complexe à démontrer. Pour ce faire, nous utiliserons un outil particulièrement utile pour l'étude des polynômes à coefficients $p$-adique : les polygones de Newton.

\textit{Preuve du sens réciproque :} 
	Soit $P = X^n + \sum \limits_{i=0}^{n-1} a_{i} X^i $ un polynôme à coefficients dans $\mathbb{Z}_{p}$. 
	Montrons par l'absurde que toute racine de $P$ est de valuation positive. Soit $\rho \in \overline{\mathbb{Q}_{p} }$ une racine de $P$, supposons $\val\left( \rho\right)<0$. D'après la propriété \ref{propval}, $\val\left(P( \rho)\right) = \min \left( \val\left( \rho^n \right), \val(a_{n-1} \rho^{n-1}),\ldots, \val(a_0) \right)$. Or on a l'inégalité $\val\left( \rho^n\right) = n \val\left( \rho\right)\le \val\left(a_{i} p^{i} \right) = i\val\left( \rho \right) + \val\left(a_{i}\right)$ pour $i=1,\ldots,n$, car $a_i \in \mathbb{Z}_p$. Ainsi $\val\left(P( \rho)\right) = n \val\left( \rho\right) <0$, or $\rho$ est une racine de $P$ et donc $\val\left( P( \rho) \right) = \val\left(0\right) = +\infty$. Ce qui est absurde, donc $\val\left( \rho\right) \ge 0$.

	\textit{Preuve du sens direct :} 

	Pour la preuve du sens direct on utilisera les polygones de Newton, un outil permettant de relier très facilement les coefficient d'un polynôme à la valuation de ses racines. On ne prouvera pas le principal résultat sur les polygones de Newton dont la preuve peut se retrouver en section 6.4 de \cite{gouvea_p-adic_2003} . 

\begin{definition}
	Polygone de Newton

Soit $P = \sum 	\limits_{i=0}^{n} a_i X^i$ un polynôme à coefficients dans $\mathbb{Q}_{p}$. On appelle \emp{polygone de Newton} l'ensemble des points enveloppe convexe inférieure de l'ensemble  $\mathcal{P} = \{\left( i, \val\left(a_{i}\right) \right)| 1\le i\le n \}$.
\end{definition}

On entend par enveloppe convexe inférieure le graphe de la plus grande fonction convexe $f$ de $[0,n]$ étant "sous" les points de  $\mathcal{P}$, i.e. telle que $f(i) > a_i$ pour $i=1\ldots n$.
 \begin{ex}
	 Par exemple au polynôme $P = 3 + \frac{5}{2} X+ 32 X^2+ \frac{9}{4} X^4+ 44 X^5 + 2 X^6$ on associe 
\begin{figure}[!ht]
      \centering
      \includegraphics[scale=0.25]{figures/polygone_newton.pdf}
      \caption{Polygone de Newton $\mathcal{P}$ associé à $P$}
      \label{poly_newton}
    \end{figure}
  \end{ex}

On pourra alors prouver que l'enveloppe convexe inférieure ainsi définie est une ligne brisée constitués de segments de pentes deux à deux différentes. On appelle pentes du polygones les pentes des segments de la ligne brisée et longueur d'un segment de la ligne brisé la longueur du projeté du segment sur l'axe des abscisses. 

\begin{theoreme}
Les pentes du polygone de Newton $\mathcal{P}$ sont exactement les opposés des valuations des racines de $P$ (dans $\overline{\mathbb{Q}_{p}}$). De plus, le nombre de racines de valuation $v$ est égal à la longueur du segment de pente $-v$.
\end{theoreme}

On considère alors un polynôme $P =X^n \sum_{i=1}^{n-1} a_{i}X^i$ à coefficient dans $\mathbb{Q}_p$ unitaire. On remarque immédiatement que le point d'abscisse $n$ du polygone de Newton $\mathcal{P}$ associé à $P$ est d'ordonnée nulle. On en déduit alors que si un des coefficients $(a_{i})_{0\le i\le n-1}$ est de valuation strictement négative, $\mathcal{P}$ admet une pente strictement positive et donc $P$ admet une racine de valuation strictement négative. 

Par contraposée, on en déduit que si polynôme unitaire à coefficient dans $\mathbb{Q}_{p} $ est à racine positive alors il est élément de $\mathbb{Z}_p[X]$.
>>>>>>> afed46df6ee66c8447d25c77be41fc228335ae00

\end{proof} 

    \section{Forme normale de Smith}
\label{smith} 
On revient dans cette partie sur la construction de la forme normale de Smith d'une matrice ainsi que sur les principaux résultats sur cette dernière. Ces résultats sont utilisé dans l'algorithme présenté en section \ref{sectionalgo} 

\begin{propriete}
	Il existe une unique matrice $0$ 


\end{propriete}

\begin{proof}
i
\end{proof}

    \section{Résolution de la programmation linéaire \texorpdfstring{$p$}{p}-adique} 
\label{appendixalgo} 
Cette section présente le pseudo-code ainsi qu'une implémentation en \sage de l'algorithme de la section \ref{sectionalgo}.
\usepackage{algorithms}


\renewcommand{\algorithmicrequire}{\textbf{Require:}}
\renewcommand{\algorithmicensure}{\textbf{Ensure:}}
\renewcommand{\algorithmiccomment}[1]{\{#1\}}
\renewcommand{\algorithmicend}{\textbf{end}}
\renewcommand{\algorithmicif}{\textbf{if}}
\renewcommand{\algorithmicthen}{\textbf{then}}
\renewcommand{\algorithmicelse}{\textbf{else}}
\renewcommand{\algorithmicelsif}{\algorithmicelse\ \algorithmicif}
\renewcommand{\algorithmicendif}{\algorithmicend\ \algorithmicif}
\renewcommand{\algorithmicfor}{\textbf{for}}
\renewcommand{\algorithmicforall}{\textbf{for all}}
\renewcommand{\algorithmicdo}{\textbf{do}}
\renewcommand{\algorithmicendfor}{\algorithmicend\ \algorithmicfor}
\renewcommand{\algorithmicwhile}{\textbf{while}}


 

\begin{algorithm}
\caption{Résolution de la programmation $p$-adique}
%\Entry Une matrice $A$ de taille $m \times n$, un vecteur $b$ de taille $m$ et un vecteur $c$ de taille $n$.
\begin{algorithmic}[0]
\State $S,P,Q$ = FormeNormaleDeSmith($A$)
\State $r$ = rang($S$)
\State $b' = P \times b$
\State $c'$ = $c \times Q$
\For{$i$ allant de $r+1$ à $m$}
\If{$\val\left(b'[i] \right) < 0$}
\State Pas de solution
\EndIf
\EndFor
\For{$i$ allant de $r+1$ à $n$} 
\If{$c'[i]\neq 0$}
\State Le problème n'est pas borné
\EndIf
\EndFor
\State $\tilde{c} =$Projection$\left( c', 1, r \right) $ \Comment On réduit le problème à un problème de taille $r$ 
\State $\tilde{b} =$ Projection$\left(b', 1, r \right) $ 
\State $\tilde{S} = $SousMatrice$\left( S, (1,r),(1,r) \right) $
\State $\lambda_0 = \tilde{c}. S^{-1}. \tilde{b} $
\State $v_0 = \min \left\{\val\left(z_{i}\right)| \left( z_1,\ldots,z_r \right) = \tilde{c}.S^{-1}  \right\}$

\If{$\val\left( \lambda_0\right) < v_0$}
\Return $\val\left( \lambda_0 \right)$
\Else \Return $v_0$ \Comment Si l'on cherche à calculer la valuation maximale à la place il suffit de soulever une erreur au lieu de renvoyer $v_0$.
\EndIf
\end{algorithmic}
\end{algorithm}

Ou FormeNormaleDeSmith renvoie la forme normale de Smith de la matrice $A$ ainsi que les matrices $P$ et $Q$ telles que \todo{stick to a convention}. 

L'algorithme s'exécute en $O\left( n^3 \right) $


    \newpage
>>>>>>> afed46df6ee66c8447d25c77be41fc228335ae00
    \printbibliography
    
\end{document}
