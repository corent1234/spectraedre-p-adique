\section*{Abstract}

	Les spectraèdres sont une des généralisations les plus communes et utilisées des polyèdres. Ce sont des ensembles réels convexes définis comme les ensembles de matrices linéaires $A(x) = A_0 + x_1 A_1 + \ldots + x_s A_s$ qui sont symétriques définies positives, avec $x = \left( x_1,\ldots,x_s \right)$ parcourant $\in \mathbb{R}^s $ et $A_0,A_1,\ldots,A_s$ des matrices symétriques réelles fixées.
Ils sont d'une grande importance en optimisation car au cœur de la programmation semi-définie, une généralisation de la programmation linéaire visant à minimiser une forme linéaire sur un spectraèdre au lieu et en place d'un polyèdre convexe. Ainsi, une large variété de problèmes peuvent se réduire à la programmation semi-définie. En effet, aux nombreuses applications de la programmation linéaire (problème de la coupe maximale, nombreuses applications en finance, médecine, dans l'industrie\ldots) s'ajoutent des problèmes plus spécifiques dont un certain nombre est par exemple présentés dans \cite{vandenberghe_applications_1999}. Si les spectraèdres ont été largement étudiés dans le cas réel, en 2016  Allamigeon, Gaubert et Skomra ont ouvert la voie à l'étude des spéctraèdres sur les corps non-archimédien dans \cite{allamigeon_tropical_2020} en définissant les spectraèdres tropicaux.

L'objectif de ce stage était alors d'essayer de trouver une définition de spectraèdre générale sur les corps non-archimédien. (Rappelons qu'un corps (valué) non-archimédien est un corps muni d'une valeur absolue\footnote{une valeur absolue sur un corps est une norme sur ce même corps compatible avec le produit.} $\left| \cdot  \right|$ vérifiant l'inégalité non-archimédienne : pour tous $ x,y \in \mathbb{K}  \left| x+y \right|\le \max\left( \left| x \right|, \left| y \right|\right)$ ). En s'appuyant, pour ce faire, sur les corps $p$-adiques. Un corps $p$-adique $\mathbb{Q}_{p}$ étant un ensemble de nombres définis par les sommes formelles $x = \ldots + x_k p^k + \ldots + x_1 p + x_0 +x_{-1} p^{-1} + \ldots x_{n}p^n$ avec $n \in \mathbb{Z}$ et où les $(x_k)_{k\ge m}$ sont éléments de $\{0,\ldots,p-1\} $, pour $p$ un nombre premier fixé. On peut munir ces corps de la valuation $p$-adique $\val\left(x\right) = \sup \{i \ge m | x_{i}\neq 0\} $ qui définit une valeur absolue non-archimédienne $\left| \cdot  \right|_p : x\mapsto p^{-\val\left(x\right)} $. 

Ce rapport propose une définition de spectraèdre sur les corps $p$-adiques. Pour ce faire, on introduira une nouvelle notion, celle de matrice semi-définie positive. Une matrice à coefficients $p$-adiques étant semi-définie positive si et seulement si les racines de son polynôme caractéristiques sont de valuation positive. Ce qui permet de définir les spéctraèdres comme les matrices linéaires $A(x) = A_0 + x_1 A_1 + \ldots x_s A_s$ semi-définies positives avec $x = \left( x_1,\ldots,x_s \right)$ parcourant $\mathbb{Q}_{p} ^s$ et $A_0,\ldots,A_s$ quelconques fixées. On peut alors en déduire le principal résultat découvert lors de ce stage : les couronnes $p$-adiques sont des projections de spectraèdres. De plus, ce rapport définit les polyèdres $p$-adiques comme les ensembles de points $x \in \mathbb{Q}_{p} ^s$ définis par inégalités de la forme $\val\left( \ell_1(x)\right)\ge 0,\ldots, \val\left( \ell_n\left( x \right) \right)\ge 0$ pour $\ell_1,\ldots, \ell_n$ des formes linéaires $p$-adiques.
Ainsi, si le temps a manqué pour proposer des résultats sur la résolution de la programmation semi-définie positive $p$-adique, il sera toute fois présenté un algorithme résolvant la programmation linéaire en $O\left( \complalgo \right) $, où $m$ et $n$ sont les dimension de la matrice de contraintes. De plus, des pistes pour la résolution du problème $LMI$ ( \textit{Linear Matrix Inequality}) $p$-adique visant à déterminer la vacuité ou non d'un spectraèdre sont également évoquées. 
