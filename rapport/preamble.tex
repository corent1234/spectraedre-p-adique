% Some basic packages
\usepackage[utf8]{inputenc}
\usepackage[T1]{fontenc}
\usepackage{textcomp}
\usepackage[french]{babel}
\usepackage{url}
\usepackage{graphicx}
\usepackage{float}
%\usepackage{booktabs}
%\usepackage{fontenc}

\usepackage{xlop}%pour poser des additions

\usepackage{listings}
\usepackage{hyperref}
\hypersetup{
    colorlinks=true,
    linkcolor=blue,
    filecolor=magenta,      
    urlcolor=cyan,
    }

\urlstyle{same}
%\pdfminorversion=7

% Don't indent paragraphs, leave some space between them
%\usepackage{parskip}

% Hide page number when page is empty
\usepackage{emptypage}
\usepackage{subcaption}
\usepackage{multicol}
\usepackage[dvipsnames]{xcolor}

% Other font I sometimes use.
% \usepackage{cmbright}

% Math stuff
\usepackage{amsmath, amsfonts, mathtools, amsthm, amssymb}
% Fancy script capitals
\usepackage{mathrsfs}
\usepackage{cancel}
% Bold math
\usepackage{bm}
% Some shortcuts
\newcommand\N{\ensuremath{\mathbb{N}}}
\newcommand\R{\ensuremath{\mathbb{R}}}
\newcommand\Z{\ensuremath{\mathbb{Z}}}
\renewcommand\O{\ensuremath{\emptyset}}
\newcommand\Q{\ensuremath{\mathbb{Q}}}
\newcommand\C{\ensuremath{\mathbb{C}}}
\newcommand\Qp{ \ensuremath {\mathbb{Q}_p} }
\newcommand\Zp{ \ensuremath{ \mathbb {Z}_p }} 
\newcommand\val{ \ensuremath{\text{val}_p} }  
\renewcommand\P{ \mathcal{P} } 
\newcommand\Pn{ \mathcal{M}_n \left( \Zp \right)} 
% Easily typeset systems of equations (French package)
\usepackage{systeme}

% Put x \to \infty below \lim
\let\svlim\lim\def\lim{\svlim\limits}

%Make implies and impliedby shorter
\let\implies\Rightarrow
\let\impliedby\Leftarrow
\let\iff\Leftrightarrow
\let\epsilon\varepsilon

% Add \contra symbol to denote contradiction
\usepackage{stmaryrd} % for \lightning
\newcommand\contra{\scalebox{1.5}{$\lightning$}}

\let\phi\varphi

% Command for short corrections
% Usage: 1+1=\correct{3}{2}

\definecolor{correct}{HTML}{009900}
\newcommand\correct[2]{\ensuremath{\:}{\color{red}{#1}}\ensuremath{\to }{\color{correct}{#2}}\ensuremath{\:}}
\newcommand\green[1]{{\color{correct}{#1}}}

% horizontal rule
\newcommand\hr{
    \noindent\rule[0.5ex]{\linewidth}{0.5pt}
[dvipsnames][dvipsnames]}

% hide parts
\newcommand\hide[1]{}

% si unitx
\usepackage{siunitx}
\sisetup{locale = FR}

% Environments
\makeatother
% For box around Definition, Theorem, \ldots
\usepackage{mdframed}
\mdfsetup{skipabove=.2em,skipbelow=0em}
\theoremstyle{definition}
\newmdtheoremenv[nobreak=true]{consequence}{Conséquence}[subsection]
\newmdtheoremenv[nobreak=true]{propriete}[consequence]{Propriété}
\newmdtheoremenv[nobreak=true]{proprietes}[consequence]{Propriétés}
\newmdtheoremenv[nobreak=true]{lemme}[consequence]{Lemme}
\newmdtheoremenv[nobreak=true]{proposition}[consequence]{Proposition}
\newmdtheoremenv[nobreak=true]{theoreme}[consequence]{Théorème}
\newmdtheoremenv[nobreak=true]{loi}[consequence]{Loi}
\newmdtheoremenv[nobreak=true]{postulat}[consequence]{Postulat}
\newmdtheoremenv{conclusion}[consequence]{Conclusion}
\newmdtheoremenv{corollaire}[consequence]{Corollaire}
\newmdtheoremenv{corrollaire}[consequence]{Corollaire}
\newmdtheoremenv{conjecture}[consequence]{Conjecture}
\newtheorem*{rappel}{Rappel}
\newtheorem*{remarque}{Remarque}
\newtheorem*{remarques}{Remarques}
\newtheorem*{notation}{Notation}
\newtheorem*{observation}{Observation}
\newtheorem*{exo}{Exercice}
\newtheorem*{praktisch}{Praktisch}
\newtheorem*{probleme}{Problème}
\newtheorem*{terminologie}{Terminologie}
\newtheorem*{application}{Application}
\newtheorem*{uovt}{UOVT}
\newtheorem*{ex}{Exemple}
\newtheorem*{vraag}{Vraag}

\newmdtheoremenv[nobreak=true]{definition}[consequence]{Définition}
\newtheorem*{eg}{Example}
%\newtheorem*{notation}{Notation}
\newtheorem*{previouslyseen}{As previously seen}
\newtheorem*{remark}{Remark}
\newtheorem*{note}{Note}
\newtheorem*{problem}{Problem}
\newtheorem*{observe}{Observe}
\newtheorem*{property}{Property}
\newtheorem*{intuition}{Intuition}
\newmdtheoremenv[nobreak=true]{prop}{Proposition}
\newmdtheoremenv[nobreak=true]{theorem}{Theorem}
\newmdtheoremenv[nobreak=true]{corollary}{Corollary}

% End example and intermezzo environments with a small diamond (just like proof
% environments end with a small square)
\usepackage{etoolbox}
\AtEndEnvironment{vb}{\null\hfill$\diamond$}%
\AtEndEnvironment{intermezzo}{\null\hfill$\diamond$}%
% \AtEndEnvironment{opmerking}{\null\hfill$\diamond$}%

% Fix some spacing
% http://tex.stackexchange.com/questions/22119/how-can-i-change-the-spacing-before-theorems-with-amsthm
\makeatletter
\def\thm@space@setup{%
  \thm@preskip=\parskip \thm@postskip=0pt
}


% Exercise 
% Usage:
% \oefening{5}
% \suboefening{1}
% \suboefening{2}
% \suboefening{3}
% gives
% Oefening 5
%   Oefening 5.1
%   Oefening 5.2
%   Oefening 5.3
\newcommand{\oefening}[1]{%
    \def\@oefening{#1}%
    \subsection*{Oefening #1}
}

\newcommand{\suboefening}[1]{%
    \subsubsection*{Oefening \@oefening.#1}
}


% \lecture starts a new lecture (les in dutch)
%
% Usage:
% \lecture{1}{di 12 feb 2019 16:00}{Inleiding}
%
% This adds a section heading with the number / title of the lecture and a
% margin paragraph with the date.

% I use \dateparts here to hide the year (2019). This way, I can easily parse
% the date of each lecture unambiguously while still having a human-friendly
% short format printed to the pdf.

\usepackage{xifthen}
\def\testdateparts#1{\dateparts#1\relax}
\def\dateparts#1 #2 #3 #4 #5\relax{
    \marginpar{\small\textsf{\mbox{#1 #2 #3 #5}}}
}

\def\@partie{}%
\newcommand{\partie}[3]{
    \ifthenelse{\isempty{#3}}{%
        \def\@partie{Partie #1}%
    }{%
        \def\@partie{Partie #1: #3}%
    }%
    \subsection*{\@partie}
    \marginpar{\small\textsf{\mbox{#2}}}
}



% These are the fancy headers
\usepackage{fancyhdr}
\pagestyle{fancy}

% LE: left even
% RO: right odd
% CE, CO: center even, center odd
% My name for when I print my lecture notes to use for an open book exam.
\fancyhead[LE,RO]{Corentin Cornou}

\fancyhead[RO,LE]{\leftmark} % Right odd,  Left even
\fancyhead[RE,LO]{}          % Right even, Left odd

\fancyfoot[RO,LE]{\thepage}  % Right odd,  Left even
\fancyfoot[RE,LO]{}          % Right even, Left odd
\fancyfoot[C]{\leftmark}     % Center

\makeatother




% Todonotes and inline notes in fancy boxes
\usepackage{todonotes}
\usepackage{tcolorbox}

% Make boxes breakable
\tcbuselibrary{breakable}

% Verbetering is correction in Dutch
% Usage: 
% \begin{verbetering}
%     Lorem ipsum dolor sit amet, consetetur sadipscing elitr, sed diam nonumy eirmod
%     tempor invidunt ut labore et dolore magna aliquyam erat, sed diam voluptua. At
%     vero eos et accusam et justo duo dolores et ea rebum. Stet clita kasd gubergren,
%     no sea takimata sanctus est Lorem ipsum dolor sit amet.
% \end{verbetering}
\newenvironment{correction}{\begin{tcolorbox}[
    arc=0mm,
    colback=white,
    colframe=green!60!black,
    title=Opmerking,
    fonttitle=\sffamily,
    breakable
]}{\end{tcolorbox}}

% Noot is note in Dutch. Same as 'verbetering' but color of box is different
\newenvironment{notes}[1]{\begin{tcolorbox}[
    arc=0mm,
    colback=white,
    colframe=white!60!black,
    title=#1,
    fonttitle=\sffamily,
    breakable
]}{\end{tcolorbox}}




% Figure support as explained in my blog post.
\usepackage{import}
\usepackage{xifthen}
\usepackage{pdfpages}
\usepackage{transparent}
\newcommand{\incfig}[1]{%
    \def\svgwidth{\columnwidth}
    \import{./figures/}{#1.pdf_tex}
}

% Fix some stuff
% %http://tex.stackexchange.com/questions/76273/multiple-pdfs-with-page-group-included-in-a-single-page-warning
\pdfsuppresswarningpagegroup=1

\usepackage{csquotes}
\usepackage[style=alphabetic, sorting=ynt]{biblatex}
\usepackage{enumitem}
% My name
\author{Corentin Cornou} 

\usepackage{algorithm}
\usepackage{algpseudocode}

\usepackage{pdfpages} 

<<<<<<< HEAD
%% macro Simone
\newcommand\simone[1]{{\color{blue} #1}}
=======
%to create trees
\usepackage{tikz-cd} 
%% macro Simone
\newcommand\simone[1]{{\color{blue} #1}}


>>>>>>> afed46df6ee66c8447d25c77be41fc228335ae00
