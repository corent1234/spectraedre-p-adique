\section{Polyèdres convexes \texorpdfstring{$p$}{p}-adiques}
\begin{notation}

	Similairement à, pour toute matrice $M$ à coefficient dans $\Qp$ on note $M\ge 0$ et on dit que $M$ est \textit{positive} si tous les coefficients de $M$  sont positifs, c'est à dire si $M \in M_n\left( \Zp \right) $ . On infère également les notation $M\le 0$, $M>0$ et $M<0$.     
\end{notation}

\subsection{Matrices symétriques positive} 

\begin{definition}
	On note $\P_n\left( \Qp \right) $ l'ensemble des matrices symétriques dont tous les mineurs principaux ont une valuation positive.
\end{definition} 

\begin{rappel}
	
Un élément de $ \Qp$ a une valuation positive si et seulement si il est élément de $\Zp$. 
\end{rappel}

\begin{propriete}
	
	$\P_n\left( \Qp \right) = \{ M \in S_n\left( \Qp \right)$ | les\- min\-eurs\- prin\-ci\-paux\- de\- $M$ \-sont \-à \-va\-leur \-dans\- $\Zp \} $.
\end{propriete}

	\textit{Preuve :} découle directement du rappel précédent. 
	\medskip


\begin{prop}
	 \[
		 \Pn = S_n\left( \Zp \right) 
	.\]  
\end{prop}

\begin{remarque}
	On remarque alors que l'ensemble $\P_n$ correspond aux matrices symétriques à coefficients positifs.   
\end{remarque}
	\textit{Preuve :}

 Le déterminant étant une fonction polynomiale en les coefficients de la matrice, toute matrice de à coefficient dans $\Zp$ a un déterminant à valeur dans $\Zp$. D'où, par la propriété 2, $S_n\left( \Zp \right) \subset \Pn $.

 L'inclusion réciproque se montre par récurrence. On note pour tout $n \in \mathbf{N}$ $\mathcal{H}_n :  \Pn \subset  S_n(\Zp )$.

 
 On notera $ \Delta_{i_1,\ldots,i_n}\left( M \right) $ le mineur principal de $M$ composé des lignes et des colonnes d'indices $i_1,\ldots,i_n \in \left\{ 1,\ldots,n \right\} $ pour tout matrice $M$. On notera d'ailleurs simplement  $ \Delta_{i_1,\ldots,i_n}$ lorsque le contexte est explicite.

 Les cas $n=0$ et $n=1$ se démontrent sans difficultés aucunes. Montrons le cas $n=2$ qui servira par la suite.

 Soit $M \in \P_2\left( \Qp \right) $, $M$ s'écrit $M = \begin{pmatrix} \alpha & \gamma \\ \gamma & \beta \end{pmatrix}$, avec $\alpha, \beta, \gamma \in \Qp^3$.



 On sait alors que $ \alpha = \Delta_1$ et $ \beta = \Delta_j$ sont des entiers $p$-adiques, il suffit de montrer que $\gamma$ en est également un. Pour ce faire supposons que $ \val (\gamma) < 0$, on a alors $\val ( \gamma^2) = 2 \val\left( \gamma \right) < \val\left( \alpha \beta \right) $ et on en déduit $ \val \left( \Delta_{1,2} \right)= \min\left( \val\left( \alpha \beta \right),2~ \val \left( \gamma \right) \right) = 2 \val\left( \gamma \right) <0 $ ce qui contredit la positivité de $\Delta_{1,2}$ et est donc absurde. On conclut alors que $ \gamma \in \Zp$ et $M \in  S_n\left( \Zp \right) $. On a montré $\mathcal{H}_2$. 

 Soit $n \in \mathbf{N}$ tel que la propriété $\mathcal{H}_n $ soit vérifiée et $M$ une matrice de $\Pn$. 

 $M$ s'écrit 
 \[
 M= \left(\begin{array}{ccc|c}
  &      &     &   \beta_1  \\
  &  M'  &     &\vdots\\
  &      &     &   \beta_n  \\
\hline
\beta_1 &\cdots&  \alpha_n  & \alpha_{n+1}
\end{array} \right)
\]
avec $M' \in S_n\left( \Qp \right) $ et $\beta_1,\ldots, \beta_n, \alpha_{n+1} \in \Qp$.

On note $ \alpha_1,\ldots, \alpha_n$ les coefficients diagonaux de $M'$ qui sont des entiers $p$-adique par hypothèse de récurrence. 


 Par définition $ \alpha_{n+1} = \Delta_{n+1}$ est un entier $p$-adique. Puis on se ramène au cas $n=2$ en utilisant le fait que pour i=1,\ldots, n, $\Delta_{i,n+1} = \begin{vmatrix} \alpha_{i} & \beta_{i}\\ \beta_{i} & \alpha_{n+1} \end{vmatrix} $ et on en déduit que $\beta_{i} \in \Zp$ pour $i=1,\ldots,n$. On conclut en appliquant l'hypothèse de récurrence à $M'$.


\hfill \qedsymbol
\begin{remarque}
	
	La preuve de la proposition précédente montre qu'il suffit en réalité que les mineurs principaux de taille au plus $2$ aient une valuation positive (ou soient éléments de $\Zp$) ce qui correspond à la définition de matrice semi-définie positive sur le semi-corps tropical développée par Allamigeon, Gaubert et Skorma dans \cite{allamigeon_tropical_2020} . Ce n'est toutefois pas la définition qui sera choisie ici, pour des raisons développées en partie \hyperlink{subsection.1.2}{1.2}.
\end{remarque}

\begin{propriete}
	$\Pn$ est un anneau.
\end{propriete}

\textit{Preuve :} Par la propriété précédente $\Pn = S_n \left( \Zp \right)$ comme intersection des anneaux $M_n \left( \Qp \right)$ et $S_n\left( \Qp \right) $.
\begin{propriete}
	L'ensemble $\Pn$ est :
	\begin{enumerate}[label=\roman*.]
	\item ouvert
	\item fermé
	\item borné
	\item compact
	\item convexe au sens de \parencite{monna_mo58_1958}  
\end{enumerate}
\end{propriete}
\textit{Preuve :}
$i.$ et $ii.$ se déduisent du fait que $\Zp$ soit ouvert et fermé dans $\Qp$, $ii.$ découle directement du fait que $\|M\|_\infty = \sup |M_{i,j}|_p \le 1$ et $iv.$ se déduit de $ii.$ et $iii.$.
Quand à $v.$ c'est une conséquence directe de la convexité de $\Zp$ et $S_n\left( \Qp \right) $.


\todo{$\Pn$ est un cône $p$-adique pour la définition :Soit $\mathbb{E} $ un $\mathbb{Q}_{ p } $ espace vectoriel $C \subset \mathbb{E} $ est un cône si pour tout $x \in C$ et $\lambda \ge 0 $ $\lambda x \in C$. La preuve pour $ \Pn$ est assez triviale et pour $S_n^+\left( \mathbb{Q}_p \right)$ elle est laissée en exercice au lecteur} 
\subsection{Polyèdres convexes \texorpdfstring{$p$}{p}-adiques} 


\begin{definition}
	
On définit un polyèdre convexe $P$ comme l'intersection de $\Pn$ avec un hyperplan affine $\mathcal{L} $ de $S_n\left( \Qp \right) $. 
\end{definition}

Soit $P$ un polyèdre convexe et $\mathcal{L} $ un plan affine tel que $P = \Pn \cap \mathcal{L}$, et dont on note $s$ la dimension de l'espace vectoriel associé. On dispose de donc de $s +1 $ matrices $M^0,M^1,\ldots, M^s$ telles que $\mathcal{L} = M^0 + \text{Vect}\left( M^1,\ldots,M^s \right)$. Le polyèdre $P$ s'écrit alors $P = \left\{ M^0 + x_1 M^1 + \ldots + x_s M^s \ge 0 \right\}$. On identifie alors souvent $P$ avec $\left\{ \left( x_1,\ldots,x_s \right) \in \Qp^s |M^0 + x_1 M^1 + \ldots + x_s M^s \ge 0 \right\}$ ce qui permet d'écrire qu'un vecteur $x_1,\ldots,x_s$ de $\Qp^s$ est élément de $P$ si et seulement si il vérifie :
 
	\begin{equation}
	\label{eq:1} 
\forall i,j ~  M^0_{i,j} + x_1 M^1_{i,j} + \ldots + M^s_{i,j} \ge 0
	\end{equation}

On peut alors réécrire l'inégalité matricielle en  $P = \left\{ x \in \Qp^s \-| Ax + b \ge 0 \right\} $ avec 
\[A = \begin{pmatrix} M^1_{1,1} & \ldots & M^s_{1,1} \\
\vdots & & \vdots \\
M^1_{n,n} & \ldots & M^s_{1,1}\\ \end{pmatrix} \in M_{n^2,s}\left( \Qp \right) \text{ et } 
b = \begin{pmatrix} M^0_{1,1} \\
\vdots\\
M^0_{n,n} \end{pmatrix} \in M_{n^2, 1}\left( \Qp \right) 
.\]  
\begin{remarque}
	On peut réduire de moitié la tailles des matrices $A$ et $b$ en considérant que les coefficients diagonaux et supradigonaux des $M^i$ pour $i = 0,1,\ldots,n$. Ce qui permet de se ramener à des matrices équivalente\footnote{puisque les inéquations de \ref{eq:1} impliquant le couple $(i,j)$ $i>j$ sont redondantes avec les équations impliquant $(j,i)$}  avec $\frac{n(n+1)}{2}$ lignes.  
\end{remarque}

En mettant sous cette forme le polyèdre on reconnait alors aisément que le problème de minimiser une application linéaire sur un polyèdre correspond exactement à résoudre le problème de la \textit{Programmation linéaire}.

\begin{remarque}
	On se ramène au cas quelconque du problème de la \textit{Programmation linéaire} (taille quelconque et non seulement avec un nombre de ligne en $\frac{n(n+1)}{2}$ ou $n^2$) en annulant des coefficients $M^k_{i,j}$ pour tout $i=1,\ldots,n$.   
\end{remarque}

\begin{propriete}
	Un polyèdre $p$-adique convexe est convexe au sens de \cite{monna_mo58_1958}.  
\end{propriete}

\textit{Preuve :} Provient immédiatement du fait qu'un polyèdre s'écrit comme intersection d'ensemble convexe.

\begin{ex}
	La boule unité pour la norme infinie est un polyèdre. En effet, considérons le polyèdre défini par l'intersection de $S_n\left( \mathbb{Q}_{ p }  \right) $ avec le plan linéaire induit par les matrices $E_k$, k=1\ldots n, de $S_n\left( \Zp \right) $ telles que le seul coefficient non nul de $E_k$ soit le k-ième coefficient diagonal lequel est égal à $1$ . On observe que pour tout $x_1,\ldots,x_n \in \mathbb{Q}_{ p } ^n$, $\sum x_k E_k \ge 0$ si et seulement si $x_1,\ldots,x_s \in \mathbb{Z}_{ p }^n $ i.e. si et seulement si $\|(x_1,\ldots,x_n)\|_\infty \le 1$. 

\end{ex}

\subsection{Résolution \texorpdfstring{p}{$p$}-adique de la programmation linéaire}

Dans ce paragraphe, il sera étudié une forme équivalente du problème de programmation linéaire, appelée \textit{programmation linéaire $p$-adique}. Lequel consiste simplement à minimiser la norme $p$-adique d'une application linéaire sur un polyèdre $p$-adique. 

	Du fait des natures profondément différentes des polyèdres $p$-adiques et du cas réel, les techniques classiques de résolution sont mises à mal. Il est effet complexe d'appliquer la méthode du simplexe à un ensemble sans frontière ou des techniques d'analyse convexe dans un espace sans notion de convexité. Il convient donc alors de développer de nouvelles techniques pour résoudre ces problèmes. C'est ce qui est proposé dans cette section, qui présente un algorithme en $O(n^3)$ \todo{vérifier que c'est améliorable} pour résoudre le problème de la programmation linéaire en $p$-adique.

	Cet algorithme est centré sur l'utilisation de la forme normale de Smith d'une matrice, dont seul la définition et quelques remarques sont présenté dans cette section, les preuves des résultats présentés ici ainsi que d'autres résultats sont disponible en annexe. \todo{faire le lien (et l'annexe)} 
i

On appelle programmation linéaire $p$-adique le problème :
\begin{equation}
	  \tag{PLp}
\begin{matrix}
	\text{Minimiser } |c.x|_p \text{ tel que }\\
	Ax + b \ge 0
 \end{matrix}
	    \label{eqn:Proglinp}
\end{equation}

avec $x$ le vecteur de taille $n$ que l'on fait varier, $c$ un vecteur de taille $n$ représentant le coût, $A$ une matrice de taille $m \times n $ et $b$ un vecteur de taille $m$.

\todo{la transition} 

\begin{definition}
	Forme Normale de Smith

	Pour toute matrice $M \in \mathcal{M}_{m,n}\left( \mathbb{Q}_{ p }  \right) $ il existe une unique matrice $S$ de $\mathcal{M}_{m,n}\left( \mathbb{Q}_{ p }  \right) $ diagonale, dont les coefficients sont triés par valuation croissante et telle que
	\[
		M = P^{-1} S Q
	\]
 avec $P \in \mathcal{G}L_m\left( \mathbb{Z}_p \right)$ et $ Q \in \mathcal{G}L_n\left( \mathbb{Z}_p \right)$. La matrice $S$ est appelée forme normale de Smith de $M$.
\end{definition}

\todo{donner un exemple} 
\begin{remarques}
~

	\begin{enumerate}[label=\roman*.]
		\item Les coefficients de la forme normale de Smith sont uniques à chaque matrice et sont appelés \textit{facteurs invariants de Smith} ou, plus simplement, \textit{invariants de Smith}.
		\item La valuation $p$-adique du premier coefficient de la forme normale de Smith d'une matrice $M \in \mathcal{M}_n \left( \Qp \right) $ est égale au minimum des valuation des termes de $M$.
		\item En particulier, la forme normale de Smith d'une matrice de $\mathcal{M}_n \left( \mathbb{Z}_p \right) $ est à coefficients dans $\mathbb{Z}_p$ .
		\item Seuls les $r = \text{rang} M$ premiers coefficients diagonaux de $S$ sont non nuls.  
	\end{enumerate}

\end{remarques}

Avant de pouvoir utiliser la forme de normale de Smith pour résoudre \ref{eqn:Proglinp}, il nous faut démontrer le lemme suivant:

\begin{lemme}
	Pour tous $z \in \mathbb{Q}_{ p } ^n$ et $M \in \mathcal{M}_{m,n}\left( \mathbb{Q}_{ p }  \right)  $ si $z\ge 0$ et $M\ge 0$ alors $Mz\ge 0$.  
\end{lemme}
\textit{Preuve :} Pour tous $z \in \mathbb{Q}_{ p } ^n$ et $M \in \mathcal{M}_{m,n}\left( \mathbb{Q}_{ p }  \right)  $,  $z\ge 0$ et $M\ge 0$ si et seulement si  $M$ et $z$ sont à coefficients dans $\mathbb{Z}_p$, or les coefficients de $Mz$ s'écrivant comme sommes et produits de coefficients de $M$ et $z$, on a $Mz \in \mathbb{Z}_p^m$ i.e. $Mz\ge 0$.       
\hfill\qedsymbol



En mettant alors la matrice $A$ de \ref{eqn:Proglinp} sous sa forme normale normale de Smith il vient que résoudre \ref{eqn:Proglinp} revient à résoudre :    


\begin{equation}
	  \tag{PLp'}
\begin{matrix}
	\text{Minimiser } |c'.y|p \text{ tel que }\\
	Sy + b' \ge 0
 \end{matrix}
	    \label{eqn:Proglinp2}
\end{equation}
où $b' = Pb$, $c' = cQ$, $S$ est la forme normale de Smith de $A$ et $A = P S Q$ avec $P \in \mathcal{G}L_m\left( \mathbb{Z}_p \right)$ et $ Q \in \mathcal{G}L_n\left( \mathbb{Z}_p \right)$.

Il en vient immédiatement plusieurs résultats:
\begin{enumerate}
	\item $x^*$ est une solution admissible de \ref{eqn:Proglinp} si et seulement si $y^* := Q^{-1} x$ est une solution admissible de \ref{eqn:Proglinp2}
	\item \ref{eqn:Proglinp2} possède des solutions admissible si et seulement si les $m-r$ coefficients de $Pb$ sont non nuls, où $r$ le rang de $S$.     
	\item Si les $m-r$ derniers coefficients de $cQ$ sont non nuls \ref{eqn:Proglinp2} n'est pas borné et n'admet donc pas de solution.   
\end{enumerate}

Il sort de ces remarques que si les $m-r$ derniers coefficients de $b'$ et de $c'$ ne sont pas tous nuls, le problème n'admet pas de solution. Sinon on peut réduire la problème en ne considérant que les $r$ premiers coefficients de $y, b', c'$ et la sous matrice de $S$ composée des $r$ premières lignes et colonnes et dont les coefficients sont alors les exactement les facteurs invariants de Smith non nuls. L'ensemble $Adm$ des solutions admissible s'écrit alors comme l'ensemble des vecteurs $y \in \mathbb{Q}_{ p } ^n$ vérifiant $
\forall 1 \le i\le r \ s_i y_i + b'_i \in \mathbb{Z}_p$ c'est-à-dire vérifiant :

\[ \boxed{ 
\forall 1 \le i\le r \ y_i \in -\frac{b'_i}{s_i} + \frac{1}{s_i} \mathbb{Z}_p
}.\]  

Résoudre \ref{eqn:Proglinp2} revient donc à minimiser $|c'.y|_p$ sur $Adm$. Or l'image de $Adm$ par $y \mapsto c'y$ est $\sum_{i=1}^r -c'_i.\frac{b'_i}{s_i} + \sum_{i=1}^r\left( \frac{c_i}{s_{i}} \mathbb{Z}_p \right)$ qui se réécrit :
$$c'Adm = \lambda_0 + p^{v_0} \mathbb{Z}_p$$
où $\lambda_0 = \sum\limits_{i=1}^r -c'_i.\frac{b'_i}{s_i}$ et $v_{0} = \min\limits_{1\le i\le r} \val \frac{c'_{i}}{s_{i}} $.

Ainsi, deux cas apparaissent. Soit $\val( \lambda_0) < v_0$ auquel cas le minimum de $y\mapsto |c'.y|_p$ sur $Adm$ est atteint en n'importe quel point de $Adm$ et vaut $| \lambda_0|_p$. Soit $\val\left( \lambda_0 \right) \ge v_0$, auquel cas $\lambda_0 \in p^{v_0} \mathbb{Z}_p$ et le minimum vaut $p^{-v_0} $ et est atteint en tous les points $y$ de $Adm$ tels que $\left|\sum \limits_{1\le i\le r} y_{i}\right|_p = 1$.

