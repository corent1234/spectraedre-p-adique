\section{Introduction} 
Les spectraèdres se présentent comme une généralisation des polyèdres, définis comme les points $x = (x_1,\ldots,x_s)$pour lesquels une matrice linéaire $A(x) = A_0 + x_1A_1+\ldots+x_s A_s$ est symétrique semi-définie positives, avec $A_0,\ldots,A_s$ des matrice symétriques. Ils sont d'une grand utilité en optimisation car ils permettent de résoudre non seulement des problèmes d'optimisation linéaire mais également d'autres plus spécifiques. En effet, une sur-classe des problèmes de programmation linéaire, les problèmes de programmation semi-définie consiste à maximiser une application linéaire sur un spectraèdre.

Si ces objets ont été largement étudiés dans le cadre réel, une voie vers leur étude dans des corps non archimédiens s'est récemment ouverte dans \cite{allamigeon_tropical_2020}. C'est ce que ce rapport étudie dans le cadre des corps $p$-adiques, corps non-archimédiens qui peuvent être vus comme des extensions du coprs $\mathbb{Q}$ des rationnels autres que le corps des réels et dans lesquels les techniques traditionnelles ne marchent pas. Ainsi, le produit scalaire n'est en $p$-adique pas une forme bilinéaire particulièrement distinguée, ce qui annule tout bénéfice de la symétrie et le corps $p$-adique $\mathbb{Q}_{p}$ n'est de plus pas algébriquement clos. Il a donc fallu trouver une nouvelle définition de matrice semi-définie positive. Celle choisie ici est celle des matrices dont la valuation $p$-adique des valeurs propres est positive dans la clôture de $\mathbb{Q}_{p}$. On en déduit alors aisément une définition de spectraèdre $p$-adique et prouve qu'avec cette dernière les couronnes $p$-adiques sont des projetés de spectraèdre.


Ce rapport commence par décrire le déroulement du stage. Suite à quoi, les spectraèdres sont définis et quelques unes de leur propriétés décrites. Puis, suit une présentation élémentaire des corps $p$-adiques. Ensuite, les polyèdres seront définis dans le cas $p$-adique, et un algorithme résolvant le problème de la programmation linéaire sera décrit. Enfin, on y construira une définition des spectraèdre $p$-adiques, basé sur la nouvelle définition de matrice semi-définie positive après avoir brièvement discuté des clôtures algébriques des corps $p$-adiques. Ultimement, on prouvera que les couronnes $p$-adiques s'écrivent comme ombre de spectraèdre et y adjoindra d'éventuelle piste pour l'étude des propriétés algorithmiques des spectraèdre nouvellement définis.

