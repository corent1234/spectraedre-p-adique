\section{Conclusion}
Il a été décrit au cours de ce rapport de nouvelles définitions pour les polyèdres et les spectraèdres sur les corps $p$-adiques. Les polyèdres sont définis par des inégalités de valuation linéaire. On leur a de plus adjoint un algorithme permettant de résoudre la programmation linéaire en $O( \complalgo)$. Afin de définir les spectraèdres sur les corps $p$-adiques il a été nécessaire de trouver une nouvelle définition de matrice semi-définie positive. Pour ce faire l'on s'est débarrassé de la symétrie, obligatoire dans le cas réel, en demandant à ce que les valeurs propres de la matrice soit de valuation positive dans la clôture de $\mathbb{Q}_{p}$. La définition de spectraèdre était alors immédiate et nous a permis de trouver que les couronnes $p$-adiques sont des ombres de spectraèdre, dressant un portrait bien différent du cas réel.

Ces résultats gagneraient toutefois à être d'avantage étudiés, en effet, si la définition des polyèdres dans le cadre choisi semble difficilement pouvoir varier, celle des spectraèdre peut toutefois être discutée. En effet, l'on peut par exemple se demander si l'on obtient de résultats similaires en demandant la symétrie aux matrices semi définies positives, comme dans le cas réel en demandant des condition particulières sur les matrices de sortes à ce que leurs valeurs propres soit dans $\mathbb{Q}_{p} $. De plus, les problèmes informatiques comme la programmation semi-définie ou le problème $\left( LMI \right) $ n'ont pas été résolus (loin s'en faut). Mais il peut toutefois être interessant d'explorer les pistes laissées en \ref{piste} dans d'éventuels travaux futurs. 
