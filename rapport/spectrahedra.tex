\newcommand\mat{matrice semi-définie positive } 
\newcommand\Mat{Matrice semi-définie positive }
\newcommand\mats{matrices semi-définies positives }
\newcommand\Mats{Matrices semi-définies positives }


\section{Spectraèdre \texorpdfstring{$p$}{p}-adiques } 
Ce paragraphe tend à fournir un définition de la notion de \mat sur les corps $p-$adiques pour en déduire une définition de spectraèdre qui serait pertinente sur un corps non-archimédien. 
La première étape pour définir un spectraèdre est de définir un équivalent des matrices symétriques définies positives. Sans théorème spectrale et le produit scalaire n'étant qu'une forme bilinéaire "banale" (ni positive ni définie), la symétrie est en $p$-adique parfaitement inutile et ne sera pas exigé. De plus, se par le manque cruel du théorème spectral la plupart des caractérisations des matrices symétriques semi-définies positives peinent à faire sens en $p$-adique. 
Il a donc été choisi de définir les \mats \footnote{notez l'absence du mot symétrique}comme les matrices à valeurs propres positives.
Or là un second problème se pose : $\mathbb{Q}_{p} $ n'est pas algébriquement clos et les matrices $\mathcal{M}_n\left( \mathbb{Q}_{p}  \right) $ peuvent donc avoir des valeurs propres hors de $\mathbb{Q}_{p}$. Ce problème sera réglé en étendant la valuation $p$-adique aux extensions de $\mathbb{Q}_{p} $.

\subsection{Clôture algébrique $p$-adique \texorpdfstring{p}{$p$}-adique } 

\begin{definition}
	On appelle \textit{extension de corps} d'un corps $\mathbb{K}$ tout corps $\mathbb{L}$ muni d'un morphisme de corps injectif de $\mathbb{K}$ dans $\mathbb{L}$. On note $\mathbb{L}/\mathbb{K}$ le fait que $\mathbb{L}$ soit une extension de $\mathbb{K}.$
\end{definition}


\begin{definition}
	Un corps $\mathbb{K}$ est dit algébriquement clos si tout polynôme $P\in \mathbb{K}[X]$ de degré au moins $1$ possède une racine dans $\mathbb{K}$.
\end{definition}

\begin{proposition}
	$\mathbb{Q}_{p} $ n'est pas algébriquement clos.
\end{proposition}
\textit{Preuve :} Le polynôme $X^2 - p$ n'a pas de racine dans $\mathbb{Q}_{p}$ \hfill \qedsymbol.

\begin{definition}
	On appelle clôture algébrique l'unique (à isomorphisme près) corps $\mathbb{L}$ tel que tout élément de $\mathbb{L}$ est racine d'un polynôme de $\mathbb{K}[X]$ et $\mathbb{L}$ est algébriquement clos.
\end{definition}

On notera $\overline{\mathbb{Q}_{p} }$ la clôture algébrique de $\mathbb{Q}_{p}$, il est alors possible d'y étendre la valuation $p$-adique comme suit :

\begin{definition}
	On définit la valuation $p$-adique sur $\overline{\mathbb{Q}_{p} }$ comme 
\begin{align*}
	\val : \overline{\mathbb{Q}_{p}} & \longrightarrow \mathbb{Q}\cup \{+\infty\} \\
x & \longmapsto \val^{\mathbb{Q}_{p} } \left(a\right)/d
\end{align*}
	où $d$ et $a$ sont respectivement le degré et le terme constant du polynôme minimal de $x$.
\end{definition}

On retrouve alors une notion de positivité dans l'extension de corps et étendra à $\overline{\mathbb{Q}_{p}} $ la notation $x\ge 0$ si et seulement si $\val\left(x\right)\ge 0$. Il n'est cependant informatiquement pas très pratique de travailler dans $\overline{\mathbb{Q}_{p} }$ qui est n'est pas une extension finie de $\mathbb{Q}_{p} $ %\footnote{c'est-à-dire $\overline{\mathbb{Q}_{p} }$ est un $\mathbb{Q}_{p}$-espace vectoriel de dimension infinie }. On essayera donc de trouver des caractérisations permettant de savoir plus simplement si un polynôme est à racines positives, ce qui sera fait dans la prochaine section.

\subsection{\Mats} 

\begin{definition}
	On appelle \mat toute matrice $M \in \mathcal{M}_n\left( \Qp \right) $ dont toutes les valeurs propres sont de valuation positive ou nulle.

	On note $ \mathcal{M}_n^+\left( \mathbb{Q}_p \right)$ l'ensemble des \mats.
\end{definition}
\begin{propriete}
	\label{caracsnp}
	Caractérisation des \mats

	Une matrice est symétrique définie positive si et seulement si son polynôme caractéristique est à coefficient dans $\Zp$ .
	
\end{propriete}
	\textit{Preuve :} Voir le LIEEEEEEEEEEEEEEEEEEEEEEEEEEEEEEEEEEEEEEEEEEEEEN0

\begin{consequence}
	$\Pn \subset \mathcal{M}_n^+\left( \mathbb{Q}_p \right)$ 
\end{consequence}

\textit{Preuve : }  Le polynôme caractéristique d'une matrice à coefficients dans $\Zp$ étant à coefficient dans $\Zp$ on obtient le résultat par la propriété \ref{caracsnp} .

\begin{remarque}
	En général l'inclusion réciproque est fausse. Ainsi pour $M = \begin{pmatrix} 5 + \frac{3}{5} & \frac{4}{5} \\ \frac{4}{5} & -\frac{3}{5} \end{pmatrix} $, on a $\chi_M = X^2  - 5 X - 4$. Or une fois $M$ plongé dans $\mathbb{Q}_5$ on a $\chi_M \in \mathbb{Z}_5[X]$ donc $M \in S_2^+\left( \mathbb{Q}_5 \right)$ or aucun des coefficients de $M$ n'est dans $\mathbb{Z}_5 $
\end{remarque}

	\todo{probablement pas nécessaire ira peut-être en annexe} 
\begin{propriete}
	L'ensemble $S_n^+\left( \mathbb{Q}_p \right)$  est :
	\begin{enumerate}[label = \textit{\roman*}.]
		\item ouvert
		\item fermé
		\item un cône
	\end{enumerate}
\end{propriete}

\textit{Preuve :}  
Montrons tout d'abord que $\Zp[X]$ est fermé et ouvert dans $\mathbb{Q}_{ p }[X] $ muni de la norme infinie $\|.\|_\infty : P = \sum_{k=1}^{n} a_k X^k \to \sup |a_k|_p$. 

Soit $P \in \Zp[X]$. On peut alors montrer que la boule ouverte de centre $P$ et de rayon $1$ est incluse dans $\Zp[X]$. Pour ce faire on considère $Q \in B_{o}\left( P,1 \right)_\infty $ et on remarque que cela signifie que $\|Q-P\|_\infty \le 1$ c'est-à-dire $Q-P \in \Zp[X]$. On a donc $Q = P + P-Q \in \Zp[X]$. 
$\Zp[X]$ est donc ouvert.

Maintenant soit $P \in \mathbb{Q}_{ p } [X]^c$ et $Q \in B_o\left( P, 1 \right) $. On note $P = \sum_{k=1}^{n} a_k X^k$ et $Q = \sum_{k=1}^{n} a_k X^k$. On dispose alors de $i in \left\{  1,\ldots,n \right\} $ tel que $|a_{i}|_p >1$ et $a_i + b_{i} \in \Zp$ (car $P-Q \in \Zp$). Donc on a nécessairement $|b_i|_p = |a_i|_p > 1$ et $Q \in Z_p[X]^c$.
Donc $\Zp[X]^c$ est ouvert et $\Zp[X]$ est fermé. 

 
On montre alors $i.$ et $ii.$ grâce à $\Pn = \chi^{-1}( \Zp[X]) $ ou $\chi$ est l'application qui a une matrice associe son polynôme caractéristique qui est continue car polynomiale en les coefficients de la matrice.



Le $iii$. est laissé en exercice au lecteur. 
\begin{remarque}
	L'ensemble $S_n^+\left( \Qp \right) $ n'est pas convexe en général.

	Par exemple pour $p = 5$, si on considère les matrices $M_1 = \begin{pmatrix} 5 + \frac{3}{5} & \frac{4}{5} \\ \frac{4}{5} & -\frac{3}{5} \end{pmatrix} $ et $M_2 = \begin{pmatrix} 25+\frac{7}{25} & \frac{24}{25} \\ \frac{24}{25}&-\frac{7}{25} \end{pmatrix} $, on a $\chi_{M_1} = X^2 -5 X -4  $ et $\chi_{M_2} = X^2 - 25 X -8$ donc en plongeant ces matrices dans $\mathbb{Q}_5$ il vient que $\chi_{M_1}, \chi_{M_2} \in \mathbf{Z}_5[X]$ i.e. $M_1,M_2 \in S_2^+\left( \Q_5 \right) $. Or $\chi_{M_1 + M_2}  = X^{2} - 150 X - \frac{3784}{5}$ qui une fois plongé dans $\mathbb{Q}_5$ n'est pas à coefficient dans $\mathbb{Z}_5$, donc $ S_2^+\left( \Q_5 \right) $ n'est pas convexe. Ce résultat se généralise pour tout $n$ en considérant les matrice par blocs $M'_i = \text{diag}\left( M_{i},0,\ldots,0 \right) $.

\end{remarque}


\subsection{Zoologie Spectraèdrique}
p
