\partie{2}{mardi 20 juin 11:00}{Spectraèdres $p$-adiques }
\newcommand\mat{matrice symétrique semie-définie positive } 
\newcommand\Mat{Matrice symétrique semie-définie positive }
\newcommand\mats{matrices symétriques semie-définies positives }
\newcommand\Mats{Matrices symétriques semie-définies positives }


\section{Spectraèdre \texorpdfstring{$p$}{p}-adiques } 
Ce paragraphe tend à fournir un définition de la notion de \mat sur les corps $p-$adiques pour en déduire une définition de spectraèdre qui serait pertinente sur un corps non-archimédien. 
\subsection{\Mats} 

\begin{definition}
	On appelle \mat toute matrice $M \in S_n\left( \Qp \right) $ dont toutes les valeurs propres sont de valuation positive ou nulle[lien du pragarâgre avec les extesnions de corps lolololololololol].

	On note $S_n^+\left( \mathbb{Q}_p \right)$ l'ensemble des \mats.
\end{definition}
\begin{propriete}
	\label{caracsnp}
	Caractérisation des \mats

	Une matrice est symétrique définie positive si et seulement si son polynôme caractéristique est à coefficient dans $\Zp$ .
	
\end{propriete}
	\textit{Preuve :} Voir le LIEEEEEEEEEEEEEEEEEEEEEEEEEEEEEEEEEEEEEEEEEEEEEN0

\begin{consequence}
	$\Pn \subset S_n^+\left( \mathbb{Q}_p \right)$ 
\end{consequence}

\textit{Preuve : }  Le polynôme caractéristique d'une matrice à coefficients dans $\Zp$ étant à coefficient dans $\Zp$ on obtient le résultat par la propriété \ref{caracsnp} .

\begin{remarque}
	En général l'inclusion réciproque est fausse. Ainsi pour $M = \begin{pmatrix} 5 + \frac{3}{5} & \frac{4}{5} \\ \frac{4}{5} & -\frac{3}{5} \end{pmatrix} $, on a $\chi_M = X^2  - 5 X - 4$. Or une fois $M$ plongé dans $\mathbb{Q}_5$ on a $\chi_M \in \mathbb{Z}_5[X]$ donc $M \in S_2^+\left( \mathbb{Q}_5 \right)$ or aucun des coefficients de $M$ n'est dans $S_2^+\left( \mathbb{Q}_5 \right)$

\end{remarque}

\note{vérifier si le fait qu'une matrice snp soit soit positive ou non est lié à la présence ou non de racine uniquement dans $\mathbb{Z}_p$ }.
