\section{Méta-informations}

Cette courte section dévoile quelques informations non-scientifique sur le déroulement du stage.

Pour les activités non liées aux stage de façon immédiate, j'ai pu lors de mon séjour participer à 2 repas organisés respectivement par le laboratoire et par le département dans lequel je me trouvais et j'ai également assisté à deux conférences données sur place, une par mon encadrant Tristan Vaccon et une seconde par un intervenant extérieur. De plus, je mangeais régulièrement avec les chercheurs.

Je disposais d'un bureau dans une salle que je partageais avec 3 autres stagiaires \footnote{tous fort sympathiques}. Il était convenu d'un rendez-vous hebdomadaire afin de discuter des mes avancées ou de mes doutes et interrogations. Cependant, en cas de questionnement je pouvais contacter directement mes encadrants qui étaient présents une majeure partie de la journée.

Le stage a approximativement suivie le déroulement suivant :

La première semaine a été passée à se documenter sur les corps $p$-adiques. La deuxième et la troisième ont été consacrées alternativement à essayer de trouver une définition pertinente de spectraèdre $p$-adique. La quatrième a principalement servie à commencer l'écriture du rapport ainsi qu'à se documenter sur les spectraèdres et l'optimisation convexe afin de dresser des similarités entre spectraèdres réels et $p$-adiques. La cinquième et la sixième ont été utilisées afin de continuer ce rapport démontrer que les couronnes $p$-adiques étaient des spectraèdres et concevoir l'algorithme présenté en \ref{sectionalgo}. L'ultime semaine semaine a permis de peaufiner quelques détails (preuves plus formelles, légères erreurs corrigées, etc.) et de continuer mon rapport de stage. Cependant cette dernière a été peu productive car je devais préparer mon départ et je suis tombé malade (de manière totalement indépendante).

Enfin, je tiens à remercier chaleureusement mes encadrants Tristan Vaccon et Simone Naldi pour, entre autres, leur disponibilité, leurs conseils et leur patience. Mes camarades de bureaux Léo et Lucile qui ont su égayer le bureau ainsi que Abdu Razik pour sa participation à \cite{rozik_borgir_2021}. 
