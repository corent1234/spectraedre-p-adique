\documentclass[a4paper,12pt]{article}

\newenvironment{proof}{\hbox{}\vspace{-0.5cm} {\bf Proof:}}{\hfill $\Box$ \\}

\newtheorem{theorem}{Theorem}
\newtheorem{lemma}{Lemma}
\newtheorem{proposition}{Proposition}
\newtheorem{corollary}{Corollary}
\newtheorem{algorithm}{Algorithm}
\newtheorem{conjecture}{Conjecture}
\newtheorem{condition}{Condition}
\newtheorem{definition}{Definition}
\newtheorem{assumption}{Assumption}
\newtheorem{remark}{Remark}
\newtheorem{problem}{Problem}
\newtheorem{example}{Example}
\usepackage{booktabs}

\textheight235mm
\textwidth165mm
\voffset-10mm
\hoffset-12.5mm
\parindent0cm
\parskip2mm

\usepackage{amsmath}
\usepackage{amssymb}
\usepackage{amsfonts}
\usepackage{mathrsfs} 
\usepackage{graphicx}
\usepackage{color}
\usepackage{ulem}
\usepackage{bm}

\usepackage{cleveref}

% updates

\newcommand{\R}{\mathbb{R}} % real numbers
\newcommand{\C}{\mathbb{C}} % complex numbers
\newcommand{\N}{\mathbb{N}} % integers
\newcommand{\sym}{\mathbb{S}} % matrices symétriques
\renewcommand{\span}[1]{{\text{span}(#1)}} % simone's comments
\newcommand{\calL}{\mathcal{L}} % simone's comments

\newcommand{\simone}[1]{{\color{blue} #1}} % simone's comments
\newcommand{\corentin}[1]{{\color{red} #1}} % corentin's comments
\newcommand{\tristan}[1]{{\color{olive} #1}} % tristan's comments

\usepackage{biblatex}
\addbibresource{../rapport/bibstage.bib}

\title{\bf Title}

\begin{document}

\author{Corentin Cornou$^{1}$, Simone Naldi$^2$ and Tristan Vaccon$^2$}

\footnotetext[1]{ENS Paris-Saclay, Université Paris-Saclay, France.}
\footnotetext[2]{Université de Limoges, CNRS, XLIM, Limoges, France.}

\date{Draft of \today}

\maketitle

\begin{abstract}
\end{abstract}


\section{Introduction}
\subsection{Motivations and main results}


il faudra mentionner (et peut être motiver l'approche p-adique) :
\begin{itemize}
\item les questions ouvertes de complexité concernant la programmation semidefinie (pas clair si NP $\cap$ co-NP
  dans le modele de Turing)
\item les questions ouvertes géométriques (Conjecture de Lax)
\end{itemize}

\subsection{Real spectrahedra}

Let $\sym_m(\R)$ be the vector space of $m \times m$ real symmetric matrices. We recall that
a matrix $M \in \sym_m(\R)$ is called positive semidefinite ($M \succeq 0$) whenever the quadratic
form $x \mapsto x^TMx$ is globally nonnegative. Recall that by the Spectral Theorem, $M \succeq 0$
if and only if the eigenvalues of $M$ are nonnegative, if and only if all the principal minors of
$M$ are nonnegative. The set of positive semidefinite matrices, denoted $\sym_m^+(\R) \subset \sym_m(\R)$,
is a closed convex cone with non-empty interior in $\sym_m(\R)$.

Let $A_0,A_1,\ldots,A_n \in \sym_m(\R)$, and let $\calL = A_0+\span{A_1,\ldots,A_n}$ be the affine space
containing $A_0$ and with direction the vector space generated by $A_1,\ldots,A_n$ (which we assume linearly
independent). The intersection
$\calL \cap \sym_m^+(\R) = \left\{A(x) := A_0+\sum_i x_i A_i \in \sym_m(\R) : A(x) \succeq 0\right\}$, or
equivalently, its pre-image $S = \{x \in \R^n : A(x) \succeq 0\}$ under the map $x \mapsto A(x)$, is called
a {\it real spectrahedron}. As affine sections of $\sym_m^+(\R)$, real spectrahedra are closed convex cones,
but might have empty interior in $\calL$ (resp. in $\R^n$). Moreover, they are basic semialgebraic sets
defined by sign conditions on the principal minors of the defining matrix as well as sign conditions on the
coefficients of its characteristic polynomial (cf. \Cref{ell3}).

\begin{example}\label{ell3}
  The $3$-elliptope is the three-dimensional spectrahedron given by
  $$
  \mathcal{E}_3 :=
  \left\{
  x
%  =
%  \begin{bmatrix}
%    x_1 \\
%    x_2 \\
%    x_3
%  \end{bmatrix}
  \in \R^3 :
  \begin{bmatrix}
    1 & x_1 & x_2 \\
    x_1 & 1 & x_3 \\
    x_2 & x_3 & 1
  \end{bmatrix}
  \succeq 0
  \right\}
  =
  \left\{
  x
  \in \R^3 :
  \begin{array}{r}
    3-x_1^2-x_2^2-x_3^2 \geq 0 \\
    1+2x_1x_2x_3-x_1^2-x_2^2-x_3^2 \geq 0
  \end{array}
  \right\}.
  $$
  The polynomials on the right are the (non-constant) coefficients of the univariate polynomial
  $p(t) = \det(A(x)+t I)$,
  whose roots are the opposites of the eigenvalues of the defining matrix. Equivalently $\mathcal{E}_3$
  is defined by positivity of the principal minors of the defining matrix:
  $$
  1-x_1^2 \geq 0, 1-x_2^2 \geq 0, 1-x_3^2 \geq 0 \,\,\, \text{ and } \,\,\, 1+2x_1x_2x_3-x_1^2-x_2^2-x_3^2 \geq 0
  $$
\end{example}

\section{Notation}

\section{Polyhedra and linear programming}

\subsection{Polyhedra}
\subsection{Linear programmming}

\section{Spectrahedra}

\subsection{Definition}
\subsection{Annuli (couronnes)}
\subsection{Polyhedrality}
Cf. \cite{bhardwaj2015deciding}

\section{Linear matrix inequalities}


\printbibliography
\end{document}
