\documentclass[a4paper,12pt]{article}

\newenvironment{proof}{\hbox{}\vspace{-0.5cm} {\bf Proof:}}{\hfill $\Box$ \\}

\newtheorem{theorem}{Theorem}
\newtheorem{lemma}{Lemma}
\newtheorem{proposition}{Proposition}
\newtheorem{corollary}{Corollary}
\newtheorem{algorithm}{Algorithm}
\newtheorem{conjecture}{Conjecture}
\newtheorem{condition}{Condition}
\newtheorem{definition}{Definition}
\newtheorem{assumption}{Assumption}
\newtheorem{remark}{Remark}
\newtheorem{problem}{Problem}
\newtheorem{example}{Example}
\usepackage{booktabs}

\textheight235mm
\textwidth165mm
\voffset-10mm
\hoffset-12.5mm
\parindent0cm
\parskip2mm

\usepackage{amsmath}
\usepackage{amssymb}
\usepackage{amsfonts}
\usepackage{mathrsfs} 
\usepackage{graphicx}
\usepackage{color}
\usepackage{ulem}
\usepackage{bm}

\usepackage{cleveref}

% updates

\newcommand{\R}{\mathbb{R}} % real numbers
\newcommand{\C}{\mathbb{C}} % complex numbers
\newcommand{\N}{\mathbb{N}} % integers
\newcommand{\sym}{\mathbb{S}} % matrices symétriques

\newcommand{\simone}[1]{{\color{blue} #1}} % simone's comments
\newcommand{\corentin}[1]{{\color{red} #1}} % corentin's comments
\newcommand{\tristan}[1]{{\color{olive} #1}} % tristan's comments

\usepackage{biblatex}
\addbibresource{../rapport/bibstage.bib}

\title{\bf Title}

\begin{document}

\author{Corentin Cornou$^{1}$, Simone Naldi$^2$ and Tristan Vaccon$^2$}

\footnotetext[1]{ENS Paris-Saclay, Université Paris-Saclay, France.}
\footnotetext[2]{Université de Limoges, CNRS, XLIM, Limoges, France.}

\date{Draft of \today}

\maketitle

\begin{abstract}
\end{abstract}


\section{Introduction}
\subsection{Motivations and main results}


il faudra mentionner (et peut être motiver l'approche p-adique) :
\begin{itemize}
\item les questions ouvertes de complexité concernant la programmation semidefinie (pas clair si NP $\cap$ co-NP
  dans le modele de Turing)
\item les questions ouvertes géométriques (Conjecture de Lax)
\end{itemize}

\subsection{Real spectrahedra}

Let $\sym_m(\R)$ be the vector space of $m \times m$ real symmetric matrices. We recall that
a matrix $M \in \sym_m(\R)$ is called positive semidefinite ($M \succeq 0$) whenever the quadratic
form $x \mapsto x^TMx$ is globally nonnegative. Recall that by the Spectral Theorem, $M \succeq 0$
if and only if the eigenvalues of $M$ are nonnegative, if and only if all the principal minors of
$M$ are nonnegative. The set of positive semidefinite matrices, denoted $\sym_m^+(\R) \subset \sym_m(\R)$,
is a closed convex cone with non-empty interior in $\sym_m(\R)$.

A {\it (real) spectrahedron} is an affine slice of the 

\section{Notation}

\section{Polyhedra and linear programming}

\subsection{Polyhedra}
\subsection{Linear programmming}

\section{Spectrahedra}

\subsection{Definition}
\subsection{Annuli (couronnes)}
\subsection{Polyhedrality}
Cf. \cite{bhardwaj2015deciding}

\section{Linear matrix inequalities}


\printbibliography
\end{document}
