\documentclass[a4paper,12pt]{article}


\newenvironment{proof}{\hbox{}\vspace{-0.5cm} {\bf Proof:}}{\hfill $\Box$ \\}

\newtheorem{theorem}{Theorem}
\newtheorem{lemma}{Lemma}
\newtheorem{proposition}{Proposition}
\newtheorem{corollary}{corollary}
\newtheorem{algorithm}{Algorithm}
\newtheorem{conjecture}{Conjecture}
\newtheorem{condition}{Condition}
\newtheorem{definition}{Definition}
\newtheorem{assumption}{Assumption}
\newtheorem{remark}{Remark}
\newtheorem{problem}{Problem}
\newtheorem{example}{Example}
\usepackage{booktabs}

\textheight235mm
\textwidth165mm
\voffset-10mm
\hoffset-12.5mm
\parindent0cm
\parskip2mm

\usepackage{amsmath}
\usepackage{amssymb}
\usepackage{amsfonts}
\usepackage{mathrsfs} 
\usepackage{graphicx}
\usepackage{color}
\usepackage{ulem}
\usepackage{bm}

\usepackage{cleveref}

\usepackage{stmaryrd} %for double brackets


% updates

\newcommand{\R}{\mathbb{R}} % real numbers
\newcommand{\C}{\mathbb{C}} % complex numbers
\newcommand{\N}{\mathbb{N}} % integers
\newcommand{\sym}{\mathbb{S}} % matrices symétriques
\renewcommand{\span}[1]{{\text{span}(#1)}} % simone's comments
\newcommand{\calL}{\mathcal{L}} % simone's comments

\newcommand{\simone}[1]{{\color{blue} #1}} % simone's comments
\newcommand{\corentin}[1]{{\color{red} #1}} % corentin's comments
\newcommand{\tristan}[1]{{\color{olive} #1}} % tristan's comments

%p-adic commands
\DeclareMathOperator{\val}{val}
\def\QQ{\ensuremath{\mathbb{Q}}}
\def\ZZ{\ensuremath{\mathbb{Z}}}
\newcommand{\OK}{\mathcal{O}_K}


\usepackage{biblatex}
\addbibresource{../rapport/bibstage.bib}

\title{\bf Title}

\begin{document}

\author{Corentin Cornou$^{1}$, Simone Naldi$^2$ and Tristan Vaccon$^2$}

\footnotetext[1]{ENS Paris-Saclay, Université Paris-Saclay, France.}
\footnotetext[2]{Université de Limoges, CNRS, XLIM, Limoges, France.}

\date{Draft of \today}

\maketitle

\begin{abstract}
\end{abstract}


\tableofcontents


\section{Introduction}

\subsection{Context and motivations}

Il faudra entre autre mentionner (et peut être motiver l'approche p-adique) :
\begin{itemize}
\item les questions ouvertes de complexité concernant la programmation semidefinie (pas clair si NP $\cap$ co-NP
  dans le modele de Turing)
\item les questions ouvertes géométriques (Conjecture de Lax)
\end{itemize}


\subsection{Main results}



\section{Preliminaries}

\subsection{Notation}

Throughout the paper, $K$ refers to a complete,
discrete valuation field, $\val : K \twoheadrightarrow \ZZ \cup \{+\infty\}$ to its valuation,
$\OK$ its ring of integers and $\pi$ a uniformizer.
For $k \in \N$, we write $O(\pi^k)$ for $\pi^k \OK.$
A typical example of $K$ as above is the field of $p$-adic numbers 
$\QQ_p$ (equipped with the $p$-adic valuation). For this example, we 
have $\OK = \ZZ_p$.
Another example is the field of Laurent series
$K=\QQ ((T))$ equiped with $T$-adic valuation.
In this case,  $\OK = \QQ \llbracket T \rrbracket$.
We refer to \cite{Serre:1979} for a
general introduction to such fields
and to Caruso's course on $p$-adic computations \cite{caruso_computations_2017}
for an introduction to effective computations
over them

\subsection{Real spectrahedra}

Let $\sym_m(\R)$ be the vector space of $m \times m$ real symmetric matrices. A matrix $M \in \sym_m(\R)$
is called {\it positive semidefinite} ($M \succeq 0$) whenever the quadratic
form $x \mapsto x^TMx$ is nonnegative for all $x\in \R^n$. By the Spectral Theorem, $M \succeq 0$
if and only if the eigenvalues of $M$ are nonnegative, and this is equivalent to all the principal minors
of $M$ being nonnegative. The set of positive semidefinite matrices, denoted $\sym_m^+(\R) \subset \sym_m(\R)$,
is a closed convex cone with non-empty interior in $\sym_m(\R)$.

Let $A_0,A_1,\ldots,A_n \in \sym_m(\R)$, and let $\calL = A_0+\span{A_1,\ldots,A_n}$ be the affine space
containing $A_0$ and with direction the vector space spanned by $A_1,\ldots,A_n$ (which we assume linearly
independent). The intersection
$\calL \cap \sym_m^+(\R) = \left\{A(x) := A_0+\sum_i x_i A_i \in \sym_m(\R) : A(x) \succeq 0\right\}$, or
equivalently, its pre-image $S = \{x \in \R^n : A(x) \succeq 0\}$ under the map $x \mapsto A(x)$, is called
a {\it real spectrahedron}. As affine sections of $\sym_m^+(\R)$, real spectrahedra are closed convex sets,
but might have empty interior in $\calL$ (resp. in $\R^n$). Moreover, they are basic semialgebraic sets,
defined by sign conditions on the principal minors of the defining matrix as well as sign conditions on the
coefficients of its characteristic polynomial (cf. \Cref{ell3}).

\begin{example}\label{ell3}
  The $3$-elliptope is the three-dimensional spectrahedron given by
  $$
  \mathcal{E}_3 :=
  \left\{
  x
  \in \R^3 :
  \begin{bmatrix}
    1 & x_1 & x_2 \\
    x_1 & 1 & x_3 \\
    x_2 & x_3 & 1
  \end{bmatrix}
  \succeq 0
  \right\}
  =
  \left\{
  x
  \in \R^3 :
  \begin{array}{r}
    3-x_1^2-x_2^2-x_3^2 \geq 0 \\
    1+2x_1x_2x_3-x_1^2-x_2^2-x_3^2 \geq 0
  \end{array}
  \right\}.
  $$
  The polynomials on the right are the (non-constant) coefficients of the univariate polynomial
  $p(t) = \det(A(x)+t I)$,
  whose roots are the opposites of the eigenvalues of the defining matrix. Equivalently $\mathcal{E}_3$
  is defined by positivity of the principal minors of the defining matrix:
  $$
  1-x_1^2 \geq 0, \,\,\, 1-x_2^2 \geq 0, \,\,\, 1-x_3^2 \geq 0 \,\,\, \text{ and } \,\,\, 1+2x_1x_2x_3-x_1^2-x_2^2-x_3^2 \geq 0
  $$
\end{example}

The optimization problem of miminizing a linear function over a real spectrahedron is called {\it semidefinite programming} (SDP). Given $C,A_0,A_1,\ldots,A_n \in \sym_m(\R)$, the corresponding semidefinite program is defined as
\begin{equation}
  \label{SDP}
\begin{array}{rcll}
  p^* & := & \inf_{x \in \R^n} & \left\langle c, x \right\rangle \\
  &    & \text{s.t.}         & A_0+\sum_{i=1}^n x_i A_i \succeq 0.
\end{array}
\end{equation}
Since real polyhedra are special cases of real spectrahedra, SDP is a generalization of linear programming (LP).
In a fixed-precision model of computation, SDP is solvable in polynomial time in the input size (matrix size,
number of variables, bit size of the coefficients), in $\log(\frac{1}{\varepsilon})$ (where $\varepsilon$ is the
precision) and $\log(\frac{1}{R})$, where $R$ is an {\it a priori} upper bound on the norm of the solution,
cf. \cite[Sec.1.9]{deKlerk}. In exact arithmetic, the complexity status of SDP is essentially open but complexity
bounds are known based either on general quantifier elimination \cite{ramana1997exact,porkolab1997complexity}
or on the determinantal structure of the constraints \cite{henrion2016exact}.

A more general class of convex semialgebraic sets are obtained by linear projections of real spectrahedra
called {\it semidefinite representable sets} (or spectrahedral shadows): these are sets of the form
$$
P = \left\{x = \left[\begin{smallmatrix} x_1 \\ \vdots \\ x_n \end{smallmatrix}\right] \in \R^n : \exists\,y\in\R^m, \, A_0 + \sum_{i=1}^n x_i A_i + \sum_{j=1}^m y_j B_j \succeq 0\right\}
$$

\begin{example}
\label{fermat_quartic}
The basic closed semialgebraic set $P = \{(x_1,x_2) \in \R^2 : 1-x_1^4-x_2^4 \geq 0\}$ is not a spectrahedron
but it is semidefinite representable:
$$
P = \left\{x = \begin{bmatrix} x_1 \\ x_2 \end{bmatrix} \in \R^2 :
\exists\, (y_1,y_2) \in \R^2, \,
\begin{bmatrix}
  1+y_1 & y_2 \\
  y_2 & 1-y_1
\end{bmatrix}
\bigoplus
\begin{bmatrix}
  1 & x_1 \\
  x_1 & y_1
\end{bmatrix}
\bigoplus
\begin{bmatrix}
  1 & x_2 \\
  x_2 & y_2
\end{bmatrix}
\succeq 0
\right\}
$$
\end{example}


\section{Polyhedra and linear programming}
\subsection{Polyhedra}

\begin{definition}
	We call {\it polyhedra} subsets of $K^s$ defined by linear inequalities $\val\left( \ell_1(x)\right)\ge 0$,\ldots, $ \val\left( \ell_n\left( x \right) \right)\ge 0$ for $ \ell_1,\ldots, \ell_n $ fixed linear forms and for all $x \in K ^s$.
	
	
\end{definition}

\begin{example}
	The closed unit ball of $K^s$ for the infinity norm [not sure if it is really used ] is convex polyhedron.It is the set of $\left( x_1,\ldots,x_s \right) \in  ^s$ so that $$
	\begin{pmatrix} x_1 &  & & \\
		& x_2 &0 &  \\
		&  0& \ddots & \\
		&  & & x_s/
	\end{pmatrix} \ge 0.$$ which is in fact $\OK^s$. 
	
\end{example}

\subsection{linear programming}
	We call { \it $p$-adic linear programming} the problem :
\begin{equation}
	\tag{PLp}
	\begin{matrix}
		\text{Minimize } \val\left(\left<c,x \right>\right) \text{ subject to }\\
		Ax + b \ge 0
	\end{matrix}
	\label{eqn:Proglinp}
\end{equation}

with $x$ a vector of $K ^n$, and $c$ a fixed vector of size $n$ that represents the cost, $A$ a fixed $m \times n $ matrix and $b$ fixed a vector of size $m$.

The method used in this article consists in reducing the Smith normal form of the matrix $A$.

\begin{definition} Smith normal form
	
	We call {\it Smith normal form} of a matrix $M \in \mathcal{M}_{m,n}\left(K \right) $ of rank $r$ the only matrix $S$ of the form $$S =  
	\begin{pmatrix} p^{a_1} & \\
		& \ddots \\
		& & p^{a_r}\\
		& & & 0\\
		& & & & \ddots \end{pmatrix} $$
	so that $(a_1)_{1\le i\le r}$ is a increasing sequence of integers and $M$ can be written $M =  Q^{-1} S P$ for $P \in \mathcal{G}L_n\left( \ZZ_p \right) $, $Q \in \mathcal{G}L_m\left( \ZZ_p \right) $.
\end{definition}

\begin{remark}
	\begin{enumerate}%[label=\roman*.]
		\item The coefficient of the Smith normal form are unique and called \textit{invariant factors}.
		\item The valuation of the first coefficient of the Smith normal form of a matrix $M \in \mathcal{M}_n \left( \QQ_p \right) $ is the minimum of the valuations of the matrix $M$'s terms.
		\item In particular, the the normal Smith form of matrix of $\mathcal{M}_n \left( \OK \right) $ have coefficients in $\ZZ_p$ .
		\item The $r = \text{rank} M$ first diagonals coefficients of $S$ are exactly its non-zero coefficient.
	\end{enumerate}
	
\end{remark}

In order to be able to use the Smith normal form to resolve \ref{eqn:Proglinp} we need the following lemma :

\begin{lemma}
	Forall $p$-adic numbers $z \in \QQ_p ^n$ and matrix $M \in \mathcal{M}_{m,n}\left( \QQ_p  \right)  $ if $z\ge 0$ and $M\ge 0$ then $Mz\ge 0$.  
\end{lemma}
\begin{proof}
	$\ZZ_p$ is a ring.
\end{proof}


If we put the matrix $A$ of \ref{eqn:Proglinp} in its normal Smith form, it comes that resolving \ref{eqn:Proglinp} is equivalent to resolve :    


\begin{equation}
	\tag{PLp'}
	\begin{matrix}
		\text{Minimize } \val\left(\left<c',x \right>\right) \text{ subject to }\\
		Sy + b' \ge 0
	\end{matrix}
	\label{eqn:Proglinp2}
\end{equation}
where $b' = Qb$, $c' = P^Tc$, $S$ is the Smith normal form of $A$ and $A = Q^{-1} S P$ for some $P \in \mathcal{G}L_m\left( \ZZ_p \right)$ and $ Q \in \mathcal{G}L_n\left( \ZZ_p \right)$.

\begin{remark}
	
	We have then the following results
	\begin{enumerate}
		\item $x^*$ is a feasible solution of \ref{eqn:Proglinp} if and only if $y^* = P x^*$ is a feasible solution of \ref{eqn:Proglinp2}
		\item \ref{eqn:Proglinp2} have feasible solutions if and only if the $m-r$ coefficients of $b'$ are non-zero, with $r$ the rank of $S$.
		\item If at least one of the $n-r$ last coefficients of $c'$ is non-zero \ref{eqn:Proglinp2} is unbounded and does not have a soution.
	\end{enumerate}
	
\end{remark}

\begin{proposition}
	\begin{itemize}~
		
		\item[$\circ$] If not all the $m-r$ coefficients of $b'$ are zeros then there is no $y \in K ^n$ so that $A'y+b' \ge  0$.
		\item[$\circ$] If not all the $n-r$ coefficients of $c'$ are zero then the problem is unbounded and does not have solution. \end{itemize} 
\end{proposition}
If one considers only the itération of the problem which have des solution one can reduce the problem by only taking acount the $r$ first coefficients of $y, b', c'$ and the submatrix of $S$ made with only $r$ first lines et columns of and which coefficients are exactly the nonzero invariants factors. The set $Adm$ of the feasible solutions can then be written as the set of vectors $y \in \mathbb{Q}_{ p } ^n$ that verifies $\forall 1 \le i\le r \ s_i y_i + b'_i \in \OK$ i.e. that verifies :

\begin{equation}
	\forall 1 \le i\le r  \ y_i \in -\frac{b’_i}{s_i} + \frac{1}{s_i} \OK
\end{equation}


Resolving \ref{eqn:Proglinp2} consists in minimizing $\val\left(\left<c',y \right>\right)$ over $Adm$. The image of $Adm$ by $y \mapsto \left<c',y \right>$ is $\sum_{i=1}^r -c'_i.\frac{b'_i}{s_i} + \sum_{i=1}^r\left( \frac{c'_i}{s_{i}} \OK \right)$ which can be rewritten in 
$$\left<c',Adm \right> = \lambda + p^{v} \OK$$
with $\lambda = \sum\limits_{i=1}^r -c'_i.\frac{b'_i}{s_i}$ et $v = \min\limits_{1\le i\le r} \val \frac{c'_{i}}{s_{i}} $.

The result can be split depending on the valuation of $\lambda$.
\begin{itemize}
	\item If $\val( \lambda) < v$ the minimum of $y\mapsto \val\left(\left<c',y \right>\right)$ over $Adm$ is reached in any point of $Adm$ and is equal to $\val\left( \lambda\right)$.
	\item If $\val\left( \lambda \right) \ge v$ then $\lambda \in p^{v} \OK$ and the minimum is $v $ and is reached in any points $y$ of $Adm$ that verifies 
	$$\val\sum \limits_{\begin{array}{c} 1\le i\le r\\ \val\left(c'_{i}/{s_{i}} \right) = v  \end{array}   } y_{i} = 0.$$ 
	
\end{itemize}
\begin{remark}
	To maximize the valuation of a linear map over a convex $p$-adic polyhedron instead of minimizing it (which is equuivalent to minimize the absolute value) the same reasoning can be aplied. The only differnece being that if $\val\left( \lambda\right) < v$ the problem is unbounded and then does not have any solution. 
\end{remark}




\section{Spectrahedra}

\subsection{Definition}
\newcommand\mat{postive semidefinite matrix } 
\newcommand\Mat{Positive semidefinite matrix }
\newcommand\mats{positive semidefinite matrices }
\newcommand\Mats{positive semidefinite matrices }
\begin{definition}
	We call {\it \mat} any matrix $M \in \mathcal{M}_n\left( K \right) $ whose eigenvalues have nonegative valuation (in $\overline{K}$) .
	
	We will denote as $ \mathcal{M}_n^+\left( K \right)$ the set of \mats and we will write $M \succeq 0$ if $M \in \mathcal{M}_n^+\left( K\right) $.
\end{definition}
\begin{theorem}
	\label{caracsdp}
	Characterization of the \mats
	
	A matrix is positive semidefinite if and only if its characteristic polynomial have coefficients in $\OK$ .
	
\end{theorem}


\begin{corollary}
	$\mathcal{M}_n(\OK) \subset \mathcal{M}_n^+\left( K \right)$ 
\end{corollary}

\begin{proof}
$\OK$ is a ring therefore the characteristic polynomial of matrix whose coefficients are $p$-adic integers have coefficients ins $Pp$. We conclude by \ref{caracsdp} .
\end{proof}



\begin{remark}
	In general, the reverse inclusion is false. For instance the characteristic polynomial of $M = \begin{pmatrix} 5 + \frac{3}{5} & \frac{4}{5} \\ \frac{4}{5} & -\frac{3}{5} \end{pmatrix} $ is $\chi_M = X^2  - 5 X - 4$. However when seen as an element of $\mathbb{Q}_5$, $\chi_M \in \mathbb{Z}_5[X]$ then $M \in \mathcal{M}_2^+\left( \mathbb{Q}_5 \right)$ but no coefficient of $M$ is in $\mathbb{Z}_5 $
\end{remark}

\begin{proposition}
	The set $\mathcal{M}_n^+\left( K \right)$ is :
	\begin{enumerate}%[label = \textit{\roman*}.]
		\item open
		\item closed
		%\item a convex cone
	\end{enumerate}
\end{proposition}

\textit{Preuve :}  
Firstly, $\OK[X]$ is closed and open in $\mathbb{Q}_{ p }[X] $ equiped with the infinitiy norm $\|.\|_\infty : P = \sum_{k=1}^{n} a_k X^k \to \sup |a_k|_p$. $\OK$ is indeed the open unit ball for $\|\cdot \|_\infty$ which defines a discrete distance for which open balls are also closed.
Furthermore, $\mathcal{M}^+_{n}\left(K \right)  = \chi^{-1}( \OK[X]) $ where $\chi$ is the function that maps its characteristic polynomial to a matrix which is continuous.

%Le $iii$. est laissé en exercice au lecteur. 
% Je ne parle plus de convexité dans le rapport mais on ne sait jamais
%\begin{remarque}
	L'ensemble $S_n^+\left( \Qp \right) $ n'est pas convexe en général.

	Par exemple pour $p = 5$, si on considère les matrices $M_1 = \begin{pmatrix} 5 + \frac{3}{5} & \frac{4}{5} \\ \frac{4}{5} & -\frac{3}{5} \end{pmatrix} $ et $M_2 = \begin{pmatrix} 25+\frac{7}{25} & \frac{24}{25} \\ \frac{24}{25}&-\frac{7}{25} \end{pmatrix} $, on a $\chi_{M_1} = X^2 -5 X -4  $ et $\chi_{M_2} = X^2 - 25 X -8$ donc en plongeant ces matrices dans $\mathbb{Q}_5$ il vient que $\chi_{M_1}, \chi_{M_2} \in \mathbf{Z}_5[X]$ i.e. $M_1,M_2 \in S_2^+\left( \Q_5 \right) $. Or $\chi_{M_1 + M_2}  = X^{2} - 150 X - \frac{3784}{5}$ qui une fois plongé dans $\mathbb{Q}_5$ n'est pas à coefficient dans $\mathbb{Z}_5$, donc $ S_2^+\left( \Q_5 \right) $ n'est pas convexe. Ce résultat se généralise pour tout $n$ en considérant les matrice par blocs $M'_i = \text{diag}\left( M_{i},0,\ldots,0 \right) $.

\end{remarque}



\begin{definition}
	We call {\it spectrahedron} the intersection of $\mathcal{M}_n^+\left( K \right) $ with an affine plan $\mathcal{L}$ of $\mathcal{M}_n\left( K \right) $.
\end{definition}

\begin{remark}
	As in the real case, the spectrahedron associated with the affine plan $\mathcal{L}$ will often be identified with its preimage in $K^s$ (i.e. the set of vectors $x \in K ^s$ so that the linear matrix $A(x) = A_0 + x_1A_1 + \ldots + x_sA_s$ is positive semidefinite with $A_0,A_1,\ldots,A_s$ spanning $\mathcal{L}$). 
\end{remark}

The latter allows gives us the following property :
\begin{proposition}
	A polyhedron is a spectrahedron.
\end{proposition}
\begin{proof}
	Same as in the real case (maybe modify it or add the link to real proof).
\end{proof}


\subsection{Annuli (couronnes)}
\begin{definition}
	Annulus
	
	We call {\it annulus} a set $A$ defined as $C = \{x \in K  | a\le \val\left(x\right) \le b\} $ for $a < b$ two nonnegative real numbers.
\end{definition}


\begin{theorem}
	Annuli are spectrahedral shadows.
\end{theorem} 

\textit{Preuve :}
Let $A$ be the annuli defined by $a<b \in R^*_+$.

We consider for all $x,y \in K $ the matrix  $$M(x,y) :=
\begin{pmatrix} 
	p^ax & 0 & 0 & 0 \\
	0 & p^{-b}y & 0 & 0 \\
	0 & 0 & p^{-1} & p^{-1}x \\
	0 & 0 & p^{-1}y & - p^{-1}  \\
\end{pmatrix} $$ and the associated spectrahedron $\mathcal{S}= \{(x,y) \in K  : M(x,y) \succeq 0\} $

Let us show that $x \in C$ if and only if  $\exists y \in K \left( x,y \right) \in \mathcal{S}$.

Let $x,y$ be two $p$-adic numbers.
$M(x,y)$ can be decomposed in three blocks : $p^ax$, $p^{-b}y $ and $M'(x,y) = \begin{pmatrix} p^{-1} & p^{-1}x \\ p^{-1} y & p^{-1}\end{pmatrix} $. Using \ref{caracsdp} we know that $M(x,y)$ is positive semidefinfite if and only if
$
\begin{cases}
	p^ax \ge 0 \\
	p^{-b}y \ge 0\\
	\text{Tr}M'\left( x,y \right) = 0 \ge 0\\
	\text{det} M'(x,y ) = p^{-2}\left( xy-1 \right)  \ge 0 
\end{cases}
$ 
that is if and only if 
\begin{equation}
	\label{eq:caracsdp} 
	\begin{cases} 
		\val\left(x\right)\ge a\\
		\val\left(y\right)\ge -b\\
		p^{-2} \left( xy-1 \right) \ge 0
	\end{cases}
\end{equation}




Therefore $x \in C$ if and only if there exists $y \in K $ so that $(x, y)$ verifies \ref{eq:caracsdp}. Si $x, y$ verifies indeed \ref{caracsdp} then $\val\left(x\right)\ge a$, $\val\left(y\right)\ge -b$ and $p^{-2} \left( xy-1 \right) \ge 0$ implies that $\val\left(x\right)+\val\left(y\right)=\val\left(xy\right) = \val\left(-1\right) =0$ and therefore that $\val\left(x\right)=-\val\left(y\right)\le -(-b)  =b$, so $x \in $. Reciprocally if $x \in C$ then $(x,x^{-1}) $ verifies \ref{eq:caracsdp}  .


\subsection{Polyhedrality}
Cf. \cite{bhardwaj2015deciding}

\section{Linear matrix inequalities}


\printbibliography
\end{document}
