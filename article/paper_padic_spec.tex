\documentclass[a4paper,12pt]{article}


\newenvironment{proof}{\hbox{}\vspace{-0.5cm} {\bf Proof:}}{\hfill $\Box$ \\}

\newtheorem{theorem}{Theorem}
\newtheorem{lemma}{Lemma}
\newtheorem{proposition}{Proposition}
\newtheorem{corollary}{Corollary}
\newtheorem{algorithm}{Algorithm}
\newtheorem{conjecture}{Conjecture}
\newtheorem{condition}{Condition}
\newtheorem{definition}{Definition}
\newtheorem{assumption}{Assumption}
\newtheorem{remark}{Remark}
\newtheorem{problem}{Problem}
\newtheorem{example}{Example}
\newtheorem{notation}{Notation}
\usepackage{booktabs}

\textheight235mm
\textwidth165mm
\voffset-10mm
\hoffset-12.5mm
\parindent0cm
\parskip2mm

\usepackage{amsmath}
\usepackage{amssymb}
\usepackage{amsfonts}
\usepackage{mathrsfs} 
\usepackage{graphicx}
\usepackage{color}
\usepackage{ulem}
\usepackage{bm}

\usepackage{cleveref}

\usepackage{stmaryrd} %for double brackets

%\usepackage{mathptmx}

% updates

\newcommand{\R}{\mathbb{R}} % real numbers
\newcommand{\C}{\mathbb{C}} % complex numbers
\newcommand{\N}{\mathbb{N}} % integers
\newcommand{\allmat}{\mathbb{M}} % matrices
\newcommand{\sym}{\mathbb{S}} % matrices symétriques
\renewcommand{\span}[1]{{\text{span}(#1)}} % simone's comments
\newcommand{\calL}{\mathcal{L}} % simone's comments

\newcommand{\simone}[1]{{\color{blue} #1}} % simone's comments
\newcommand{\corentin}[1]{{\color{red} #1}} % corentin's comments
\newcommand{\tristan}[1]{{\color{olive} #1}} % tristan's comments

%p-adic commands
\DeclareMathOperator{\val}{val}
\def\QQ{\ensuremath{\mathbb{Q}}}
\def\ZZ{\ensuremath{\mathbb{Z}}}
\newcommand{\OK}{\mathcal{O}_K}

\newcommand{\GL}{\mathrm{GL}}

\newcommand{\exend}{\hfill $\blacksquare$}


\usepackage{biblatex}
\addbibresource{../rapport/bibstage.bib}

\title{\bf Title}

\begin{document}

\author{Corentin Cornou$^{1}$, Simone Naldi$^2$ and Tristan Vaccon$^2$}

\footnotetext[1]{ENS Paris-Saclay, Université Paris-Saclay, France.}
\footnotetext[2]{Université de Limoges, CNRS, XLIM, Limoges, France.}

\date{Draft of \today}

\maketitle

\begin{abstract}
\end{abstract}


\tableofcontents


\section{Introduction}

\subsection{Context and motivations}

Il faudra entre autre mentionner (et peut être motiver l'approche p-adique) :
\begin{itemize}
\item les questions ouvertes de complexité concernant la programmation semidefinie (pas clair si NP $\cap$ co-NP
  dans le modele de Turing)
\item les questions ouvertes géométriques (Conjecture de Lax)
\end{itemize}


\subsection{Main results}



\section{Preliminaries}

\subsection{Notation}

Throughout the paper, $K$ refers to a complete,
discrete valuation field, $\val : K \twoheadrightarrow \ZZ \cup \{+\infty\}$ to its valuation,
$\OK$ its ring of integers and $\pi$ a uniformizer.
For $k \in \N$, we write $O(\pi^k)$ for $\pi^k \OK$.
A typical example of $K$ as above is the field of $p$-adic numbers 
$\QQ_p$ (equipped with the $p$-adic valuation). For this example, we 
have $\OK = \ZZ_p$.
Another example is the field of Laurent series
$K=\QQ(\!(t)\!)$ equiped with $t$-adic valuation.
In this case, $\OK = \QQ \llbracket t \rrbracket$.
We refer to \cite{Serre:1979} for a
general introduction to such fields
and to X.~Caruso's course on $p$-adic computations \cite{caruso_computations_2017}
for an introduction to effective computations over $p$-adic numbers.

The $K$-vector space of polynomials of degree $\leq d$ on $x=(x_1,\ldots,x_n)$
with coefficients in $K$ is denoted by $K[x]_{d}$. The vector space of matrices of
size $p \times q$ with entries in $K$ is denoted by $\allmat_{p,q}(K)$. For the sake
of ease of writing, block-diagonal matrices with blocks $B_1,\ldots,B_d$ are denoted by
$B_1 \oplus \cdots \oplus B_d$.

\subsection{Real spectrahedra}

Let $\sym_d(\R) \subset \allmat_{d,d}(\R)$ be the vector space of $d \times d$ real symmetric
matrices. A matrix $M \in \sym_d(\R)$
is called {\it positive semidefinite} ($M \succeq 0$) whenever the quadratic
form $x \mapsto x^TMx$ is nonnegative for all $x\in \R^n$. By the Spectral Theorem, $M \succeq 0$
if and only if the eigenvalues of $M$ are nonnegative, and this is equivalent to all the principal minors
of $M$ being nonnegative. The set of positive semidefinite matrices, denoted $\sym_d^+(\R) \subset \sym_d(\R)$,
is a closed convex cone with non-empty interior in $\sym_d(\R)$.

Let $A_0,A_1,\ldots,A_n \in \sym_d(\R)$, and let $\calL = A_0+\span{A_1,\ldots,A_n}$ be the affine space
containing $A_0$ and with direction the vector space spanned by $A_1,\ldots,A_n$ (which we assume linearly
independent). The intersection $\calL \cap \sym_d^+(\R) = \left\{A(x) := A_0+\sum_i x_i A_i \in \sym_d(\R) :
A(x) \succeq 0\right\}$, or
equivalently, its pre-image $S = \{x \in \R^n : A(x) \succeq 0\}$ under the map $x \mapsto A(x)$, is called
a {\it real spectrahedron}. As affine sections of $\sym_d^+(\R)$, real spectrahedra are closed convex sets,
but might have empty interior in $\calL$ (resp. in $\R^n$). Moreover, they are basic semialgebraic sets,
defined by sign conditions on the principal minors of the defining matrix as well as sign conditions on the
coefficients of its characteristic polynomial (cf. \Cref{ell3}).

\begin{example}\label{ell3}
  Let $d=n=3$. The $3$-elliptope is the three-dimensional spectrahedron given by
  $$
  \mathcal{E}_3 :=
  \left\{
  x
  \in \R^3 :
  \begin{bmatrix}
    1 & x_1 & x_2 \\
    x_1 & 1 & x_3 \\
    x_2 & x_3 & 1
  \end{bmatrix}
  \succeq 0
  \right\}
  =
  \left\{
  x
  \in \R^3 :
  \begin{array}{r}
    3-x_1^2-x_2^2-x_3^2 \geq 0 \\
    1+2x_1x_2x_3-x_1^2-x_2^2-x_3^2 \geq 0
  \end{array}
  \right\}.
  $$
  The polynomials on the right are the (non-constant) coefficients of the univariate polynomial
  $p(t) = \det(A(x)+t I_3)$,
  whose roots are the opposites of the eigenvalues of the defining matrix. Equivalently $\mathcal{E}_3$
  is defined by positivity of the principal minors of the defining matrix:
  $$
  1-x_1^2 \geq 0, \,\,\, 1-x_2^2 \geq 0, \,\,\, 1-x_3^2 \geq 0 \,\,\, \text{ and } \,\,\, 1+2x_1x_2x_3-x_1^2-x_2^2-x_3^2 \geq 0
  $$
  \exend
\end{example}

The optimization problem of miminizing a linear function over a real spectrahedron is called {\it semidefinite programming} (SDP). Given $c \in \R^n$ and $A_0,A_1,\ldots,A_n \in \sym_d(\R)$, the corresponding semidefinite program is defined (in standard dual form) as
\begin{equation}
  \label{SDP}
\begin{array}{rcll}
  p^* & := & \inf_{x \in \R^n} & \left\langle c, x \right\rangle \\
  &    & \text{s.t.}         & A_0+\sum_{i=1}^n x_i A_i \succeq 0.
\end{array}
\end{equation}
Real polyhedra (subsets of $\R^n$ defined by finitely many affine inequalities) are special cases of real
spectrahedra, where the defining matrices $A_0, \ldots, A_n$ can be chosen diagonal, so that $A(x) =
\ell_1(x) \oplus \cdots \oplus \ell_d(x)$. In other words, SDP is a generalization of Linear Programming (LP).

In a fixed-precision model of computation, SDP is solvable in polynomial time in the input size (matrix size,
number of variables, bit size of the coefficients), in $\log(\frac{1}{\varepsilon})$ (where $\varepsilon$ is the
precision) and $\log(\frac{1}{R})$, where $R$ is an {\it a priori} upper bound on the norm of the solution,
cf. \cite[Sec.1.9]{deKlerk}. In exact arithmetic, the complexity status of SDP is essentially open but complexity
bounds are known based either on general quantifier elimination \cite{ramana1997exact,porkolab1997complexity}
or by exploiting the determinantal structure of the constraints \cite{henrion2016exact}.

A more general class of convex semialgebraic sets are obtained by linear projections of real spectrahedra
called {\it semidefinite representable sets} (or spectrahedral shadows): these are sets of the form
$$
P = \left\{x = \left[\begin{smallmatrix} x_1 \\ \vdots \\ x_n \end{smallmatrix}\right] \in \R^n : \exists\,y\in\R^p, \, A_0 + \sum_{i=1}^n x_i A_i + \sum_{j=1}^p y_j B_j \succeq 0\right\}
$$

\begin{example}
\label{fermat_quartic}
The basic closed semialgebraic set $P = \{(x_1,x_2) \in \R^2 : 1-x_1^4-x_2^4 \geq 0\}$ is not a spectrahedron
but it is semidefinite representable:
$$
P = \left\{\begin{bmatrix} x_1 \\ x_2 \end{bmatrix} \in \R^2 :
\exists\,
\begin{bmatrix} y_1 \\ y_2 \end{bmatrix} \in \R^2, \,
\begin{bmatrix}
  1+y_1 & y_2 \\
  y_2 & 1-y_1
\end{bmatrix}
\bigoplus
\begin{bmatrix}
  1 & x_1 \\
  x_1 & y_1
\end{bmatrix}
\bigoplus
\begin{bmatrix}
  1 & x_2 \\
  x_2 & y_2
\end{bmatrix}
\succeq 0
\right\}
$$
\end{example}


\section{Polyhedra and linear programming}

%\subsection{Polyhedra}

In this section, we define polyhedra over a complete discrete valuation field $K$ with valuation $\val$ and a uniformizer $\pi$,
and we discuss the associated linear optimization problem.

\begin{definition}
  A {\it polyhedron} is a subset $P \subset K^n$ of the form
  $P = \{x \in K^n : \val(\ell_1(x))\geq0,\ldots,\val(\ell_n(x))\geq 0\}$,
  for some $\ell_1,\ldots,\ell_d \in K[x]_1$. This will often be denoted with
  the compact notation $\val(\ell_1 \oplus \cdots \oplus \ell_d) \geq 0$, where $\val$
  and $\geq$ are applied entrywise.
\end{definition}

\begin{example}
  The closed unit ball $B_\infty(K^n)$ in $K^n$ for the supremum norm $||\cdot||_{\infty}$ is the
  polyhedron defined as the set of $x=(x_1,\ldots,x_n) \in K^n$ such that $||x||_{\infty} =
  \max_i\{x_i\} \leq 1$, in other words, such that $\val(x_i) \geq 0$, for all $i=1, \ldots, n$.
  In other words, $\OK^n = B_\infty(K^n)$ is the polyhedron in $K^n$ defined by the affine polynomials
  $\ell_i(x)=x_i$, $i=1,\ldots, n$.
%  $$
%  \begin{pmatrix} x_1 &  & & \\
%    & x_2 &0 &  \\
%    &  0& \ddots & \\
%    &  & & x_s/
  %  \end{pmatrix} \ge 0.$$
\end{example}

%\subsection{linear programming}

A {\it linear programming problem} (or simply {\it linear program}) is given by the following
optimization problem:
\begin{equation}
  \tag{LP}\label{LP}
\begin{array}{rcll}
  p^* & := & \inf_{x \in K^n} & \val(\left\langle c, x \right\rangle) \\
  &    & \text{s.t.}         & A x + b \geq 0
\end{array}
\end{equation}
with $c \in K^n$ a fixed vector of size $n$ that represents the cost, $A \in \allmat_{m,n}(K)$, $b \in K^m$,
and $\left\langle c, x \right\rangle := c_1x_1+\cdots +c_nx_n$.
We present an algorithm for solving \eqref{LP} which consists in reducing the matrix $A$ in Smith normal form.
We recall its definition below, in our context.

\begin{definition}[Smith Normal Form]\label{smith_nf}
  Let $M \in \allmat_{m,n}(K)$ be of rank $r$. There exist unique integers $a_1 \leq \cdots \leq a_r$, and
  matrices $P \in \GL_n(\OK)$ and $Q \in \GL_m(\OK)$, such that
  $$
  S := QMP^{-1} = \pi^{a_1} \oplus \pi^{a_2} \oplus \cdots \oplus \pi^{a_r} \oplus {\bf 0}_{\min(m,n)-r}
  $$
  where ${\bf 0}_{\min(m,n)-r}$ is a null square matrix of size $\min(m,n)-r$.
  The matrix $S$ is called the \emph{Smith Normal Form (SNF)} of $M$, and the elements $\pi^{a_i}$ the
  \emph{invariant factors}.
\end{definition}
\begin{remark}
  The valuation of the first coefficient\footnote{invariant factor?} of the SNF of a matrix
  $M \in \allmat_{n,n} \left( K \right) $ is the minimum of the valuations of the matrix
  $M$'s terms (\simone{proof or reference}). In particular, the the SNF of matrix of $\allmat_n \left( \OK \right) $ have coefficients in $\OK$.
\end{remark}

The most interesting aspect in the SNF regarding the resolution of \ref{LP} is that the transition matrices are invertible in $\OK$. Indeed, if $z \in K^n$ and $M \in \allmat_n(\OK)$ such that $z \geq 0$ then $Mz \geq 0$. The latter property is a direct consequence of $\OK$ being a ring and the converse is true if the matrix $M$ is invertible in $\OK$ \corentin{est-ce qu'il y a besoin de le prouver ?}. Therefore, for all matrix $M \in \allmat_n(K)$ with SNF $S$ and all $z \in K^n$ $Mz \geq 0$ if and only if $Sz \geq 0$. 

That property applied to the matrix $A$ in \ref{LP} allow to reduce the problem to an equivalent simpler iteration of the \ref{LP} problem that will be called {\it diagonal linear programming problem} \corentin{or maybe reduced linear programming problem?}. This problem is the optimization problem: 
\begin{equation}
  \tag{dLP}\label{dLP}
\begin{array}{rcll}
  p^* & := & \inf_{y \in K^n} & \val(\left\langle c', y \right\rangle) \\
  &    & \text{s.t.}         & S y + b' \geq 0
\end{array}
\end{equation}
with $c' \in K^n$ a fixed vector of size $n$, $b' \in K^m$ and $S$ a diagonal $m \times n$ matrix of the form $S = s_1 \oplus s_2 \oplus \ldots \oplus s_r \oplus { \bm 0}_{m-r} $ for $r={\rm rank} S$.

The {\it associated diagonal linear programming problem} of a linear program is the iteration of the \ref{dLP} problem defined by $b' = Qb$, $c' = \left(P^{-1}\right)^T c$ and where $S$ is the Smith normal form of $A$ and $P \in GL_m\left( \OK \right)$ and $ Q \in GL_n\left( \OK \right)$ are such that $A = Q^{-1} S P$.

A vector $x^* \in K^n$ is {\it feasible solution} of \ref{LP} if $Ax^* + b \geq 0$. Similarly, a vector $y^* \in K^n$ is a feasible of \ref{dLP} if $Sy^*+b' \geq 0$. We will denote as $Adm$ \corentin{Probablement à changer mais je ne sais pas quel nom lui donner} the set of feasible solution of \ref{dLP}.

\begin{proposition}\label{prop:reduc}
	
	We have then the following results
	\begin{enumerate}
		\item $x^*$ is a feasible solution of \ref{LP} if and only if $y^* = P x^*$ is a feasible solution of the associated \ref{dLP} problem.
		\item $x$ is a solution of \ref{LP} if and only if $y = P x$ is a solution of the associated \ref{dLP} and the the infimum is the same.
		\item \ref{dLP} have feasible solutions if and only if the $m-r$ last coefficients of $b'$ have nonegative valuation, with $r$ the rank of $S$.
		\item If at least one of the $n-r$ last coefficients of $c'$ is non-zero, if \ref{dLP} admits feasible solutions it is unbounded and does not have a solution.
		 
  \end{enumerate}
\end{proposition}
	
\begin{proof}

  First of all, 1. is a consequence of the fact that $Mx^*+b = MP^{-1}y^* +b\geq 0$ if and only if $Q(MP^{-1}y^* + b) = Sy^* + b' \geq 0$ because $Q \in GL_m(\OK)$.

  Then, 2. comes from the fact that for all $c \in K^n$ $\langle c, x\rangle = \langle c, P^{-1}Py \rangle = \langle \left(P^{-1}\right)^T c, y \rangle$.

  A vector $y =  \left[\begin{smallmatrix} y_1 \\ \vdots \\ y_n \end{smallmatrix}\right] \in K^n$ is feasible solution of \ref{dLP} if and only if $\val(s_i y_i + b_i) >=0$ for all $i \in \{1...n\}$ \corentin{préciser les notations}. 
  However, $s_i =0$ for $n-r \leq i\leq n$. Therefore, \ref{dLP} have solution only if the $m-r$ last coefficient of b' are zero.
  
  Conversely, if the $m-r$ last coefficient of $b'$ are zero, the vector $y \in K^n$ such that $y_i = -\frac{-b'_i}{s_i}$ if $1 \leq i \leq r$ and $y_i =0$ otherwise is a feasible solution.

  Finally, let $y^*$ be a feasible solution of \ref{dLP}. Let us suppose there exists $r+1 \leq i \leq m$ such that $c'_i \neq  0$ and fix such $i$. Let us then consider the vector $e^i$ whose only non-zero coefficient is the $i-$th, which is $1$. Then $y^*+\lambda e^i$ is a feasible solution for all $\lambda \in K$ and $\langle c', y^* + \lambda e^i\rangle = \langle c' , y^* \rangle + \lambda c'_i$. In particular, for $\lambda = -\frac{\langle c', y^* \rangle}{c'_i}$ $\val \left( \langle c', y^* + \lambda e ^i \rangle\right) = \val (0) = + \infty$. 

\end{proof}

The two first points of the proposition \ref{prop:reduc} show that we can resolving a linear program is equivalent to resolution the associated \ref{dLP} problem. Furthermore, if one considers only the iteration of \ref{dLP} which have solutions. By point 3. and 4. of proposition \ref{prop:reduc} one can reduce the problem by only taking acount only the $r$ first coefficients of $y, b', c'$ and the submatrix of $S$ made with only $r$ first lines et columns of and which coefficients are exactly the nonzero invariants factors. The set $Adm$ of the feasible solutions can then be written as the set of vectors $y \in \mathbb{Q}_{ p } ^n$ that verifies $\forall 1 \le i\le r \ s_i y_i + b'_i \in \OK$ i.e. that verifies :

\begin{equation}
	\forall 1 \le i\le r  \ y_i \in -\frac{b`_i}{s_i} + \frac{1}{s_i} \OK
\end{equation}


Resolving \ref{dLP} consists in minimizing $\val\left(\left<c',y \right>\right)$ over $Adm$. The image of $Adm$ by $y \mapsto \left<c',y \right>$ is $\sum_{i=1}^r -c'_i.\frac{b'_i}{s_i} + \sum_{i=1}^r\left( \frac{c'_i}{s_{i}} \OK \right)$ which can be rewritten in $\left<c',Adm \right> = \lambda + \pi^{v} \OK$ with $\lambda = \sum_{i=1}^r -c'_i.\frac{b'_i}{s_i}$ and $v = \min\limits_{1\le i\le r} \val \frac{c'_{i}}{s_{i}} $.

Finally, two cases can occur depending on $v$ and the valuation of $\lambda$.
\begin{itemize}
	\item If $\val( \lambda) < v$ the minimum of $y\mapsto \val\left(\left<c',y \right>\right)$ over $Adm$ is reached in any point of $Adm$ and is equal to $\val\left( \lambda\right)$.
	\item If $\val\left( \lambda \right) \ge v$ then $\lambda \in \pi^{v} \OK$ and the minimum is $v $ and is reached in any points $y$ of $Adm$ that verifies 
	\[\val \underset{ \begin{array}{c} 1\le i\le r\\ \val\left(c'_{i}/{s_{i}} \right) = v  \end{array}}{\sum} y_{i} = 0. \]
	
\end{itemize}
\begin{remark}
	To maximize the valuation of a linear map over a convex $p$-adic polyhedron instead of minimizing it (which is equuivalent to minimize the absolute value) the same reasoning can be aplied. The only difference being that if $\val\left( \lambda\right) < v$ the problem is unbounded and as such does not have any solution. 
\end{remark}




\section{Spectrahedra}

\subsection{Definition}
%\newcommand\mat{postive semidefinite matrix } 
\newcommand\Mat{Positive semidefinite matrix }
\newcommand\mats{positive semidefinite matrices }
\newcommand\Mats{positive semidefinite matrices }
\begin{definition}
	We call ... any matrix $M \in \mathcal{M}_n\left( K \right) $ whose eigenvalues have nonegative valuation (in $\overline{K}$) .
	
	We will denote as $ \mathcal{M}_n^+\left( K \right)$ the set of \mats and we will write $M \succeq 0$ if $M \in \mathcal{M}_n^+\left( K\right) $.
\end{definition}
\begin{theorem}
	\label{caracsdp}
	Characterization of the \mats
	
	A matrix is positive semidefinite if and only if its characteristic polynomial have coefficients in $\OK$.
	
\end{theorem}


\begin{corollary}
	$\mathcal{M}_n(\OK) \subset \mathcal{M}_n^+\left( K \right)$ 
\end{corollary}

\begin{proof}
$\OK$ is a ring therefore the characteristic polynomial of matrix whose coefficients are $p$-adic integers have coefficients ins $Pp$. We conclude by \ref{caracsdp} .
\end{proof}



\begin{remark}
	In general, the reverse inclusion is false. For instance the characteristic polynomial of $M = \begin{pmatrix} 5 + \frac{3}{5} & \frac{4}{5} \\ \frac{4}{5} & -\frac{3}{5} \end{pmatrix} $ is $\chi_M = X^2  - 5 X - 4$. However when seen as an element of $\mathbb{Q}_5$, $\chi_M \in \mathbb{Z}_5[X]$ then $M \in \mathcal{M}_2^+\left( \mathbb{Q}_5 \right)$ but no coefficient of $M$ is in $\mathbb{Z}_5 $
\end{remark}

\begin{proposition}
	The set $\mathcal{M}_n^+\left( K \right)$ is :
	\begin{enumerate}%[label = \textit{\roman*}.]
		\item open
		\item closed
		%\item a convex cone
	\end{enumerate}
\end{proposition}

\textit{Preuve :}  
Firstly, $\OK[X]$ is closed and open in $\mathbb{Q}_{ p }[X] $ equiped with the infinitiy norm $\|.\|_\infty : P = \sum_{k=1}^{n} a_k X^k \to \sup |a_k|_p$. $\OK$ is indeed the open unit ball for $\|\cdot \|_\infty$ which defines a discrete distance for which open balls are also closed.
Furthermore, $\mathcal{M}^+_{n}\left(K \right)  = \chi^{-1}( \OK[X]) $ where $\chi$ is the function that maps its characteristic polynomial to a matrix which is continuous.

%Le $iii$. est laissé en exercice au lecteur. 
% Je ne parle plus de convexité dans le rapport mais on ne sait jamais
%\begin{remarque}
	L'ensemble $S_n^+\left( \Qp \right) $ n'est pas convexe en général.

	Par exemple pour $p = 5$, si on considère les matrices $M_1 = \begin{pmatrix} 5 + \frac{3}{5} & \frac{4}{5} \\ \frac{4}{5} & -\frac{3}{5} \end{pmatrix} $ et $M_2 = \begin{pmatrix} 25+\frac{7}{25} & \frac{24}{25} \\ \frac{24}{25}&-\frac{7}{25} \end{pmatrix} $, on a $\chi_{M_1} = X^2 -5 X -4  $ et $\chi_{M_2} = X^2 - 25 X -8$ donc en plongeant ces matrices dans $\mathbb{Q}_5$ il vient que $\chi_{M_1}, \chi_{M_2} \in \mathbf{Z}_5[X]$ i.e. $M_1,M_2 \in S_2^+\left( \Q_5 \right) $. Or $\chi_{M_1 + M_2}  = X^{2} - 150 X - \frac{3784}{5}$ qui une fois plongé dans $\mathbb{Q}_5$ n'est pas à coefficient dans $\mathbb{Z}_5$, donc $ S_2^+\left( \Q_5 \right) $ n'est pas convexe. Ce résultat se généralise pour tout $n$ en considérant les matrice par blocs $M'_i = \text{diag}\left( M_{i},0,\ldots,0 \right) $.

\end{remarque}



\begin{definition}
	We call {\it spectrahedron} the intersection of $\mathcal{M}_n^+\left( K \right) $ with an affine plan $\mathcal{L}$ of $\mathcal{M}_n\left( K \right) $.
\end{definition}

\begin{remark}
	As in the real case, the spectrahedron associated with the affine plan $\mathcal{L}$ will often be identified with its preimage in $K^s$ (i.e. the set of vectors $x \in K ^s$ so that the linear matrix $A(x) = A_0 + x_1A_1 + \ldots + x_sA_s$ is positive semidefinite with $A_0,A_1,\ldots,A_s$ spanning $\mathcal{L}$). 
\end{remark}

The latter allows gives us the following property :
\begin{proposition}
	A polyhedron is a spectrahedron.
\end{proposition}
\begin{proof}
	Same as in the real case (maybe modify it or add the link to real proof).
\end{proof}


\subsection{Annuli (couronnes)}
\begin{definition}
	Annulus
	
	We call {\it annulus} a set $A$ defined as $C = \{x \in K  | a\le \val\left(x\right) \le b\} $ for $a < b$ two nonnegative real numbers.
\end{definition}


\begin{theorem}
	Annuli are spectrahedral shadows.
\end{theorem} 

\textit{Preuve :}
Let $A$ be the annuli defined by $a<b \in R^*_+$.

We consider for all $x,y \in K $ the matrix  $$M(x,y) :=
\begin{pmatrix} 
	p^ax & 0 & 0 & 0 \\
	0 & p^{-b}y & 0 & 0 \\
	0 & 0 & p^{-1} & p^{-1}x \\
	0 & 0 & p^{-1}y & - p^{-1}  \\
\end{pmatrix} $$ and the associated spectrahedron $\mathcal{S}= \{(x,y) \in K  : M(x,y) \succeq 0\} $

Let us show that $x \in C$ if and only if  $\exists y \in K \left( x,y \right) \in \mathcal{S}$.

Let $x,y$ be two $p$-adic numbers.
$M(x,y)$ can be decomposed in three blocks : $p^ax$, $p^{-b}y $ and $M'(x,y) = \begin{pmatrix} p^{-1} & p^{-1}x \\ p^{-1} y & p^{-1}\end{pmatrix} $. Using \ref{caracsdp} we know that $M(x,y)$ is positive semidefinfite if and only if
$
\begin{cases}
	p^ax \ge 0 \\
	p^{-b}y \ge 0\\
	\text{Tr}M'\left( x,y \right) = 0 \ge 0\\
	\text{det} M'(x,y ) = p^{-2}\left( xy-1 \right)  \ge 0 
\end{cases}
$ 
that is if and only if 
\begin{equation}
	\label{eq:caracsdp} 
	\begin{cases} 
		\val\left(x\right)\ge a\\
		\val\left(y\right)\ge -b\\
		p^{-2} \left( xy-1 \right) \ge 0
	\end{cases}
\end{equation}




Therefore $x \in C$ if and only if there exists $y \in K $ so that $(x, y)$ verifies \ref{eq:caracsdp}. Si $x, y$ verifies indeed \ref{caracsdp} then $\val\left(x\right)\ge a$, $\val\left(y\right)\ge -b$ and $p^{-2} \left( xy-1 \right) \ge 0$ implies that $\val\left(x\right)+\val\left(y\right)=\val\left(xy\right) = \val\left(-1\right) =0$ and therefore that $\val\left(x\right)=-\val\left(y\right)\le -(-b)  =b$, so $x \in $. Reciprocally if $x \in C$ then $(x,x^{-1}) $ verifies \ref{eq:caracsdp}  .


\subsection{Polyhedrality}
Cf. \cite{bhardwaj2015deciding}

\section{Linear matrix inequalities}


\section{Other questions}
\begin{itemize}
\item The projection of a real polyhedron is a real polyhedron (thanks to Fourier-Motzkin elimination procedure). Is it the same for $K$-polyhedra?
\item Annuli are semidefinite representable, we should remark/prove that they are not spectrahedra.
\end{itemize}

\printbibliography
\end{document}
