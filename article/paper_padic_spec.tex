\documentclass[a4paper,12pt]{article}

%%%\newenvironment{proof}{\hbox{}\vspace{-0.8cm} {\bf Proof:}}{\hfill $\Box$ \vspace{0.2cm}}

\usepackage{amsmath}
\usepackage{amsthm}
\usepackage{amssymb}
\usepackage{amsfonts}
\usepackage{mathrsfs} 
\usepackage{graphicx}
\usepackage{color}
\usepackage{ulem}
\usepackage{bm}

\usepackage{cleveref}

\newtheorem{theorem}{Theorem}[section]
\newtheorem{lemma}{Lemma}
\newtheorem{proposition}{Proposition}
\newtheorem{corollary}{Corollary}
\newtheorem{algorithm}{Algorithm}
\newtheorem{conjecture}{Conjecture}
\newtheorem{condition}{Condition}
\newtheorem{definition}{Definition}
\newtheorem{assumption}{Assumption}
\newtheorem{remark}{Remark}
\newtheorem{problem}{Problem}
\newtheorem{example}{Example}
\newtheorem{notation}{Notation}
\usepackage{booktabs}

\textheight235mm
\textwidth165mm
\voffset-10mm
\hoffset-12.5mm
\parindent0.8cm
\parskip0mm


\usepackage{stmaryrd} %for double brackets

%\usepackage{mathptmx}

% updates

\newcommand{\R}{\mathbb{R}} % real numbers
\newcommand{\C}{\mathbb{C}} % complex numbers
\newcommand{\N}{\mathbb{N}} % integers
\newcommand{\allmat}{\mathbb{M}} % matrices
\newcommand{\sym}{\mathbb{S}} % matrices symétriques
\newcommand{\PP}{\mathcal{P}}
\newcommand{\Qc}{\mathcal{Q}}
\renewcommand{\span}[1]{{\text{span}(#1)}} % simone's comments
\newcommand{\aff}[1]{{\text{aff}(#1)}} % simone's comments
\newcommand{\relint}[1]{{\text{relint}(#1)}} % simone's comments
\newcommand{\calL}{\mathcal{L}} % simone's comments

\newcommand{\simone}[1]{{\color{blue} #1}} % simone's comments
\newcommand{\corentin}[1]{{\color{red} #1}} % corentin's comments
\newcommand{\tristan}[1]{{\color{red} #1}} % tristan's comments

%p-adic commands
\DeclareMathOperator{\val}{val}
\def\QQ{\ensuremath{\mathbb{Q}}}
\def\ZZ{\ensuremath{\mathbb{Z}}}
\newcommand{\OK}{\mathcal{O}_K}
\def\diag{\mathrm{diag}}

\newcommand{\GL}{\mathrm{GL}}

\newcommand{\exend}{\hfill $\blacksquare$}


\usepackage{biblatex}
\addbibresource{../rapport/bibstage.bib}

%\title{\bf Title}
%\title{\bf On polyhedra and spectrahedra over valuation fields}
\title{\bf On semidefinite representable sets over valuation fields}

\begin{document}

\author{Corentin Cornou$^{1}$, Simone Naldi$^{2,3}$ and Tristan Vaccon$^{2}$}

\footnotetext[1]{ENS Paris-Saclay, Université Paris-Saclay, France.}
\footnotetext[2]{Université de Limoges, CNRS, XLIM, UMR 7252, F-87000 Limoges, France.}
\footnotetext[3]{Sorbonne Université, CNRS, LIP6, Equipe PolSys, F-75005 Paris, France.}

\date{Draft of \today}

\maketitle

\begin{abstract}
  \noindent
  Real polyhedra and real spectrahedra (or their projections) are respectively the feasible sets
  of linear programming (LP) and semidefinite programming (SDP). 
  This paper investigates versions of these sets and of
  the related optimization problems with data over a field $K$ with valuation.
  DIRE CE QU'ON FAIT.
  We showcase examples of non-polyhedral $K$-spectrahedra and of sets that are semidefinite
  representable over $K$ but are not $K$-spectrahedra.
\end{abstract}

\tableofcontents

\section{Introduction}

%\subsection{Context and motivations}
%Il faudra entre autre mentionner (et peut être motiver l'approche p-adique) :
%\begin{itemize}
%\item les questions ouvertes de complexité concernant la programmation semidefinie (pas clair si NP $\cap$ co-NP
%  dans le modele de Turing)
%\item les questions ouvertes géométriques (Conjecture de Lax)
%\end{itemize}
%\subsection{Main results}


%\newpage
\section{Preliminaries}

\subsection{Notation}

Throughout the paper, $K$ refers to a complete,
discrete valuation field, $\val : K \twoheadrightarrow \ZZ \cup \{+\infty\}$ to its valuation,
$\OK$ its ring of integers and $\pi$ a uniformizer.
For $k \in \N$, we write $O(\pi^k)$ for $\pi^k \OK$.
A typical example of $K$ as above is the field of $p$-adic numbers 
$\QQ_p$ (equipped with the $p$-adic valuation). For this example, we 
have $\OK = \ZZ_p$.
Another example is the field of Laurent series
$K=\QQ(\!(t)\!)$ equiped with $t$-adic valuation,
in which case $\OK = \QQ \llbracket t \rrbracket$.
We refer to \cite{Serre:1979} for a
general introduction to such fields
and to Caruso's course on $p$-adic computations \cite{caruso_computations_2017}
for an introduction to effective computations over $p$-adic numbers.

The $K$-vector space of polynomials of degree $\leq d$ on $x=(x_1,\ldots,x_n)$
with coefficients in $K$ is denoted by $K[x]_{d}$. The vector space of matrices of
size $p \times q$ with entries in a ring $R$ is $R^{p \times q}$ and the general
linear group in $R^{n \times n}$ is $\GL_n(R)$.
For the sake
of ease of writing, block-diagonal matrices with blocks $B_1,\ldots,B_d$ are denoted by
$\diag(B_1, \ldots, B_d)$.

We recall the definition of Smith Normal Form, in our context.
\begin{definition}[Smith Normal Form]\label{smith_nf}
  Let $M \in K^{m \times n}$ be of rank $r$. There exist unique integers $a_1 \leq \cdots \leq a_r$, and
  matrices $P \in \GL_n(\OK)$ and $Q \in \GL_m(\OK)$, such that
  $$
  SNF(M) := QMP^{-1} = \diag(\pi^{a_1}, \pi^{a_2}, \ldots, \pi^{a_r}, {\bf 0}_{\min(m,n)-r})
  $$
  where ${\bf 0}_{t}$ is a null square matrix of size $t$.
  The matrix $SNF(M)$ is called the \emph{Smith Normal Form (SNF)} of $M$, and the elements $\pi^{a_i}$ the
  \emph{invariant factors}.
\end{definition}
\begin{remark}
  The valuation of the first invariant factor of $SNF(M)$ of a matrix
  $M \in K^{n \times n}$ is the minimum of the valuations of the entries of
  $M$. In particular, $SNF(M) \in \OK^{n \times n}$.
\end{remark}


\subsection{Real spectrahedra}

Let $\sym_d(\R) \subset \R^{d\times d}$ be the vector space of $d \times d$ real symmetric
matrices. A matrix $M \in \sym_d(\R)$
is called \emph{positive semidefinite} ($M \succeq 0$) whenever the quadratic
form $x \mapsto x^TMx$ is nonnegative for all $x\in \R^n$. By the Spectral Theorem, $M \succeq 0$
if and only if the eigenvalues of $M$ are nonnegative, and this is equivalent to all the principal minors
of $M$ being nonnegative. The set of positive semidefinite matrices, denoted $\sym_d^+(\R) \subset \sym_d(\R)$,
is a closed convex cone with non-empty interior in $\sym_d(\R)$.

Let $A_0,A_1,\ldots,A_n \in \sym_d(\R)$, and let $\calL = A_0+\span{A_1,\ldots,A_n}$ be the affine space
containing $A_0$ and with direction the vector space spanned by $A_1,\ldots,A_n$ (which we might assume
to be linearly independent). The intersection
$$
\calL \cap \sym_d^+(\R) = \left\{A(x) := A_0+\sum_i x_i A_i \in \sym_d(\R) :
A(x) \succeq 0\right\},
$$
or equivalently, its pre-image $S = \{x \in \R^n : A(x) \succeq 0\}$ under the map $x \mapsto A(x)$, is called
a \emph{real spectrahedron}. As (linear preimage of) affine sections of $\sym_d^+(\R)$, real spectrahedra
are closed convex sets, but might have empty interior in $\calL$ (resp. in $\R^n$).
Moreover, they are basic semialgebraic sets, indeed they are defined by sign conditions on the principal
minors of the defining matrix as well as sign conditions on the coefficients of its characteristic polynomial
(cf. \Cref{ell3}).

\begin{example}\label{ell3}
  Let $d=n=3$. The $3$-elliptope is the three-dimensional spectrahedron given by
  $$
  \mathcal{E}_3 :=
  \left\{
  x
  \in \R^3 :
  \begin{bmatrix}
    1 & x_1 & x_2 \\
    x_1 & 1 & x_3 \\
    x_2 & x_3 & 1
  \end{bmatrix}
  \succeq 0
  \right\}
  =
  \left\{
  x
  \in \R^3 :
  \begin{array}{r}
    3-x_1^2-x_2^2-x_3^2 \geq 0 \\
    1+2x_1x_2x_3-x_1^2-x_2^2-x_3^2 \geq 0
  \end{array}
  \right\}.
  $$
  The polynomials on the right are the (non-constant) coefficients of the univariate polynomial
  $p(t) = \det(A(x)+t I_3)$,
  whose roots are the opposites of the eigenvalues of the defining matrix. Equivalently $\mathcal{E}_3$
  is defined by positivity of the principal minors of the defining matrix:
  $$
  1-x_1^2 \geq 0, \,\,\, 1-x_2^2 \geq 0, \,\,\, 1-x_3^2 \geq 0 \,\,\, \text{ and } \,\,\, 1+2x_1x_2x_3-x_1^2-x_2^2-x_3^2 \geq 0.$$ \exend
\end{example}

The optimization problem of miminizing a linear function over a real spectrahedron is called \emph{semidefinite programming} (SDP). Given $c \in \R^n$ and $A_0,A_1,\ldots,A_n \in \sym_d(\R)$, the corresponding semidefinite program is defined (in standard dual form) as
\begin{equation}
  \label{SDP}
\begin{array}{rcll}
  p^* & := & \inf_{x \in \R^n} & \left\langle c, x \right\rangle \\
  &    & \text{s.t.}         & A_0+\sum_{i=1}^n x_i A_i \succeq 0.
\end{array}
\end{equation}
Real polyhedra (subsets of $\R^n$ defined by finitely many affine inequalities) are special cases of real
spectrahedra, where the defining matrices $A_0, \ldots, A_n$ can be chosen diagonal, so that $A(x) =
\ell_1(x) \oplus \cdots \oplus \ell_d(x)$. In other words, SDP is a generalization of Linear Programming (LP).

In a fixed-precision model of computation, SDP is solvable in polynomial time in the input size (matrix size,
number of variables, bit size of the coefficients), in $\log(\frac{1}{\varepsilon})$ (where $\varepsilon$ is the
precision) and $\log(\frac{1}{R})$, where $R$ is an {\it a priori} upper bound on the norm of a solution,
cf. \cite[Sec.1.9]{deKlerk}. In exact arithmetic, the complexity status of SDP is essentially open but complexity
bounds are known based either on general quantifier elimination \cite{ramana1997exact,porkolab1997complexity}
or by exploiting the determinantal structure of the constraints \cite{henrion2016exact}.

A more general class of convex semialgebraic sets are obtained by linear projections of real spectrahedra
called \emph{semidefinite representable sets} (or spectrahedral shadows): these are sets of the form
$$
P = \left\{x = \left[\begin{smallmatrix} x_1 \\ \vdots \\ x_n \end{smallmatrix}\right] \in \R^n : \exists\,y\in\R^p, \, A_0 + \sum_{i=1}^n x_i A_i + \sum_{j=1}^p y_j B_j \succeq 0\right\}
$$

\begin{example}
\label{fermat_quartic}
The basic closed semialgebraic set $P = \{(x_1,x_2) \in \R^2 : 1-x_1^4-x_2^4 \geq 0\}$ is not a spectrahedron
but it is semidefinite representable:
$$
P = \left\{\begin{bmatrix} x_1 \\ x_2 \end{bmatrix} \in \R^2 :
\exists\,
\begin{bmatrix} y_1 \\ y_2 \end{bmatrix} \in \R^2, \,
\diag\left(
\begin{bmatrix}
  1+y_1 & y_2 \\
  y_2 & 1-y_1
\end{bmatrix},
\begin{bmatrix}
  1 & x_1 \\
  x_1 & y_1
\end{bmatrix},
\begin{bmatrix}
  1 & x_2 \\
  x_2 & y_2
\end{bmatrix}
\right)
\succeq 0
\right\}
$$
\end{example}


\section{Polyhedra}

In this section, we define polyhedra over a complete discrete valuation field $K$ with valuation $\val$ and a uniformizer $\pi$, and we prove that the class of polyhedra is closed under linear transformations, as in the real case. We end by presenting an algorithm for solving linear optimization problems over $K$.

\begin{definition}
  \label{def_polyhedra}
  A \emph{polyhedron} is a subset $P \subset K^n$ of the form
  $$
  P = \{x \in K^n : \val(\ell_i(x)) \geq 0, i=1,\ldots,d \text{ and }
  \val(m_j(x)) = +\infty, j=1,\ldots,e\},
  $$
  for some $\ell_1,\ldots,\ell_d,m_1,\ldots,m_e \in K[x]_1$.
\end{definition}

%%We recall that that the smallest affine space of $K^n$ containing a set $S \subset K^n$ is called its \emph{affine hull} and denoted $\aff{S}$. The assumption on forms $\ell_i$ in \Cref{def_polyhedra} implies that the constraint $\val(m_1 \oplus \cdots \oplus m_e) = +\infty$ defines exactly the affine hull of $P$, in other words $\aff{P} = \{x \in K^n : m_j(x) = 0, j=1,\ldots,e\}$ \simone{(il nous faudrait une preuve, du genre: $\val(m(x))=+\infty$ pour tout $x \in P$ si et seulement si $m$ est une combinaison affine de $m_1,\ldots,m_e$.)}
%%The \emph{dimension} of $P$ is defined as the dimension of $\aff{P}$ (for $e=0$, $\aff{P}=K^n$ and $P$ has dimension $n$). The \emph{relative interior} of $P$ is defined as $\relint{P} = \{x \in P : \val(\ell_1 \oplus \cdots \oplus \ell_d) < +\infty\}$ (\simone{pas clair si besoin de dimension et relint}).
Remark that \Cref{def_polyhedra} includes the case $d=0$ corresponding to affine spaces, as in the real case.
Moreover by definition the class of polyhedra is closed under intersection. Without loss of generality, one may
assume in \Cref{def_polyhedra} that for every $i=1,\ldots,d$ there exists $x \in P$ such that
$\val(\ell_i(x))<+\infty$.

\begin{example}
  The \emph{closed unit ball} $\OK^n=B_\infty(K^n)$ in $K^n$ for the supremum norm $||\cdot||_{\infty}$ is the
  polyhedron defined as the set of $x=(x_1,\ldots,x_n) \in K^n$ such that $||x||_{\infty} =
  \max_i\{x_i\} \leq 1$, in other words, such that $\val(x_i) \geq 0$, for all $i=1, \ldots, n$.
  In other words, $\OK^n$ is the polyhedron in $K^n$ defined by the affine polynomials
  $\ell_i(x)=x_i$, $i=1,\ldots, n$, and $e=0$. %%%%%(its affine hull is indeed the whole $K^n$).
\end{example}

\subsection{Projections of polyhedra}

%{\color{blue} Normalement avec la nouvelle definition, ça devrait aller, il faut par contre
%  partir d'un polyedre defini aussi par des égalités. Peut être il faudrait aussi
%  diviser la proposition en plusieures parties pour la rendre plus lisible, par exemple une
%  première proposition avec la projection.}

In this section we show that the image of a polyhedron by a linear mappingA is a polyhedron. 
We first prove the result in the cases of authomorphisms of $\OK^n$ and diagonal endomorphisms of $K^n$.
Then we use these particular cases to prove the result for projections of the form $(x_1,\dots,x_n) \mapsto (x_1,\dots, x_{n-1})$
and finally for the general case.

\begin{remark}
 First, let $f : K^n \to K^m$ be a linear map and $\PP \subset K^n$ be a polyhedron defined by 
$\PP = \left\lbrace x \in K^n : \val (l_i (x) ) \geq 0, m_i(x)=0, \textrm{ for } i \in I, j \in J \right\rbrace$ 
for some finite sets $I \subset \N$ and $J \subset \mathbb{N}$ and $m_j, l_i \in K[x]_1$, for $i \in I$  and $j \in J$.
Then  $\PP = \PP_i \cap \PP_e$ where $\PP_i := \{x \in K^n : \val(l_i(x)) \ge 0, i = 1,\dots,d\}$
and $\PP_e := \{x \in K^n : m_j(x) = 0, j = 1,\dots,e\}$.

Thus, we have  $f(\PP) = f(\PP_i) \cap f(\PP_e)$.
However, $\PP_e$ is either empty or an affine subspace of $K^n$ and therefore 
$f(\PP_e)$ is either empty or an affine subspace of $K^m$, so it is a polyhedron.
Hence, as the class of polyhedron is closed under intersection to prove that
$f(\PP)$ is a polyhedron it is sufficient to prove that $f(\PP_i)$ is one.
\end{remark}

Therefore, we can only consider the image of polyhedron of the form 
$\left\lbrace x \in K^n : \val (l_i (x) ) \geq 0 \textrm{ for } i \in I \right\rbrace$.

\begin{proposition}
  Let $\PP \subset K^n$ be a polyhedron and $f : K^n \to K^n$ be a linear mapping.
  If $f \in \GL_n(\OK)$ or if there exists $\lambda_1, \dots, \lambda_n \in K$ such that 
  $(x_1,\dots,x_{n-1}, x_n) \mapsto (x_1,\dots, x_{n-1})$.
\end{proposition}

\begin{proof}
We assume that $\PP \neq \emptyset$
As discussed above we can assume that
$\PP = \left\lbrace x \in K^n : \val (l_i (x) ) \geq 0,\textrm{ for } i \in I \right\rbrace$ for some
finite set $I \subset \N$ and $l_i \in K[x]_1$, for $i \in I$ .

First, if $f \in \GL_n(\OK)$ then, $x \in f(P)$ if and only if for all $i \in I$, 
$\val(l_i (f^{-1}(x))) \geq 0$.
The $l_i \circ f^{-1}$ are clearly in $K[x]_1$
and $f(P)$ is a polyhedron.

Similarly, if $f:K^n \rightarrow K^n$ is of the form
$f(x_1,\dots,x_n)=(\lambda_1 x_1,\dots,\lambda_n x_n)$
for some $\lambda_1,\ldots, \lambda_n \in K$,
then of course $f(P)$
can be defined by $\val (\widetilde{l_i}(x)) \geq 0$, $i \in I$
with the $\widetilde{l_i}$'s obtained from the $l_i$'s by an
obvious change of variables.
\end{proof}

The two above results allows us to demonstrate the following proposition, which we will use to prove the general case. 

\begin{proposition}
      Let $\PP \subset K^n$ be a polyhedron and $p : K^n \to K^{n-1}$ be the projection $p : (x_1,\dots,x_{n-1},x_n) \mapsto (x_1, \dots, x_{n-1})$.
       Then $p(\PP)$ is a spectrahedron.
\end{proposition}


  \begin{proof}
      
We assume that $\PP \neq \emptyset$.
As discussed above we can assume that
  $\PP = \left\lbrace x \in K^n : \val (l_i (x) ) \geq 0,\textrm{ for } i \in I \right\rbrace$ for some
finite set $I \subset \N$ and $l_i \in K[x]_1$, for $i \in I$ .

Let us write the conditions defining $\PP$
in matrix form:
$\val(A X+B) \geq 0$ and 
with $A \in M_{q,n}(K)$, $B \in K^q$
and $X=(x_1,\dots,x_n)^\intercal$. 


We can compute a Smith Normal Form
of the $q \times (n-1)$ matrix defined by the 
$n-1$ first columns of $A$
to obtain a decomposition of the form:
\[ Q_A A P_A^{-1} = \begin{bmatrix}
\delta_1	& 		& 			&   &		 &  &a_1	\\
			& \ddots& 			& 0	&		 &	&		\\
			&		& \delta_l  &   & 		 &	&		\\
			&		&			&0  & 		 &	& 		\\
			&		&			&   & \ddots &	&		\\
			&		&			&	&		 & 0&a_{n-1} \\
			&		&			&	&		 &	&a_n 	\\
			&		&			&	&		 &	&    	\\
			&		&			&	&		 &	&a_q 	\\			
\end{bmatrix},\]
with $Q_A \in \GL_q(\OK)$ and $P_A \in \GL_n(\OK)$ such that $P_A(e_n)=e_n.$

Multiplying on the left by $Q_A^{\pm 1}$ does not
modify whether the valuation inequalities are satisfied.
Moreover, thanks to our previous on the case when $f$ is an automorphism,
$\PP$ can be written
$\PP=P_A^{-1}(\PP_2)$ for some polyhedron $\PP_2 \subset K^n$
and we are reduced to proving that
$f(\PP_2)$ is polyhedron for $p$ still the canonical projection
$K^n \rightarrow K^{n-1}$ and $\PP_2$
defined by the linear matrix inequality (for some $B_2 \in K^q$):
\[  \begin{bmatrix}
\delta_1	& 		& 			&   &		 &  &a_1	\\
			& \ddots& 			& 0	&		 &	&		\\
			&		& \delta_l  &   & 		 &	& \vdots\\
			&		&			&0  & 		 &	& 		\\
			&		&			&   & \ddots &	&		\\
			&		&			&	&		 & 0&a_{n-1} \\
			&		&			&	&		 &	&a_n 	\\
			&		&			&	&		 &	& \vdots   	\\
			&		&			&	&		 &	&a_q 	\\			
\end{bmatrix} \begin{bmatrix} x_1 \\ \\ \\ \\ \\ \vdots \\ \\ \\ \\ x_n \\ \end{bmatrix} + B_2 \geq 0.\]



We remark that 
$a_s x_n + b_s \geq 0$
if and only if $x_n = -a_s^{-1} b_s + O(\pi^{- \val(a_s)}).$
This exactly corresponds to $x_n \in B(-a_s^{-1} b_s, 2^{\val(a_s)}).$
Since the conditions have to be compatible (we have assumed
$\PP$ not empty), and thanks to ultrametricity,  
the intersection of these conditions for 
$s \in \llbracket l+1,q \rrbracket$
is then given by the smallest of these balls.
Let us assume it is the one defined for $l=n.$
The only condition on $x_n$
is then $\val(a_nx_n +b_n) \geq 0.$

Thanks to the above discussion on diagonal mapping,
we can further assume that the $\delta_i$'s and $a_n$ are all $1$'s
and are reduce to $\PP_2$ defined by the linear matrix inequality :
\[  \begin{bmatrix}
1	& 		& 			&   &		 &  &a_1	\\
			& \ddots& 			& 0	&		 &	&		\\
			&		& 1  &   & 		 &	& \\
			&		&			&0  & 		 &	& 	\vdots	\\
			&		&			&   & \ddots &	&		\\
			&		&			&	&		 & 0&a_{n-1} \\
			&		&			&	&		 &	&1 	\\
		
\end{bmatrix} \begin{bmatrix} x_1 \\ \\  \\ \\ \vdots \\  \\ \\ x_n \\ \end{bmatrix} + \begin{bmatrix} b_1 \\ \\ \\  \\ \vdots  \\ \\ \\ b_n \\ \end{bmatrix}  \geq 0.\]
Finally, we can also assume that, up to a permutation of the variables, $\val(a_1)= \min_{i=1}^{n-1} \val(a_i).$
We distinguish two cases: $\val(a_1) \geq 0$ and $\val(a_1)<0.$

Let us first assume that $\val(a_1) \geq 0$

Let $x \in \PP_2$.
Then $x_n=-b_n+O(1)$
and $x_1+a_1 x_n+b_1=O(1).$
Consequently, $x_1-a_1 b_n +b_1=O(1)+O(\pi^{\val(a_1)})=O(1).$
Similarly, for all $i \in \llbracket 2,n-1 \rrbracket,$
$x_i-a_ib_n+b_i=O(1)+O(\pi^{\val(a_i)})=O(1).$

We define the polyhedron $\Qc \subset K^{n-1}$ by the following inequalities:
 for all $i \in \llbracket 1,n-1 \rrbracket,$
$\val(x_i-a_ib_n+b_i) \geq 0.$

Then the previous computations prove that $p(\PP_2)\subset \Qc.$
Conversely, let $(x_1,\dots,x_{n-1}) \in \Qc.$
Then let $x_n := -b_n.$

We check that for $i \in \llbracket 1,n-1 \rrbracket,$
$x_i+a_ix_n+b_i=x_i-a_ib_n+b_i=O(1),$
and $x_n+b_n=O(1)$. Hence $(x_1,\dots,x_n) \in \PP_2$
which means that $(x_1,\dots,x_{n-1}) \in p(\PP_2).$
Thus $p(\PP_2)= \Qc$ and $p(\PP_2)$ is a polyhedron.

We now deal with the second case, with $\val(a_1)<0.$
Let $x \in \PP_2$.
Then $x_n=-b_n+O(1)$
and $x_1+a_1 x_n+b_1=O(1).$
Consequently, $x_1-a_1 b_n +b_1=O(1)+O(\pi^{\val(a_1)})=O(\pi^{\val(a_1)}).$
Thus $a_1^{-1}x_1- b_n +a_1^{-1}b_1=O(1)$.
%and $x_n=-a_1^{-1}(x_1+b_1)+O(1).$
In addition, $x_n=-a_1^{-1}(x_1+b_1)+O(\pi^{-\val(a_1)}).$
We plug this equality in 
$x_i+a_i x_n+b_i=O(1)$ to get
$x_i-a_1^{-1}a_i(x_1+b_1)+b_i=O(1)+O(\pi^{\val(a_i)-\val(a_1)})=O(1)$
using that $\val(a_i) \geq \val (a_1).$

We define the polyhedron $\Qc \subset K^{n-1}$ by the following inequalities:
$\val(a_1^{-1}x_1- b_n +a_1^{-1}b_1) \geq 0$ and for all $i \in \llbracket 2,n-1 \rrbracket,$
$\val(x_i-a_1^{-1}a_i(x_1+b_1)+b_i) \geq 0.$

Then the previous computations prove that $p(\PP_2)\subset \Qc.$
Conversely, let $(x_1,\dots,x_{n-1}) \in \Qc.$
Let $x_n := -a_1^{-1}(x_1+b_1).$
Then
$x_1+a_1 x_n+b_1=0=O(1).$
Let $i \in \llbracket 2,n-1 \rrbracket,$
then $x_i+a_i x_n+b_i=x_i-a_1^{-1}a_i(x_1+b_1)+b_i=O(1)$
since $(x_1,\dots,x_{n-1}) \in \Qc.$
Finally, 
$x_n+b_n=-a_1^{-1}(x_1+b_1)+b_n=O(1)$ thanks to the inequality satisfied by $x_1.$
Hence $(x_1,\dots,x_n) \in \PP_2$
which means that $(x_1,\dots,x_{n-1}) \in p(\PP_2).$
Thus $p(\PP_2)= \Qc$ and $p(\PP_2)$ is a polyhedron,
and it has been proved in both cases.

Therefore,$p(\PP)$ is a polyhedron.

\end{proof}

We can finally prove that the image of polyhedron under a linear mapping is a polyhedron.
\begin{proposition}
Let $\PP \subset K^n$ be a polyhedron and $f: K^n \rightarrow K^m$
be a linear mapping. Then $f(\PP)$ is a polyhedron.
\end{proposition}

\begin{proof}

We assume that $\PP \neq \emptyset$, and we identify $f$ with its matrix in $\allmat_{n,n}(K)$.
As seen above, we can assume that $\PP$ is of the form $\left\lbrace x \in K^n : \val (l_i (x) ) \geq 0, \textrm{ for } i \in I\right\rbrace$ for some
finite sets $I \subset \N$ and $l_i \in K[x]_1$, for $i \in I$ .


Thanks to Definition \ref{smith_nf}, (the matrix of) $f$
can be written $f=Q^{-1} \Delta P$
with $P \in \GL_n(\OK),$ $Q \in \GL_m(\OK)$
and $\Delta = SNF(f)\in \allmat_{m,n}(K)$ of the form $\diag(\lambda_1, \ldots, \lambda_r, {\bf 0}_{\min(m,n)-r})$
  where ${\bf 0}_{\min(m,n)-r}$ is a null square matrix of size $\min(m,n)-r$ for $\lambda_1,\dots,\lambda_r$ non-zero and $r \in \N.$

  %Now, we turn our attention to the $\Delta$ part of $f$.
We have previously proven that if $f \in \GL_n(\OK)$ then $f(\PP)$ is a polyhedron.
It is also clear that
if $f$ is the immersion 
$K^l \rightarrow K^n$
$(x_1,\dots,x_l)\mapsto (x_1,\dots,x_l,0,\dots,0)$ 
and $\PP$ is a polyhedron in $K^n$,
then $f(\PP)$ is a polyhedron.


We now prove that for
$\Delta$ a Smith Normal Form
and $\PP$ a polyhedron,
$\Delta(\PP)$ is a polyhedron.
Then $\Delta$ is the composition
of the immersion 
$K^l \rightarrow K^n$
$(x_1,\dots,x_l)\mapsto (x_1,\dots,x_l,0,\dots,0),$ 
 the diagonal invertible mapping
$K^l \rightarrow K^l$,
$(x_1,\dots,x_l)\mapsto (\delta_1,\dots,\delta_l x_l)$,
and the projections
$(x_1,\dots,x_{l+1})\mapsto (x_1,\dots,x_{l}),$
 $(x_1,\dots,x_{l+2})\mapsto (x_1,\dots,x_{l+1}),$
 $\dots,$
 $(x_1,\dots,x_{n})\mapsto (x_1,\dots,x_{n-1}).$
 Thus $\Delta(\PP)$ is a polyhedron thanks to the previous 
 results on image of polyhedron.

Finally, if $f: K^n \rightarrow K^m$
is a linear mapping and
$f=Q^{-1} \Delta P$ is its Smith Normal Form
with $P \in \GL_n(\OK),$ $Q \in \GL_m(\OK)$
and $\Delta \in \allmat_{n,m}(K)$,
then
$P(\PP)$ is a polyhedron,
$\Delta (P(\PP))$ also,
and finally $f(\PP)=Q^{-1} (\Delta (P(\PP)))$
is a polyhedron.
This concludes the proof.
\end{proof}


%%%%% old section by Corentin
%%%%%\subsection{Linear programming}

A \emph{linear programming problem} (or simply \emph{linear program}) is given by the following
optimization problem:
\begin{equation}
  \tag{LP}\label{LP}
\begin{array}{rcll}
  p^* & := & \inf_{x \in K^n} & \val(\left\langle c, x \right\rangle) \\
  &    & \text{s.t.}         & A x + b \geq 0
\end{array}
\end{equation}
with $c \in K^n$ a fixed vector of size $n$ that represents the cost, $A \in \allmat_{m,n}(K)$, $b \in K^m$,
and $\left\langle c, x \right\rangle := c_1x_1+\cdots +c_nx_n$.

We present an algorithm for solving \eqref{LP} which consists in reducing the matrix $A$ in Smith normal form.
The most interesting aspect in the SNF regarding the resolution of \ref{LP} is that the transition matrices are invertible in $\OK$. Indeed, if $z \in K^n$ and $M \in \allmat_n(\OK)$ such that $z \geq 0$ then $Mz \geq 0$. The latter property is a direct consequence of $\OK$ being a ring and the converse holds if the matrix $M$ is invertible in $\OK$. Therefore, for all matrix $M \in \allmat_n(K)$ with SNF $S$ and all $z \in K^n$, then $Mz \geq 0$ if and only if $Sz \geq 0$.
 

That property applied to the matrix $A$ in \ref{LP} allows to reduce the problem to an equivalent simpler iteration of the problem in \eqref{LP}, that will be called \emph{diagonal linear programming problem}
%%%\corentin{or maybe reduced linear programming problem?}:
%%%\simone{il me semble que tout ça n'est pas trop clair, par exemple quelle est la difference entre diagonal LP et associated diagonal LP plus en bas, est-ce qu'il y a vraiment besoin de tout ça, il faut simplifier la presentation}
\begin{equation}
  \tag{dLP}\label{DLP}
\begin{array}{rcll}
  {p^*}' & := & \inf_{y \in K^n} & \val(\left\langle c', y \right\rangle) \\
  &    & \text{s.t.}         & S y + b' \geq 0
\end{array}
\end{equation}
with $c' \in K^n$ a fixed vector of size $n$, $b' \in K^m$ and $S$ a diagonal $m \times n$ matrix of the form $S = \diag(s_1, s_2, \ldots, s_r, { \bm 0}_{m-r})$, with $r = {\rm rank}\, S$.

%The \emph{associated diagonal linear programming problem} of a linear program is the iteration of the \ref{DLP} problem defined by $b' = Qb$, $c' = \left(P^{-1}\right)^T c$ and where $S$ is the Smith normal form of $A$ and $P \in GL_m(\OK)$ and $ Q \in GL_n(\OK)$ are such that $A = Q^{-1} S P$.

A vector $x \in K^n$ is \emph{feasible} for \eqref{LP} if $Ax + b \geq 0$. Similarly, a vector $y \in K^n$ is feasible for \eqref{DLP} if $Sy+b' \geq 0$. We denote the \emph{feasible set} of \eqref{LP} and \eqref{DLP} respectively by
\begin{equation*}
\begin{aligned}
  \PP  &= \{x \in K^n : Ax + b \geq 0\} \\
  \PP' &= \{y \in K^n : Sy + b' \geq 0\}.
\end{aligned}
\end{equation*}
A feasible vector $x^* \in \PP$ is called a \emph{solution} to \eqref{LP} if $p^* = \left\langle c,x^*\right\rangle$
(similarly for \eqref{DLP}).
%%%$Adm$ \corentin{Probablement à changer mais je ne sais pas quel nom lui donner} the set of feasible solution of \ref{DLP}.

\begin{proposition} 
  Let $A \in \allmat_{m,n}(K), b \in K^m, c\in K^n$ and let $S \in \allmat_{m,n}(K)$ be the Smith normal form of $A$ with transition matrices $P \in \GL_n(\OK)$ and $Q\in \GL_m(\OK)$ such that $A=Q^{-1}SP$.
  We define $b':= Qb$ and $c':= (P^{-1})^Tc$.  Then
   \begin{enumerate}
    \item $x^* \in K^n$ is feasible for \eqref{LP} if and only if $y^* = P x^* \in K^n$ is feasible for \eqref{DLP}.
    \item $x \in K^n$ is a solution of the \ref{LP} problem of parameters $A, b$ and $c$ if and only if $y = P x$ is a solution of the \ref{DLP} problem of parameters $S,b'$ and $c'$ and $\langle c,x \rangle = \langle c',y \rangle$.
   \end{enumerate}
\end{proposition}
\begin{proof}
  First of all, 1. is a direct consequence of the fact that $Ax^*+b = AP^{-1}y^* +b\geq 0$ if and only if $Q(AP^{-1}y^* + b) = Sy^* + b' \geq 0$ because $Q \in GL_m(\OK)$.

  Then, 2. comes from the fact that for all $c \in K^n$ $\langle c, x\rangle = \langle c, P^{-1}Py \rangle = \langle \left(P^{-1}\right)^T c, y \rangle$.
\end{proof}

The last proposition allows us to consider only the case of the \ref{DLP} problem to solve \ref{LP}.
\begin{proposition}\label{prop:reduc}
		\begin{enumerate}
		\item \ref{DLP} have feasible solutions if and only if the $m-r$ last coefficients of $b'$ have nonegative valuation, for $r$ the rank of $S$.
		\item If at least one of the $n-r$ last coefficients of $c'$ is non-zero, if \ref{DLP} admits feasible solutions it is unbounded and does not have a solution.
	 
  \end{enumerate}
\end{proposition}
	
\begin{proof}
  A vector $y =  \left[\begin{smallmatrix} y_1 \\ \vdots \\ y_n \end{smallmatrix}\right] \in K^n$ is feasible solution of \ref{DLP} if and only if $\val(s_i y_i + b_i) >=0$ for all $i = 1...n$ \corentin{préciser les notations}. 
  However, $s_i =0$ for $n-r \leq i\leq n$. Therefore, \ref{DLP} have solution only if the $m-r$ last coefficient of b' have nonnegative valuation.
  
  Conversely, if the $m-r$ last coefficient of $b'$ are zero, the vector $y \in K^n$ such that $y_i = -\frac{-b'_i}{s_i}$ if $1 \leq i \leq r$ and $y_i =0$ otherwise is a feasible solution.

  Finally, let $y^*$ be a feasible solution of \ref{DLP}. Let us suppose there exists $r+1 \leq i \leq m$ such that $c'_i \neq  0$ and fix such $i$. Let us then consider the vector $e^i$ whose only non-zero coefficient is the $i-$th, which is $1$. Then $y^*+\alpha e^i$ is a feasible solution for all $\alpha \in K$ and $\langle c', y^* + \alpha e^i\rangle = \langle c' , y^* \rangle + \alpha c'_i$. In particular, for $\alpha = -\frac{\langle c', y^* \rangle}{c'_i}$, $\val \left( \langle c', y^* + \alpha e ^i \rangle\right) = \val (0) = + \infty$. 
\end{proof}

From now on, we will only consider the iterations of \ref{DLP} which have solutions, by \ref{prop:reduc}, we can then reduce the problem by only taking acount only the $r$ first coefficients of $y, b', c'$ and the submatrix of $S$ made with only $r$ first lines et columns of and whose coefficients are exactly the nonzero invariants factors. 
The set $Adm$ of the feasible solutions can then be written as the set of vectors $y \in K^n$ that verifies $ \ s_i y_i + b'_i \in \OK$ for $i=1,...,r$ {\it i.e.} $Adm = \left\{\left[\begin{smallmatrix} y_1\\ \vdots\\ y_n \end{smallmatrix}\right] \in K^n :\ y_i \in -\frac{b`_i}{s_i} + \frac{1}{s_i} \OK, 1\le i\le r\right\}$.


Resolving \ref{DLP} consists (by definition) in minimizing $y \mapsto \val\left(\left<c',y \right>\right)$ over $Adm$. The image of $Adm$ by $y \mapsto \left<c',y \right>$ is $\sum_{i=1}^r -c'_i.\frac{b'_i}{s_i} + \sum_{i=1}^r\left( \frac{c'_i}{s_{i}} \OK \right)$ which can be expressed as $\left<c',Adm \right> = \lambda + \pi^{v} \OK$ with $\lambda = \sum_{i=1}^r -c'_i.\frac{b'_i}{s_i}$ and $v = \min\limits_{1\le i\le r} \val \frac{c'_{i}}{s_{i}} $.
Finally, two cases can occur depending on $v$ and the valuation of $\lambda$. If $\val( \lambda) < v$ the minimum of $y\mapsto \val\left(\left<c',y \right>\right)$ over $Adm$ is reached in any point of $Adm$ and is equal to $\val\left( \lambda\right)$. Whereas, if $\val\left( \lambda \right) \ge v$ then $\lambda \in \pi^{v} \OK$ and the minimum is $v $ and is reached in any points $y$ of $Adm$ that verifies 
	\[\val \underset{ \begin{array}{c} 1\le i\le r\\ \val\left(c'_{i}/{s_{i}} \right) = v  \end{array}}{\sum} y_{i} = 0. \]

\begin{remark}
	To maximize the valuation of a linear map over a convex $p$-adic polyhedron instead of minimizing it (which is equuivalent to minimize the absolute value) the same reasoning can be aplied. The only difference being that if $\val\left( \lambda\right) < v$ the problem is unbounded and as such does not have any solution. 
\end{remark}


\subsection{Linear programming}


A \emph{linear programming problem} (or simply \emph{linear program}) \emph{with data $(A,b,c,D,e)$} is defined
as:
\begin{equation}
  \tag{LP}\label{LP}
\begin{array}{rcll}
  p^* & := & \inf_{x \in K^n} & \val(\left\langle c, x \right\rangle) \\
  &    & \text{s.t.}         & A x + b \geq 0\\
  & & & D x = e
\end{array}
\end{equation}
with $c \in K^n$ a fixed vector of size $n$ that represents the cost, $A \in \allmat_{d,n}(K)$, $b \in K^d$, $D \in \allmat_{m,n}$, $e \in K^m$ 
and $\left\langle c, x \right\rangle := c_1 x_1+\cdots +c_n x_n$. When $A$ is a diagonal matrix and $D = 0$, we often
refer to \eqref{LP} as a \emph{diagonal} linear program. We denote the \emph{feasible set} of \eqref{LP} by
\begin{equation*}
\begin{aligned}
  \PP  &= \{x \in K^n : Ax + b \geq 0, Dx = e\}.
\end{aligned}
\end{equation*}
A vector $x \in K^n$ is called \emph{feasible} for~\eqref{LP} if $x \in \PP$. A feasible vector $x^* \in \PP$
is called a \emph{solution} to \eqref{LP} if $p^* = \left\langle c,x^*\right\rangle$.

\begin{remark}
The valuation $\val$ is discrete. Thus if \eqref{LP} admits feasible vectors, $p^*$ is either $- \infty$ or reached by a certain solution $x^*$.
\end{remark}

In this section we present an algorithm for solving \eqref{LP} which starts by eliminating the constraint forced by $D$ and $e$ by operating a substitution. Then it reduces the matrix $A$ to
Smith Normal Form, thus reducing a general linear program to a diagonal linear program. The latter problem being easily solved by simple calulations in $K$.

The first proposition we prove  allows to consider only iterations of~\eqref{LP} with data of the form $(A,b,c,0,0)$.

\begin{proposition}\label{reducD}
  Let $A \in \allmat_{d,n}(K), b \in K^d, c\in K^n$, $D \in \allmat_{m,n}$, $e \in K^m$ and let $J \in \allmat_{n,R}\left( K \right) $ be a matrix such that ${\rm Im } J = \ker D$ where $R$ denotes the dimension of $\ker D$. 
  Additionally, we assume the existence of $x_0 \in K^n$ that verifies $D x_0 =e$ as if
  there are no such vector the problem~\ref{LP} with data $(A,b,c,D,e)$ is not solvable.

  Let $x \in \{x \in K^n : Dx=e\} = x_0 + \ker D $ and let $y \in K^R$ denote the unique element of
  $K^n$ such that $x = x_0 + J y$, then :
  \begin{enumerate}
    \item\label{reducD_item1} $x$ is feasible for~\eqref{LP} with data $(A,b,c,D,e)$ if and only if 
      $y$ is feasible for~\eqref{LP} with data $(AJ,b+A x_0,J^T c,0,0)$.
    \item\label{reducD_item2} $x$ is a solution for~\eqref{LP} with data $(A,b,c,D)$ if and only if 
      $y$ is a solution for~\eqref{LP} with data $(AJ,b+A x_0,J^T c,0,0)$ and $\langle c,x^* \rangle = \langle J^T c,y^* \rangle$.
  \end{enumerate}
\end{proposition}

\begin{proof}
  %The existence and uniqueness of $y$ are granted by the fact that $x \in x_0 + \ker D =x_0+ {\rm Im } J$  and $\ker J = \{0\}$. 
  First,  $Dx=e$ and $Ax + b \le 0$ if and only if $AJy + b+Ax_0 \le 0$, proving \Cref{reducD_item1}.
  Next, for all $c \in K^n$, $\langle c,x\rangle = \langle c, Jy\rangle = \langle J^T c,y\rangle$, which proves \Cref{reducD_item2}.
\end{proof}
 
The most interesting aspect of the SNF regarding the resolution of~\eqref{LP} is that the transition matrices are
invertible in $\OK$ and as such they preserve inequalities. %Indeed, for all $z \in K^n$ and $M \in \allmat_n(\OK)$ if $z \geq 0$, then $Mz \geq 0$.
%The latter property is a direct consequence of $\OK$ being a ring and the converse holds if the matrix $M$ is
%invertible in $\OK$. 
Therefore, for all matrix $M \in \allmat_{n,n}(K)$ with $S=SNF(M)$ and for all $z \in K^n$,
then $Mz \geq 0$ if and only if $Sz \geq 0$.

\Cref{solsLP} allows us to reduce \eqref{LP} to diagonal linear programs.

\begin{proposition} \label{solsLP}
  Let $A \in \allmat_{d,n}(K), b \in K^d, c\in K^n$ and let $S = SNF(A) = QAP^{-1} \in \allmat_{d,n}(K)$, with
  transition matrices $P \in \GL_d(\OK)$ and $Q\in \GL_d(\OK)$. Define $b' = Qb$ and $c' = (P^{-1})^Tc$.
  Then
   \begin{enumerate}
   \item \label{solsLP_item1}
     $x \in K^n$ is feasible for \eqref{LP} with data $(A,b,c,0,0)$ if and only if $y = P x \in K^n$ is
     feasible for \eqref{LP} with data $(S,b',c',0,0)$.
   \item \label{solsLP_item2}
     $x^* \in K^n$ is a solution to \eqref{LP} with data $(A,b,c,0,0)$ if and only if $y^* = P x^*$ is
     a solution to \eqref{LP} with data $(S,b',c',0,0)$ and $\langle c,x^* \rangle = \langle c',y^* \rangle$.
   \end{enumerate}
\end{proposition}
\begin{proof}
  First of all, $Ax+b = AP^{-1}y +b \geq 0$ if and only if $Q(AP^{-1}y + b) = Sy + b' \geq 0$ because
  $Q \in GL_m(\OK)$, which yields \Cref{solsLP_item1}. Next, for all $c \in K^n$, we have $\langle c,
  x\rangle = \langle c, P^{-1}Px \rangle = \langle \left(P^{-1}\right)^T c, Px \rangle = \langle c',y\rangle$,
  which proves \Cref{solsLP_item2}.
\end{proof}


\begin{proposition}\label{prop:reduc}
  Let $(A,b,c), (S,b',c') \in \allmat_{d,n}(K) \times K^d \times K^n$ be as in \Cref{solsLP}, where
  $S=SNF(A)$. Then
  \begin{enumerate}
  \item \label{prop:reduc_it1}
    \eqref{LP} with data $(S,b',c',0,0)$ is feasible if and only if the $d-r$ last coefficients
    of $b'$ have nonnegative valuation, for $r = {\rm rank}\,S$.
  \item \label{prop:reduc_it2}
    Assume at least one of the $n-r$ last coefficients of $c'$ is non-zero. If \eqref{LP} with
    data $(S,b',c')$ is feasible, then it is unbounded and does not admit solutions.
  \end{enumerate}
\end{proposition}
\begin{proof}
  %%%%\corentin{préciser les notations}. 
  A vector $y = (y_1, \ldots, y_n)^T \in K^n$ is feasible for \eqref{LP} with data $(S,b',c',0)$
  if and only if $\val(s_i y_i + b_i) \geq 0$ for all $i = 1,\ldots,n$.
  However $s_i = 0$ for $n-r \leq i\leq n$. Therefore, if \eqref{LP} with data $(S,b',c',0)$ is feasible,
  necessarily the $d-r$ last coefficients of $b'$ have nonnegative valuation.
  Conversely, if the $d-r$ last coefficients of $b'$ are zero, the vector $y \in K^n$ such that
  $y_i = -\frac{-b'_i}{s_i}$ if $1 \leq i \leq r$, and $y_i =0$ otherwise, is feasible for \eqref{LP}
  with data $(S,b',c',0,0)$.

  Concerning \Cref{prop:reduc_it2}, let $y^*$ be feasible for \eqref{LP} with data $(S,b',c',0,0)$.
  Let $i$ be such that $r+1 \leq i \leq n$ and $c'_i \neq 0$. Consider the vector $e^i$ whose only non-zero
  coefficient is the $i-$th, which is $1$. Then $y^* + \alpha e^i$ is feasible for \eqref{LP} with data $(S,b',c',0,0)$,
  for all $\alpha \in K$ and $\langle c', y^* + \alpha e^i\rangle = \langle c' , y^* \rangle + \alpha c'_i$.
  In particular, for all $n \in \mathbb{N}$ if we set $\alpha = -\frac{\langle c', y^* \rangle + \pi^{-n}}{c'_i}$, we have $\val \left( \langle c', y^* + \alpha e ^i \rangle\right) = \val \pi^{-n} = -n$. Therefore, the problem is unbounded as $p^* = + \infty$. 
\end{proof}

From now on, we will only consider iterations of diagonal linear programs which have solutions.
By \Cref{prop:reduc} we can then reduce the problem by only taking account the first $r$ coefficients of
$y, b', c'$ and the submatrix of $S$ made with only by the $r$ first rows and columns and whose coefficients
are exactly the nonzero invariants factors.

The feasible set $\PP' = \{y \in K^n : Sy+b' \geq 0\}$ of the diagonal LP can be written as the set of vectors
$y \in K^n$ that verifies $s_i y_i + b'_i \in \OK$ for $i=1,...,r$ {\it i.e.}:
$$
\PP' = \left\{\left[\begin{smallmatrix} y_1\\ \vdots\\ y_n \end{smallmatrix}\right] \in K^n :
y_i \in -\frac{b`_i}{s_i} + \frac{1}{s_i} \OK, 1\le i\le r\right\}.
$$

Solving the diagonal \Cref{LP} consists -by definition- in minimizing the linear map $y \mapsto \val\left(\left\langle
c',y \right\rangle\right)$ over $\PP'$. The image of $\PP'$ by $y \mapsto \left\langle c',y \right\rangle$ is
\[
\left\langle c',\PP' \right\rangle = \sum_{i=1}^r -c'_i \frac{b'_i}{s_i} + \sum_{i=1}^r\left( \frac{c'_i}{s_{i}} \OK \right) = \lambda + \pi^{v} \OK
\]
where $\lambda = \sum_{i=1}^r -c'_i \frac{b'_i}{s_i}$ and $v = \min\limits_{1\le i\le r} \val \frac{c'_{i}}{s_{i}} $.

Two cases can occur depending on $v$ and the valuation of $\lambda$.
If $\val(\lambda) < v$, the minimum of $y\mapsto \val\left(\left\langle c',y \right\rangle\right)$ over $\PP'$
  is reached at any point of $\PP'$, and is equal to $\val(\lambda)$.
  Otherwise, if $\val(\lambda) \ge v$, then $\lambda \in \pi^{v} \OK$ and the minimum is $v$ and is reached at any points $y$
  of $\PP'$ such that $\left<c',y \right> \in \pi^v$.

\begin{remark}
	If $\val \lambda \ge v$ and $i_0 \in \{1,\ldots, n\} $ is such that $v = \val \frac{c_{i_0}'}{s_{i_0}²} $ 
  then the vector $e^{i_0} \in K$ is a solution of \ref{LP}, 
  where $e^{i_0}$ denotes the vector of $K^n$ whose only non-zero coefficient is the $i_0$-th which is $1$.
\end{remark}

\begin{remark}
To maximize the valuation of a linear map over a polyhedron instead of minimizing it 
(which is equivalent to minimize the absolute value) the same reasoning can be applied. 
The only difference being that if $\val\left( \lambda\right) < v$ the problem is unbounded and 
as such does not admit a solution. 
\end{remark}


\section{Spectrahedra}

\subsection{Positive semidefinite matrices}
%\newcommand\mat{postive semidefinite matrix } 
\newcommand\Mat{Positive semidefinite matrix }
\newcommand\mats{positive semidefinite matrices }
\newcommand\Mats{positive semidefinite matrices }

\begin{definition}
  We say that a matrix $M \in K^{d \times d}$ is \emph{positive semidefinite} ($M \succeq 0$)
  if all its eigenvalues have nonnegative valuation in an algebraic closure of $K$.
  We denote by $K^{d \times d}_+ \subset K^{d\times d}$ the set of positive semidefinite matrices.
  The characteristic polynomial of a matrix $M \in K^{d\times d}$ is denoted $\chi_M \in K[T]$.
\end{definition}

Recall that the {\it Newton polygon} $\mathcal{P}$ of a polynomial $P = \sum_{i=0}^{d} a_i T^i \in K[T]$ is
defined as the inferior convex hull of the set $\{(i, \val a_i) \,|\, 0 \le i \le d\}$, {\it i.e.} the
graph of the greatest convex function $\mathcal{P} : [0,d] \to \R$ such that $\mathcal{P}(i) \leq \val a_i$
for $i=1,\ldots,d$. This function is piecewise affine and its slopes are exactly the opposites of the valuations
of the roots of $P$.

The following result shows that testing positive semidefiniteness of matrices over
valuation fields reduces to checking whether the coefficients of $\chi_M$ are integers.

\begin{theorem}
  \label{caracsdp}
  $M \in K^{d\times d}_+$ if and only if $\chi_M \in \OK[T]$.
\end{theorem}
\begin{proof}
  Both senses of the equivalence will be proven by contraposition.
  Let $M \in K^{d\times d}$ with characteristic polynomial $\chi_M = T^d + \sum_{i=0}^{d-1} a_i T^i$ for $a_0,...,a_{d-1} \in K$.
  First suppose $M \not \in K^{d\times d}_+$, thus $\chi_M$ has a root $\rho$ with $\val\,\rho<0$.
  It follows that
  $$
  + \infty = \val \chi_M (\rho) = \val \left(\rho^d+ \sum_{i=0}^{d-1} a_i \rho^i\right)
  $$ and $\val \rho^d = d \val \rho <0$.
  The properties of the non-archimedean inequality and of the valuation imply that
  $$
  0> \val \rho^d = \val\left(\sum_{i=0}^{d-1} a_i \rho^i\right) \geq
  \min_{0 \leq i \leq d-1} \val \left(a_i \rho^i\right)
  $$
  %%%$\val ( \sum_{i=0}^{d-1} a_i \rho^i) = \val \rho^d$ and $\val ( \sum_{i=0}^{d-1} a_i \rho^i) \ge \min_{i=0,..., d-1} \val \left(a_i \rho^i\right)$.
  Therefore, there exists $j \in \left\{0,...,d-1\right\}$ such that $\val(a_j \rho^j) \le \val\rho^d$,
  {\it i.e.} such that $\val a_j \le (d-j) \val \rho <0$. This shows that $\chi_M \not\in \OK[T]$.
  
  %%We have proven the converse sense of the equivalence.
  %%To prove the direct sense a brief introduction to Newton's polygon will be needed, we will only state the results we use in this proof and will redirect to \cite{gouvea_p-adic_2003} to further explanations.

  Suppose now $\chi_M \not\in \OK[T]$. There exists $j \in \{0,...,d-1\}$ such that $\val a_j <0$.
  The polynomial $\chi_M$ being monic, the point with abscissa $d$ of the Newton polygon of $\chi_M$ is
  $(d,\val(1)) = (d,0)$. 
  Therefore, the point $(j, \val a_j)$ having negative ordinate implies that at least one of the slopes of
  the Newton polygon of $\chi_M$ is positive, so that $\chi_M$ has a root with negative valuation.
\end{proof}

As a consequence, matrices with integer coefficients are positive semidefinite.

\begin{corollary}\label{cor_caracsdp}
  $\OK^{d\times d} \subset K^{d\times d}_+$.
\end{corollary}
\begin{proof}
  $\OK$ is a ring, therefore if $M \in \OK^{d \times d}$, then $\chi_M \in \OK[T]$.
\end{proof}

The reverse inclusion of \Cref{cor_caracsdp} is false in general as we show in the next example.

\begin{example}
  Let
  $$
  M = \begin{bmatrix} 5 + \frac{3}{5} & \frac{4}{5} \\ \frac{4}{5} & -\frac{3}{5} \end{bmatrix}.
  $$
  The characteristic polynomial of $M$
  is $\chi_M = T^2-5T-4$. However when seen as a matrix over $K=\mathbb{Q}_5$, then $\chi_M \in \mathbb{Z}_5[T] = \OK[T]$: then $M \in (\mathbb{Q}_5)^{2 \times 2}_+$, nevertheless $M \not\in (\mathbb{Z}_5)^{2\times 2}$, hence
  $(\mathbb{Q}_5)^{2 \times 2}_+ \not\subset (\mathbb{Z}_5)^{2 \times 2}$.
\end{example}

\begin{definition}
  A \emph{cone} of $K^d$ is a subset that is closed under multiplication by elements of $\OK$.   
\end{definition}

\begin{proposition}
%  {\color{blue} [Peut être une hypothèse à rajouter, par exemple $K$ à valuation discrete]}
  $K^{d \times d}_+$ is a cone.
  If $\val$ is discrete, then it is both open and closed.
\end{proposition}
\begin{proof}
  The fact that $K^{d \times d}_+$ is a cone is immediate from the definition.
  Next, remark that $\OK[X]$ is closed and open in $K[X]$ equipped with the infinity norm
  $$
  \|\cdot\|_\infty : P = \sum_{k=1}^{n} a_k X^k \to
  \max_{1 \leq k \leq n} |a_k|,
  $$
  {where $|\cdot|$ is any norm induced by $\val$}:
  $\OK$ is indeed the open unit ball for $\|\cdot \|_\infty$ that induces a discrete distance for
  which open balls are also closed sets.
  Furthermore, $K^{d\times d}_+ = \chi^{-1}(\OK[X]) $ where $\chi$ is the function that maps a
  matrix to its characteristic polynomial which is continuous.
  Thus $K^{d\times d}_+$ is both open and closed, too.
\end{proof}


%\begin{remark}
%  \corentin{Pas nécessairement très important, à voir}
%  If $C$ is a cone that is furthermore closed under addition (what shall be called a convex cone in the real case) then $C$ is a $\OK$-module.
%\end{remark}



\subsection{Definition of spectrahedra}

\begin{definition}
  \label{def_spectrahedra}
  A \emph{spectrahedron in $K^{d\times d}$} is a set of the form
  $$
  \mathcal{S} = \mathcal{L} \cap K^{d\times d}_+ = \left\{X \in K^{d \times d} : \exists \, x \in K^n, X = A_0+\sum_{x_i A_i}, X \succeq 0\right\}
  $$
  where $\mathcal{L} = A_0+\span{A_1,\ldots,A_n} \subset K^{d \times d}$ is an affine space, for some
  $A_0,A_1,\ldots,A_n \in K^{d\times d}$.
\end{definition}

\begin{remark}
  As in the real case, one often abuses of notation, saying that the preimage
  $S_A = \varphi_A^{-1}(K^{d \times d}_+) = \{x \in K^n : A_0+x_1A_1+\cdots+x_nA_n \succeq 0\}$
  of $K^{d \times d}_+$ under the map $\varphi_A : K^n \to \mathcal{L} \subset K^{d \times d}$, defined as
  $\varphi_A(x) = A_0+x_1A_1+\cdots+x_nA_n$, is a \emph{spectrahedron
  in $K^n$}. Remark that the map $\varphi_A$ is onto $\mathcal{L}$ but not injective,
  unless $A_1,\ldots,A_n$ are linearly indepedent.
\end{remark}

The class of spectrahedra in $K^{d \times d}$ (or in $K^n$) is closed under intersection: indeed one
has $(\mathcal{L}_1 \cap K^{d \times d}_+) \cap (\mathcal{L}_2 \cap K^{d \times d}_+) =
(\mathcal{L}_1 \cap \mathcal{L}_2) \cap K^{d \times d}_+$ and
$S_A \cap S_B = S_{\diag(A,B)}$, for two affine spaces $\mathcal{L}_1 = A_0+\span{A_1,\ldots,A_n}$
et $\mathcal{L}_2 = A_0+\span{B_1,\ldots,B_n}$.
Moreover this class contains polyhedra in $K^n$.

%\begin{remark}
%	As in the real case, the spectrahedron associated with the affine plan $\mathcal{L}$ will often be identified with its preimage in $K^s$ (i.e. the set of vectors $x \in K ^s$ so that the linear matrix $A(x) = A_0 + x_1A_1 + \ldots + x_sA_s$ is positive semidefinite with $A_0,A_1,\ldots,A_s$ spanning $\mathcal{L}$). 
%\end{remark}

\begin{proposition}
  %%% OLD STATEMENT
  %%% A polyhedron is a spectrahedron.
A polyhedron in $K^n$ is a spectrahedron in $K^n$.
\end{proposition}
\begin{proof}
  Let $P$ be a polyhedron in $K^n$. We show that there exist $d \in \mathbb{N}$ and
  $A_0,A_1,\ldots,A_n \in K^{d \times d}$ such that $P = S_A$. Assume $P = L \cap M$ with
  \begin{equation*}
    \begin{aligned}
    L &= \{x \in K^n : \val \ell_i(x) \geq 0, i=1, \ldots, d\} \\
    M &= \{x \in K^n : \val m_j(x) = +\infty, j=1, \ldots, e\}
    \end{aligned}
  \end{equation*}
  for some $d, e \in \mathbb{N}$. Let $\eta \leq n$ be the dimension of $M$ as vector subspace of
  $K^n$, and let $y=(y_1, \ldots, y_\eta) \in K^{\eta}$ be variables on $M$, with $y = Tx$
  for some $T \in K^{\eta \times n}$ with $\text{rank}(T) = \eta$. Let $A_0,A_1,\ldots,A_n \in K^{d \times d}$
  be such that
  $$
  A_0+x_1A_1+\cdots+x_nA_n = \diag(\ell_1(x), \ldots, \ell_d(x)).
  $$
  One has that
  $$
  P = \{x \in \}
  $$
  {\bf à terminer -- Simone}  
%%% OLD PROOF
%  Let $\ell_i,m_j \in K[x]_1$, for $i=1,\ldots,d$ and $j=1,\ldots,e$, and let $P = Q \cap L \subset K^n$
%  be a polyhedron as in \Cref{def_polyhedra}, for $Q = \{x \in K^n : \val(\ell_i) \geq 0, i=1,\ldots,d\}$
%  and $L = \{x \in K^n : \val(m_j) = +\infty, j=1,\ldots,e\}$.
%  First, denote by $A(x) = \ell_1 \oplus \cdots \oplus \ell_d \in K[x]^{d\times d}_1$,
%  so that $A(x) = A_0+x_1A_1+\cdots+x_nA_n$ for some (diagonal) $A_i \in K^{d\times d}$.
%  We might assume that $A_0,A_1,\ldots,A_n$ are affinely independent, so that the mapping
%  $\varphi(x) = A(x)$ is injective. 
%  First, remark that $Q=S_A$ and thus $\varphi(Q) = \mathcal{L} \cap K^{d\times d}$, with
%  $\mathcal{L} = A_0+\span{A_1,\ldots,A_n}$ is a spectrahedron.
%  Next, remark that $L \subset K^n$ is the affine space defined by equations $m_1=\ldots=m_e=0$,
%  hence $\varphi(L) \subset K^{d\times d}$ is an affine space. Since $\varphi$ is injective,
%  one has $\varphi(P) = \varphi(Q \cap L) = \varphi(Q) \cap \varphi(L)$ which is the intersection
%  of a spectrahedron with an affine space, thus a spectrahedron.
%  %Define $B = m_1 \oplus \cdots \oplus m_e \in K[x]^{e\times e}_1$
\end{proof}

\begin{definition}
  A subset $S$ of $K^n$ is said \emph{semialgebraic} if it is defined by finite unions and intersections of sets defined by polynomial equalities of the form $\{ x \in K^n  : P( x) = 0 \}$ and of sets defined by polynomial inequalities of the form $\{ x\in K^n : \val P(x) \geq 0 \}$.
\end{definition}

\begin{proposition}
  Spectrahedra in $K^n$ are semialgebraic sets. 
\end{proposition}

\begin{proof}
  Let $S = S_A \subset K^n$ be spectrahedron, for some linear matrix $A(x) = A_0+\sum_{i=1}^n x_i A_i$. 
  Denote by $p_i(x) \in K[x]$ the $i$-th coefficient of $\chi_{A(x)}$, $i=0,\ldots,d$.
  By \Cref{caracsdp} one has:
  $$
  S = \left\{x \in K^n : \chi_{A(x)} \in \OK[T]\right\}
  = \bigcap_{i=0}^d \{x \in K^n : \val(p_i(x)) \geq 0\}.
  $$
\end{proof}

%\corentin{ Une autre définition qu'il pourrait être intéressant d'explorer. Attention! Pas les mêmes propriétés.}
%\begin{definition}[Alternative definition]
%  Let $A_0, ..., A_n, B_0, ..., B_n \in M_{d,d}(K)$.  We denote as $S^+_A$ the set $S_{A}^+ = \left\{ x \in K^n : A_0 + x_1 A_1+ ...+x_nA_n \succeq 0  \right\}$ and $S^-_B$ denotes the set $S^-_B = \left\{ x \in K^n : B_0 + x_1 B_1+ ...+x_nB_n \preceq 0\right\}$.   The set $S_A,B := S_A^+ \cup S_B^-$ is called a \emph{spectrahedron}.
%\end{definition}

\subsection{Annuli}

$\R^*_+$ denotes the set of positive real numbers.

\begin{definition}[Annulus]\label{def_annuli}
  A set of the form $A = \{x \in K \,|\, a \le \val(x) \le b\}$ for $a,b \in \R^*_+$ is called \emph{annulus}.
\end{definition}

\begin{theorem}
Annuli are spectrahedral shadows.
\end{theorem} 
\begin{proof}
  Let $A$ be as in \Cref{def_annuli}, for some $a,b \in \R^*_+$, assume $a<b$. Consider the spectrahedron
  $\mathcal{S} \subset K^2$ defined by
  $$
  \mathcal{S}= \left\{(x,y) \in K^2 : M(x,y) := \diag\left(\pi^ax, \pi^{-b}y, \begin{bmatrix} \pi^{-1} & -\pi^{-1}x \\ \pi^{-1}y & - \pi^{-1}  \\ \end{bmatrix}\right) \succeq 0\right\}.
  $$
We will show that $x \in A$ if and only if  there exists $y\in K$ such that $(x,y) \in \mathcal{S}$.
Let $x,y \in K$. By \Cref{caracsdp}, $M(x,y) \succeq 0$ if and only if $(x,y)$ is a solution of
\eqref{eq:annulussyst}:
\begin{equation}
  \label[type]{eq:annulussyst}
\begin{cases}
  \val(\pi^ax) \ge 0 \\
  \val(\pi^{-b}y) \ge 0\\
  \val(\pi^{-1}- \pi^{-1}) = +\infty \ge 0\\
  \val(\pi^{-2}\left( xy-1 \right))  \ge 0 
\end{cases}
\iff	
\begin{cases} 
		\val\left(x\right)\ge a\\
		\val\left(y\right)\ge -b\\
		\val\left( xy-1 \right) \ge 2
	\end{cases}
\end{equation}
Therefore, if $(x, y)$ verifies \eqref{eq:annulussyst} then $\val\left(x\right)\ge a$, $\val\left(y\right)\ge -b$ and $ \left( xy-1 \right) \ge 2$. The last inequality implies that $\val\left(x\right)+\val\left(y\right)=\val\left(xy\right) = \val\left(-1\right) =0$ and therefore that $\val\left(x\right)=-\val\left(y\right)\le -(-b)  =b$, so $x \in C$. 
Reciprocally if $x \in C$ then $(x,x^{-1}) $ is a solution of \eqref{eq:annulussyst}.
\end{proof}


%\subsection{Polyhedrality}
%Cf. \cite{bhardwaj2015deciding}

%\section{Linear matrix inequalities}


%\section{Other questions}
%\begin{itemize}
%\item The projection of a real polyhedron is a real polyhedron (thanks to Fourier-Motzkin elimination procedure). Is it the same for $K$-polyhedra?
%\item Annuli are semidefinite representable, we should remark/prove that they are not spectrahedra.
%\end{itemize}


\section*{Acknowledgments}

The second author acknowledges support from the ANR Project ``HYPERSPACE'' (ANR-21-CE48-0006-01).


\printbibliography

\end{document}
