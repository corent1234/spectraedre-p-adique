\documentclass[a4paper,12pt]{article}

\newenvironment{proof}{\hbox{}\vspace{-0.5cm} {\bf Proof:}}{\hfill $\Box$ \\}

\newtheorem{theorem}{Theorem}
\newtheorem{lemma}{Lemma}
\newtheorem{proposition}{Proposition}
\newtheorem{corollary}{Corollary}
\newtheorem{algorithm}{Algorithm}
\newtheorem{conjecture}{Conjecture}
\newtheorem{condition}{Condition}
\newtheorem{definition}{Definition}
\newtheorem{assumption}{Assumption}
\newtheorem{remark}{Remark}
\newtheorem{problem}{Problem}
\newtheorem{example}{Example}
\usepackage{booktabs}

\textheight235mm
\textwidth165mm
\voffset-10mm
\hoffset-12.5mm
\parindent0cm
\parskip2mm

\usepackage{amsmath}
\usepackage{amssymb}
\usepackage{amsfonts}
\usepackage{mathrsfs} 
\usepackage{graphicx}
\usepackage{color}
\usepackage{ulem}
\usepackage{bm}

\usepackage{cleveref}

\usepackage{stmaryrd} %for double brackets


% updates

\newcommand{\R}{\mathbb{R}} % real numbers
\newcommand{\C}{\mathbb{C}} % complex numbers
\newcommand{\N}{\mathbb{N}} % integers
\newcommand{\sym}{\mathbb{S}} % matrices symétriques
\renewcommand{\span}[1]{{\text{span}(#1)}} % simone's comments
\newcommand{\calL}{\mathcal{L}} % simone's comments

\newcommand{\simone}[1]{{\color{blue} #1}} % simone's comments
\newcommand{\corentin}[1]{{\color{red} #1}} % corentin's comments
\newcommand{\tristan}[1]{{\color{olive} #1}} % tristan's comments

%p-adic commands
\DeclareMathOperator{\val}{val}
\def\QQ{\ensuremath{\mathbb{Q}}}
\def\ZZ{\ensuremath{\mathbb{Z}}}
\newcommand{\OK}{\mathcal{O}_K}


\usepackage{biblatex}
\addbibresource{../rapport/bibstage.bib}

\title{\bf Title}

\begin{document}

\author{Corentin Cornou$^{1}$, Simone Naldi$^2$ and Tristan Vaccon$^2$}

\footnotetext[1]{ENS Paris-Saclay, Université Paris-Saclay, France.}
\footnotetext[2]{Université de Limoges, CNRS, XLIM, Limoges, France.}

\date{Draft of \today}

\maketitle

\begin{abstract}
\end{abstract}


\tableofcontents


\section{Introduction}

\subsection{Context and motivations}

Il faudra entre autre mentionner (et peut être motiver l'approche p-adique) :
\begin{itemize}
\item les questions ouvertes de complexité concernant la programmation semidefinie (pas clair si NP $\cap$ co-NP
  dans le modele de Turing)
\item les questions ouvertes géométriques (Conjecture de Lax)
\end{itemize}


\subsection{Main results}



\section{Preliminaries}

\subsection{Notation}

Throughout the paper, $K$ refers to a complete,
discrete valuation field, $\val : K \twoheadrightarrow \ZZ \cup \{+\infty\}$ to its valuation,
$\OK$ its ring of integers and $\pi$ a uniformizer.
For $k \in \N$, we write $O(\pi^k)$ for $\pi^k \OK.$
A typical example of $K$ as above is the field of $p$-adic numbers 
$\QQ_p$ (equipped with the $p$-adic valuation). For this example, we 
have $\OK = \ZZ_p$.
Another example is the field of Laurent series
$K=\QQ ((T))$ equiped with $T$-adic valuation.
In this case,  $\OK = \QQ \llbracket T \rrbracket$.
We refer to Serre's Local Fields \cite{Serre} for a
general introduction to such fields
and to Caruso's course on $p$-adic computations \cite{caruso_computations_2017}
for an introduction to effective computations
over them

\subsection{Real spectrahedra}

Let $\sym_m(\R)$ be the vector space of $m \times m$ real symmetric matrices. A matrix $M \in \sym_m(\R)$
is called {\it positive semidefinite} ($M \succeq 0$) whenever the quadratic
form $x \mapsto x^TMx$ is nonnegative for all $x\in \R^n$. By the Spectral Theorem, $M \succeq 0$
if and only if the eigenvalues of $M$ are nonnegative, and this is equivalent to all the principal minors
of $M$ being nonnegative. The set of positive semidefinite matrices, denoted $\sym_m^+(\R) \subset \sym_m(\R)$,
is a closed convex cone with non-empty interior in $\sym_m(\R)$.

Let $A_0,A_1,\ldots,A_n \in \sym_m(\R)$, and let $\calL = A_0+\span{A_1,\ldots,A_n}$ be the affine space
containing $A_0$ and with direction the vector space spanned by $A_1,\ldots,A_n$ (which we assume linearly
independent). The intersection
$\calL \cap \sym_m^+(\R) = \left\{A(x) := A_0+\sum_i x_i A_i \in \sym_m(\R) : A(x) \succeq 0\right\}$, or
equivalently, its pre-image $S = \{x \in \R^n : A(x) \succeq 0\}$ under the map $x \mapsto A(x)$, is called
a {\it real spectrahedron}. As affine sections of $\sym_m^+(\R)$, real spectrahedra are closed convex sets,
but might have empty interior in $\calL$ (resp. in $\R^n$). Moreover, they are basic semialgebraic sets,
defined by sign conditions on the principal minors of the defining matrix as well as sign conditions on the
coefficients of its characteristic polynomial (cf. \Cref{ell3}).

\begin{example}\label{ell3}
  The $3$-elliptope is the three-dimensional spectrahedron given by
  $$
  \mathcal{E}_3 :=
  \left\{
  x
  \in \R^3 :
  \begin{bmatrix}
    1 & x_1 & x_2 \\
    x_1 & 1 & x_3 \\
    x_2 & x_3 & 1
  \end{bmatrix}
  \succeq 0
  \right\}
  =
  \left\{
  x
  \in \R^3 :
  \begin{array}{r}
    3-x_1^2-x_2^2-x_3^2 \geq 0 \\
    1+2x_1x_2x_3-x_1^2-x_2^2-x_3^2 \geq 0
  \end{array}
  \right\}.
  $$
  The polynomials on the right are the (non-constant) coefficients of the univariate polynomial
  $p(t) = \det(A(x)+t I)$,
  whose roots are the opposites of the eigenvalues of the defining matrix. Equivalently $\mathcal{E}_3$
  is defined by positivity of the principal minors of the defining matrix:
  $$
  1-x_1^2 \geq 0, \,\,\, 1-x_2^2 \geq 0, \,\,\, 1-x_3^2 \geq 0 \,\,\, \text{ and } \,\,\, 1+2x_1x_2x_3-x_1^2-x_2^2-x_3^2 \geq 0
  $$
\end{example}

The optimization problem of miminizing a linear function over a real spectrahedron is called {\it semidefinite programming} (SDP). Given $C,A_0,A_1,\ldots,A_n \in \sym_m(\R)$, the corresponding semidefinite program is defined as
\begin{equation}
  \label{SDP}
\begin{array}{rcll}
  p^* & := & \inf_{x \in \R^n} & \left\langle c, x \right\rangle \\
  &    & \text{s.t.}         & A_0+\sum_{i=1}^n x_i A_i \succeq 0.
\end{array}
\end{equation}
Since real polyhedra are special cases of real spectrahedra, SDP is a generalization of linear programming (LP).
In a fixed-precision model of computation, SDP is solvable in polynomial time in the input size (matrix size,
number of variables, bit size of the coefficients), in $\log(\frac{1}{\varepsilon})$ (where $\varepsilon$ is the
precision) and $\log(\frac{1}{R})$, where $R$ is an {\it a priori} upper bound on the norm of the solution,
cf. \cite[Sec.1.9]{deKlerk}. In exact arithmetic, the complexity status of SDP is essentially open but complexity
bounds are known based either on general quantifier elimination \cite{ramana1997exact,porkolab1997complexity}
or on the determinantal structure of the constraints \cite{henrion2016exact}.

A more general class of convex semialgebraic sets are obtained by linear projections of real spectrahedra
called {\it semidefinite representable sets} (or spectrahedral shadows): these are sets of the form
$$
P = \left\{x = \left[\begin{smallmatrix} x_1 \\ \vdots \\ x_n \end{smallmatrix}\right] \in \R^n : \exists\,y\in\R^m, \, A_0 + \sum_{i=1}^n x_i A_i + \sum_{j=1}^m y_j B_j \succeq 0\right\}
$$

\begin{example}
\label{fermat_quartic}
The basic closed semialgebraic set $P = \{(x_1,x_2) \in \R^2 : 1-x_1^4-x_2^4 \geq 0\}$ is not a spectrahedron
but it is semidefinite representable:
$$
P = \left\{x = \begin{bmatrix} x_1 \\ x_2 \end{bmatrix} \in \R^2 :
\exists\, (y_1,y_2) \in \R^2, \,
\begin{bmatrix}
  1+y_1 & y_2 \\
  y_2 & 1-y_1
\end{bmatrix}
\bigoplus
\begin{bmatrix}
  1 & x_1 \\
  x_1 & y_1
\end{bmatrix}
\bigoplus
\begin{bmatrix}
  1 & x_2 \\
  x_2 & y_2
\end{bmatrix}
\succeq 0
\right\}
$$
\end{example}


\section{Polyhedra and linear programming}

\subsection{Polyhedra}
\subsection{Linear programmming}

\section{Spectrahedra}

\subsection{Definition}
\subsection{Annuli (couronnes)}
\subsection{Polyhedrality}
Cf. \cite{bhardwaj2015deciding}

\section{Linear matrix inequalities}


\printbibliography
\end{document}
